\documentclass[a4paper,12pt]{article}
\usepackage{tabularx}
\usepackage{amsmath}
\usepackage[utf8]{inputenc}
\usepackage{amsmath, amssymb, amsthm}
\usepackage{graphicx}
\usepackage{array}
\usepackage[left=2cm, right=2cm, top=2cm, bottom=2cm]{geometry}
\usepackage{fancyhdr}
\usepackage{xfp} % Für mathematische Berechnungen in LaTeX
\usepackage{pgf} % Erlaubt Berechnungen mit \pgfmathparse

% Kopf- und Fußzeile
\pagestyle{fancy}
\lhead{6. März 2025\\Vorbereitung 45min.}
\chead{Heinrich-von-Kleist-Schule}
\rhead{Mathematik - G6E}
\lfoot{}
%\cfoot{Seite \thepage}
\rfoot{}

% Makro für einzelne Punkteingaben (hier einfach Platzhalter)
\newcommand{\punkteA}{0}  % Eingetragene Punkte für Aufgabe 1
\newcommand{\punkteB}{0}  % Eingetragene Punkte für Aufgabe 2
\newcommand{\punkteC}{0}   % Eingetragene Punkte für Aufgabe 3
\newcommand{\punkteD}{0}   % Eingetragene Punkte für Aufgabe 4
\newcommand{\punkteE}{0}  % Eingetragene Punkte für Aufgabe 5

% Maximale Punktzahl insgesamt
\newcommand{\maxSumme}{100}  % Hier die maximale Punktzahl anpassen

% Automatische Berechnung der Notengrenzen
\newcommand{\noteEinsMin}{\fpeval{round(\maxSumme * 0.90,0)}}
\newcommand{\noteZweiMin}{\fpeval{round(\maxSumme * 0.75,0)}}
\newcommand{\noteDreiMin}{\fpeval{round(\maxSumme * 0.60,0)}}
\newcommand{\noteVierMin}{\fpeval{round(\maxSumme * 0.45,0)}}
\newcommand{\noteFunfMin}{\fpeval{round(\maxSumme * 0.20,0)}}
\newcommand{\noteSechsMin}{0}

% Automatische Berechnung der Summe
\newcommand{\summe}{%
	\pgfmathparse{\punkteA + \punkteB + \punkteC + \punkteD + \punkteE}%
	\pgfmathprintnumber{\pgfmathresult}
}


\begin{document}
	
	\begin{center}
		%\textbf{Klassenarbeit N°1 G8}\\[0.3cm]
		\textbf{Rechnen mit Brüchen}\\[0.2cm]
		%\textbf{© J. Mycan}\\[0.5cm]
	\end{center}
	
	\textbf{Vor- und Nachname:} \underline{\hspace{10cm}}\\%[0.5cm]
	
	%\vspace{1cm}

\textbf{Aufgabe 1}\hfill ( \punkteA{} Punkte)\

Berechne die folgenden Bruchterme. Achte auf die Klammern und kürze, wenn möglich.
Zeige alle Rechenschritte übersichtlich!

\begin{center}
	\renewcommand{\arraystretch}{2}
	\begin{tabular}{ll}
		a) $\left( \frac{3}{4} + \frac{1}{2} \right) \cdot \frac{2}{5} =$ & 
		b) $\frac{7}{8} : \left( \frac{3}{4} - \frac{1}{8} \right) =$ \\
		
		c) $\left( \frac{5}{6} - \frac{2}{3} \right) \cdot \left( \frac{3}{4} + \frac{1}{2} \right) =$ &
		d) $\frac{2}{3} + \left( \frac{4}{9} : \frac{2}{3} \right) =$ \\
	\end{tabular}
\end{center}

\textbf{Aufgabe 2} \hfill ( Punkte)\\
Eine Schule plant, drei Fünftel aller Klassenzimmer neu zu gestalten.
\(\frac{3}{10}\) dieser Räume sollen grün gestrichen werden, \(\frac{1}{4}\) in Gelbtönen.
Wie groß ist der Anteil der grün gestrichenen Räume an allen Räumen? Und wie groß ist der Anteil der gelben Räume an allen Räumen?
Gib beide Anteile vollständig gekürzt als Bruch an.\\

\textbf{Aufgabe 3} \hfill ( Punkte)\\
Ein Kuchenstück wiegt \( \frac{5}{6} \) kg.
Davon werden \( \frac{1}{4} \) kg für später eingefroren. Der Rest wird in 5 gleich große Stücke geteilt.
Wie viel wiegt ein dieser Stücke? Gib dein Ergebnis als Bruch in kg an.\\[1em]

\textbf{Aufgabe 4} \hfill ( Punkte)\\
Das Schwimmbad "Wiesenbad“ in Eschborn ist 25 m lang, 15 m breit und 2 m tief.\\[0.2cm]
	a)Berechne das Volumen des Beckens in Kubikmetern.\\[0.2cm]
	b) Wie viele Liter Wasser passen in das Schwimmbecken?\\



\textbf{Aufgabe 5} \hfill ( Punkte)\\
Gib jeweils die fehlenden Winkel an.

\begin{minipage}{0.48\textwidth}
	\textbf{a)}\\
	\includegraphics[width=0.75\textwidth]{winkel-1.png}\\
	Berechne die Winkel \(\beta\) und \(\gamma\). Begründe jeweils mit einer Winkelbeziehung.
\end{minipage}
\hfill
\begin{minipage}{0.48\textwidth}
	\textbf{b)}\\
	\includegraphics[width=0.75\textwidth]{winkel-2.png}\\
	Berechne den Winkel \(\alpha\). Begründe deine Rechnung.
\end{minipage}

\vspace{1.5cm}

\begin{minipage}{0.48\textwidth}
	\textbf{c)}\\
	\includegraphics[width=0.90\textwidth]{winkel-3.png}\\
	Berechne die Winkel \(\beta\) und \(c\). Begründe jeweils, welche Winkelbeziehungen du genutzt hast.
\end{minipage}
\hfill
\begin{minipage}{0.48\textwidth}
	\textbf{d)}\\
	\includegraphics[width=0.80\textwidth]{winkel-4.png}\\
	Berechne den Winkel \(\gamma\). Begründe deine Rechnung mithilfe der gegebenen Winkel.
\end{minipage}

\textbf{Aufgabe 6} \hfill ( Punkte)\\
Berechne die Flächeninhalte der folgenden Figuren. Zerlege die Flächen in bekannte geometrische Formen (z.\,B.\ Rechtecke, Dreiecke, Trapeze). Notiere alle Zwischenschritte und gib die Lösung jeweils in  an.\\

\begin{minipage}{0.48\textwidth}
	\textbf{a)}\\
	\includegraphics[width=0.9\textwidth]{flaeche-1.png}\\
\end{minipage}
\hfill
\begin{minipage}{0.48\textwidth}
	\textbf{b)}\\
	\includegraphics[width=0.9\textwidth]{flaeche-2.png}\\
\end{minipage}




	\vspace{0.5cm}
	\textbf{Auswertungstabelle:}
	\begin{center}
		\begin{tabular}{|c|c|c|c|c|c|c|c|}
			\hline
			Aufgabe & 1 & 2 & 3 & 4 & 5 & Summe & Note \\
			\hline
			Punkte  & \text{\quad / \punkteA } & \text{\quad / \punkteB } & \text{ \quad / \punkteC } & \text{\quad / \punkteD } & \text{\quad / \punkteE}  & \summe & \\
			\hline
		\end{tabular}
	\end{center}	
	%	\vspace{1cm}
	
	\textbf{Notenschlüssel:}
	
	\begin{center}
		\begin{tabular}{|c|c|c|c|c|c|c|}
			\hline
			Note & 1 & 2 & 3 & 4 & 5 & 6 \\
			\hline
			Prozent \% & 100--90 & 89--75 & 74--60 & 59--45 & 44--20 & 19--0 \\
			\hline
			Punkte & \maxSumme{}--\noteEinsMin{} & \fpeval{\noteEinsMin-1}--\noteZweiMin{} & \fpeval{\noteZweiMin-1}--\noteDreiMin{} & \fpeval{\noteDreiMin-1}--\noteVierMin{} & \fpeval{\noteVierMin-1}--\noteFunfMin{} & \fpeval{\noteFunfMin-1}--\noteSechsMin{} \\
			\hline
		\end{tabular}
	\end{center}
	
	\vspace{3cm}
	
	\textbf{Kenntnisnahme eines Elternteils:} \hrulefill \hfill \textbf{Note:} \hrulefill
	
\end{document}