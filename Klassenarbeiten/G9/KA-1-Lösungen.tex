\documentclass[a4paper,12pt]{article}
\usepackage{tabularx}
\usepackage{amsmath}
\usepackage[utf8]{inputenc}
\usepackage{multicol}
\usepackage{cancel}
\usepackage{amsmath, amssymb, amsthm}
\usepackage{graphicx}
\usepackage{enumitem}
\usepackage{array}
\usepackage[left=2cm, right=2cm, top=2cm, bottom=2cm]{geometry}
\usepackage{fancyhdr}
\usepackage{xfp}
\usepackage{pgf}

\usepackage{graphicx}
\usepackage{fancyhdr}

\setlength{\headheight}{28pt} % genug Platz für das Logo
\pagestyle{fancy}
\fancyhf{} % alles leeren
\fancyhead[L]{\includegraphics[height=1.2cm]{logo.png}}
\fancyhead[C]{\small Klassenarbeit – Rechnen mit Potenzen \ (Kl. G9A)}
\fancyhead[R]{\small Name:\ \rule{2.8cm}{0.4pt}}
\fancyfoot[C]{\thepage}

\fancyfoot[C]{Seite \thepage \enspace\textbullet\enspace J.\,Mycan \textcopyright~2025 *Klassenarbeit 45 min.*}

\renewcommand{\footrulewidth}{0.4pt}

%\pagestyle{fancy}
%\lhead{Klassenarbeit 45min.}
%\chead{Heinrich-von-Kleist-Schule}
%\rhead{Mathematik - G8A}
%\lfoot{}
%\cfoot{Seite \thepage}
%\rfoot{}

\newcommand{\punkteA}{6}
\newcommand{\punkteB}{6}
\newcommand{\punkteC}{6}
\newcommand{\punkteD}{18}
\newcommand{\punkteE}{6}
%\newcommand{\punkteF}{12}

\newcommand{\maxSumme}{42}
\newcommand{\noteEinsMin}{\fpeval{round(\maxSumme * 0.95,0)}}
\newcommand{\noteZweiMin}{\fpeval{round(\maxSumme * 0.80,0)}}
\newcommand{\noteDreiMin}{\fpeval{round(\maxSumme * 0.60,0)}}
\newcommand{\noteVierMin}{\fpeval{round(\maxSumme * 0.45,0)}}
\newcommand{\noteFunfMin}{\fpeval{round(\maxSumme * 0.20,0)}}
\newcommand{\noteSechsMin}{0}

\newcommand{\summe}{%
	\pgfmathparse{\punkteA + \punkteB + \punkteC + \punkteD + \punkteE}%
	\pgfmathprintnumber{\pgfmathresult}}

\begin{document}
	
%	\begin{center}
%		\textbf{Klassenarbeit - Lineare Funktionen und LGS}
%	\end{center}
	
%	\textbf{Vor- und Nachname:} \underline{\hspace{10cm}}\\[0.1cm]
%\vspace{2cm}
Lösungen! \\%[1cm]
	
%	\textbf{Vor- und Nachname:} \underline{\hspace{10cm}}\\[0.1cm]
%Die Lösungen sowie Lösungswege sollten klar strukturiert und gut nachvollziehbar sein. \\[0.1cm]

	\textbf{Aufgabe 1 (Punkte)}\\
Gegeben ist ein Koordinatensystem, in dem die Graphen der linearen Funktionen \(f\) und \(g\) eingezeichnet sind (siehe Abbildung~1).\\
Bestimme den Schnittpunkt der beiden Funktionen \emph{rechnerisch}.
Bestimme dazu zunächst aus der Abbildung die Funktionsgleichungen von \(f\) und \(g\).
Stelle anschließend das lineare Gleichungssystem
%	\[
%	\begin{cases}
	%		y = f(x)\\
	%		y = g(x)
	%	\end{cases}
%	\]
auf und löse es.
Gib das Ergebnis als Punkt \(S(x_S \mid y_S)\) an.\\[0.1pt]

%\vspace{0.5cm}
% Lösung zu Aufgabe 1
\textbf{Abgelesen:} \(f(x)=-x-1\), \(\;g(x)=-\tfrac{2}{3}x+2\).

\textbf{Lineares Gleichungssystem (Schnittpunkt):}
\[
\begin{cases}
	y=-x-1\\
	y=-\tfrac{2}{3}x+2
\end{cases}
\quad\Rightarrow\quad
-x-1=-\tfrac{2}{3}x+2
\]

\textbf{Gleichsetzungsverfahren:}
\[
\begin{aligned}
	-x+\tfrac{2}{3}x&=2+1\\
	-\tfrac{1}{3}x&=3\\
	x&=-9
\end{aligned}
\qquad
\Rightarrow\qquad
y=f(-9)= -(-9)-1=8
\]

\textbf{Probe:}\; \(g(-9)=-\tfrac{2}{3}\cdot(-9)+2=6+2=8\;\checkmark\)

\[
\boxed{S(-9,\,8)}
\]\\

	% -------------------------------------------------
	% Aufgabe 2
	% -------------------------------------------------
\textbf{Aufgabe 2 (Punkte)}\\
Im \emph{Schulkiosk} kosten \emph{Brötchen} und \emph{Saft} unterschiedlich viel.
Kundin A bezahlt für \emph{2 Brötchen und 3 Säfte} insgesamt \(13\,€\).
Kunde B bezahlt für \emph{3 Brötchen und 1 Saft} insgesamt \(9\,€\).

\begin{enumerate}[label=\alph*)]
	\item Lege geeignete Variablen fest und stelle aus den Angaben ein lineares Gleichungssystem auf.
	\item Löse das Gleichungssystem \emph{rechnerisch} und gib die beiden Einzelpreise an.
\end{enumerate}
\vspace{0.5cm}
% Lösung zu Aufgabe 2
\begin{enumerate}[label=\alph*)]
	\item \textbf{Variablen:} Sei \(b\) der Preis eines Brötchens (in €), \(s\) der Preis eines Safts (in €).\\
	Aus den Angaben folgt das LGS:
	\[
	\begin{cases}
		2b+3s=13\\
		3b+s=9
	\end{cases}
	\]
	
	\item \textbf{Rechnung (z.\,B. Einsetzungsverfahren):}\\
	Aus \(3b+s=9\) folgt \(s=9-3b\). Einsetzen in die erste Gleichung:
	\[
	2b+3(9-3b)=13 \;\Rightarrow\; 2b+27-9b=13 \;\Rightarrow\; -7b=-14 \;\Rightarrow\; b=2.
	\]
	Dann \(s=9-3\cdot2=3\).\\[2pt]
	\textbf{Antwort:} Ein Brötchen kostet \(\boxed{2\,€}\), ein Saft kostet \(\boxed{3\,€}\).
\end{enumerate}


	
	
	%\vspace{2cm}

	\newpage
	% -------------------------------------------------
	% Aufgabe 3
	% -------------------------------------------------
\textbf{Aufgabe 3 (18 Punkte)}\\
Vereinfache vollständig; schreibe ohne negative Exponenten und rationalisiere ggf. den Nenner. Es gelte \(a,b,x,y>0\).

% Lösung zu Aufgabe 4
\begin{multicols}{2}
	\begin{enumerate}[label=\alph*)]
		\item \(\sqrt{180}=\sqrt{36\cdot 5}=6\sqrt{5}\)
		\item \(\sqrt{288}=\sqrt{144\cdot 2}=12\sqrt{2}\)
		\item \(\sqrt{392}=\sqrt{196\cdot 2}=14\sqrt{2}\)
		\item \(\sqrt{450}=\sqrt{225\cdot 2}=15\sqrt{2}\)
		\item \(\sqrt{500}=\sqrt{100\cdot 5}=10\sqrt{5}\)
		\item \(\sqrt{338}=\sqrt{169\cdot 2}=13\sqrt{2}\)
	\end{enumerate}
\end{multicols}


	
	% -------------------------------------------------
	% Aufgabe 4
	% -------------------------------------------------
%	\textbf{Aufgabe 5 (15 Punkte)}\\
%	Vereinfache vollständig; schreibe ohne negative Exponenten und rationalisiere ggf. den Nenner. Es gelte \(a,b,x,y>0\).
%	\begin{multicols}{2}
%		\begin{enumerate}[label=\alph*)]
%			\item \( \dfrac{a^{2n+1}}{b^{-3n}}\cdot \dfrac{b^{-3n-5}}{a^{2n+6}} \)
%			\item \( \dfrac{y^{3n-5}}{x^{n+3}}\cdot \dfrac{x^{2n+5}}{y^{2n-4}} \)
%		\end{enumerate}
%	\end{multicols}
	
	
	%\vspace{1.5cm}
	% -------------------------------------------------
	% Aufgabe 5
	% -------------------------------------------------
	\textbf{Aufgabe 4 (Punkte)}\\
	Vereinfache vollständig \emph{(ggf. Nenner rationalisieren)}. Es gelte \(a,b,x,y>0\).
	
% Lösungen zu Aufgabe 5
\begin{multicols}{2}
	\begin{enumerate}[label=\alph*)]
		
		\item \(\sqrt{18\,x^{2}}
		=\sqrt{9\cdot2}\,\sqrt{x^{2}}
		=3x\sqrt{2}\) \quad (da \(x>0\))
		
		\item \(\sqrt{\dfrac{72\,a^{2}}{2}}
		=\sqrt{36\,a^{2}}
		=6a\)
		
		\item \(\dfrac{\sqrt{27\,a^{5}}}{\sqrt{3a}}
		=\sqrt{\dfrac{27a^{5}}{3a}}
		=\sqrt{9a^{4}}
		=3a^{2}\)
		
		\item \(\dfrac{\sqrt{75\,y^{3}}}{5\sqrt{y}}
		=\dfrac{1}{5}\sqrt{\dfrac{75y^{3}}{y}}
		=\dfrac{1}{5}\sqrt{75y^{2}}
		=\dfrac{1}{5}\cdot5\sqrt{3}\cdot y
		=y\sqrt{3}\)
		
		\item \(3\sqrt{8x}-\sqrt{2x}+2\sqrt{18x}
		=3\cdot2\sqrt{2x}-\sqrt{2x}+2\cdot3\sqrt{2x}
		=(6-1+6)\sqrt{2x}
		=11\sqrt{2x}\)
		
		\item \(\dfrac{1}{\sqrt{a}+\sqrt{b}}-\dfrac{1}{\sqrt{a}-\sqrt{b}}
		=\dfrac{(\sqrt{a}-\sqrt{b})-(\sqrt{a}+\sqrt{b})}{(\sqrt{a}+\sqrt{b})(\sqrt{a}-\sqrt{b})}
		=\dfrac{-2\sqrt{b}}{a-b}
		=\dfrac{2\sqrt{b}}{\,b-a\,}\)
	\end{enumerate}
\end{multicols}

	% -------------------------------------------------
% Aufgabe 3
% -------------------------------------------------
\textbf{Aufgabe 5 (Punkte)}\\
Gegeben ist das lineare Gleichungssystem in Matrixschreibweise
\[
A\vec{x}=\vec{b},\quad
A=\begin{pmatrix}
	1 & 0 & 1\\
	2 & 3 & 0\\
	3 & 0 & 2
\end{pmatrix},\quad
\vec{b}=\begin{pmatrix}
	4\\
	8\\
	9
\end{pmatrix}.
\]\\[0.5pt]
% Lösung zu Aufgabe 3
Sei \(\vec{x}=(x,y,z)^{\top}\). Das LGS lautet
\[
\begin{cases}
	x+z=4\\
	2x+3y=8\\
	3x+2z=9
\end{cases}
\]

\textbf{Gauß-Verfahren (Zeilenumformungen am erweiterten Schema):}
\[
\left[
\begin{array}{ccc|c}
	1&0&1&4\\
	2&3&0&8\\
	3&0&2&9
\end{array}
\right]
\ \xrightarrow{\,R_2\leftarrow R_2-2R_1\,}\ 
\left[
\begin{array}{ccc|c}
	1&0&1&4\\
	0&3&-2&0\\
	3&0&2&9
\end{array}
\right]
\ \xrightarrow{\,R_3\leftarrow R_3-3R_1\,}\ 
\left[
\begin{array}{ccc|c}
	1&0&1&4\\
	0&3&-2&0\\
	0&0&-1&-3
\end{array}
\right]
\]

\textbf{Rückwärtseinsetzen:}
\[
-\,z=-3\ \Rightarrow\ z=3,\qquad
3y-2z=0\ \Rightarrow\ 3y-6=0\ \Rightarrow\ y=2,
\]
\[
x+z=4\ \Rightarrow\ x=4-3=1.
\]

\textbf{Lösung:}\quad \(\boxed{(x,y,z)=(1,2,3)}\).

\textbf{Probe:}\;
\(2\cdot1+3\cdot2=8\),\ \ \(3\cdot1+2\cdot3=9\) \(\checkmark\).



	
	\vspace{6cm}
	\textbf{Auswertungstabelle:}
	\begin{center}
		\begin{tabular}{|c|c|c|c|c|c|c|c|}
			\hline
			Aufgabe & 1 & 2 & 3 & 4 & 5 &  Summe\\
			\hline
			Punkte & \text{\ / \punkteA} & \text{\ / \punkteB} & \text{\ / \punkteC} & \text{\ / \punkteD} & \text{\ / \punkteE} & \text{\ / \summe}\\
			\hline
		\end{tabular}
	\end{center}
	
	\textbf{Notenschlüssel:}
	\begin{center}
		\begin{tabular}{|c|c|c|c|c|c|c|}
			\hline
			Note & 1 & 2 & 3 & 4 & 5 & 6 \\
			\hline
			Prozent \% & 100--95 & 94--80 & 79--60 & 59--45 & 44--16 & 15--0 \\
			\hline
			Punkte & \maxSumme{}--\noteEinsMin{} & \fpeval{\noteEinsMin-1}--\noteZweiMin{} & \fpeval{\noteZweiMin-1}--\noteDreiMin{} & \fpeval{\noteDreiMin-1}--\noteVierMin{} & \fpeval{\noteVierMin-1}--\noteFunfMin{} & \fpeval{\noteFunfMin-1}--\noteSechsMin{} \\
			\hline
		\end{tabular}
	\end{center}
	
	\vspace{2cm}
	\textbf{Kenntnisnahme eines Elternteils:} \hrulefill \hfill \textbf{Note:} \hrulefill
	
\end{document}
