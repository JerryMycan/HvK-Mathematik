\documentclass[a4paper,12pt]{article}
\usepackage{tabularx}
\usepackage{amsmath}
\usepackage[utf8]{inputenc}
\usepackage{amsmath, amssymb, amsthm}
\usepackage{graphicx}
\usepackage{array}
\usepackage[left=2cm, right=2cm, top=2cm, bottom=2cm]{geometry}
\usepackage{fancyhdr}
\usepackage{xfp} % Für mathematische Berechnungen in LaTeX
\usepackage{pgf} % Erlaubt Berechnungen mit \pgfmathparse

% Kopf- und Fußzeile
\pagestyle{fancy}
\lhead{Klassenarbeit 45min.}
\chead{Heinrich-von-Kleist-Schule}
\rhead{Mathematik - G6E}
\lfoot{}
\rfoot{}

% Makro für Punkte
\newcommand{\punkteA}{20}
\newcommand{\punkteB}{15}
\newcommand{\punkteC}{12}
\newcommand{\punkteD}{6}
\newcommand{\punkteE}{8}
\newcommand{\maxSumme}{61}

\begin{document}
	
	\begin{center}
		\textbf{Bruchrechnung und Prozentrechnung}\\[0.2cm]
	\end{center}
	
	\textbf{Vor- und Nachname:} \underline{\hspace{10cm}}\\
	
	\textbf{Aufgabe 1} \hfill (8+12 = \punkteA{} Punkte)\\
	Berechne die folgenden Brüche.
	\begin{center}
		\renewcommand{\arraystretch}{1.5}
		\begin{tabular}{llll}
			a) $\frac{4}{7} + \frac{3}{5} =$ & b) $\frac{11}{12} - \frac{5}{8} =$ & c) $\frac{3}{4} \times \frac{7}{9} =$ & d) $\frac{6}{11} \div \frac{2}{5} =$ \\
			e) $\frac{7}{9} + \frac{1}{3} - \frac{2}{5} =$ & f) $\frac{8}{11} \times \frac{4}{7} \div \frac{5}{9} =$ & g) $\left( \frac{2}{3} + \frac{4}{9} \right) \times \frac{5}{8} =$ & h) $\frac{10}{13} \div \left( \frac{7}{9} - \frac{1}{6} \right) =$ \\
		\end{tabular}
	\end{center}
	
	\textbf{Aufgabe 2} \hfill (6+9 = \punkteB{} Punkte)\\
	Wandle die folgenden Prozentangaben in Brüche und Dezimalzahlen um.
	\begin{center}
		\renewcommand{\arraystretch}{1.5}
		\begin{tabular}{lll}
			a) $35\% =$ & b) $60\% =$ & c) $15.5\% =$ \\
			d) $82.25\% =$ & e) $9\% =$ & f) $95.4\% =$ \\
		\end{tabular}
	\end{center}
	
	\textbf{Aufgabe 3} \hfill (\punkteC{} Punkte)\\
	Berechne die entsprechenden Anteile.
	\begin{center}
		\renewcommand{\arraystretch}{1.5}
		\begin{tabular}{lll}
			a) $\frac{3}{8}$ von 240m = & b) $\frac{4}{7}$ von 6,3dm = & c) $9\%$ von 2,8m = \\
			d) $\frac{1}{6}$ von \dots min sind 45min & e) $\frac{5}{9}$ von \dots sind 81h = & f) $\frac{7}{15}$ von \dots sind 140€ \\
		\end{tabular}
	\end{center}
	
	\textbf{Aufgabe 4} \hfill (\punkteD{} Punkte)\\
	In einer Klasse mit 32 Schülern haben ein Viertel Mathematik als Lieblingsfach, ein Drittel bevorzugt Sport, und 8 Schüler mögen Kunst am liebsten. Wie viele Schüler haben mindestens eine dieser Vorlieben? Begründe deine Antwort.
	
	\textbf{Aufgabe 5} \hfill (2+6=\punkteE{} Punkte)\\
	Ein Zehntel der Menschen in einer Stadt nutzt das Fahrrad, während 40\% mit öffentlichen Verkehrsmitteln fahren. Der Rest fährt mit dem Auto oder geht zu Fuß.
	\begin{enumerate}
		\item[a)] Welcher Anteil (Bruch) der Bewohner nutzt nicht das Fahrrad oder öffentliche Verkehrsmittel?
		\item[b)] Wenn 3200 Bewohner öffentliche Verkehrsmittel nutzen, wie viele Bewohner hat die Stadt insgesamt?
	\end{enumerate}
	
\end{document}
