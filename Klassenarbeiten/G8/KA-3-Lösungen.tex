\documentclass[a4paper,12pt]{article}
\usepackage{tabularx}
\usepackage{amsmath, amssymb, amsthm}
\usepackage[utf8]{inputenc}
\usepackage{geometry}
\usepackage{fancyhdr}
\usepackage{graphicx}

\geometry{left=2cm, right=2cm, top=2cm, bottom=2cm}
\pagestyle{fancy}
\lhead{L\"osungen - Klassenarbeit}
\chead{Heinrich-von-Kleist-Schule}
\rhead{Mathematik - G8A}
\lfoot{}
\cfoot{Seite \thepage}
\rfoot{}

\begin{document}
	
	\begin{center}
		\textbf{Lösungen zur Klassenarbeit: Un- und Gleichungen, Bruchgleichungen, quadratische Ergänzung, Wahrscheinlichkeitsrechnung}\
	\end{center}
	
	\section*{Aufgabe 1: Gleichungen und Ungleichungen lösen.}
	
	\textbf{a)} \quad $(x-3)(x+3) = (x - 5)^2$
	\begin{align*}
		(x-3)(x+3) &= (x - 5)^2 \\
		x^2 - 9 &= x^2 - 10x + 25 \\
		x^2 - 9 - x^2 + 10x - 25 &= 0 \\
		10x - 34 &= 0 \\
		10x &= 34 \\
		x &= \frac{34}{10} \\
		x &= \frac{17}{5}
	\end{align*}

	
	\textbf{b)} \quad $x(2x + 14) = (x + 7)^2 - 7$
\begin{align*}
	x(2x + 14) &= (x + 7)^2 - 7 \\
	2x^2 + 14x &= x^2 + 14x + 49 - 7 \\
	2x^2 + 14x - x^2 - 14x - 42 &= 0 \\
	x^2 - 42 &= 0 \\
	x^2 = 49 \\
	x = \pm 7
\end{align*}
	
	\textbf{c)} \quad $\frac{2x + 1}{2} \geq \frac{3 - 2x}{3}$
	\begin{align*}
		\frac{2x + 1}{2} &\geq \frac{3 - 2x}{3} \\
		\text{Hauptnenner: } 6 \\
		3(2x + 1) &\geq 2(3 - 2x) \\
		6x + 3 &\geq 6 - 4x \\
		6x + 4x &\geq 6 - 3 \\
		10x &\geq 3 \\
		x &\geq \frac{3}{10}
	\end{align*}

	
	\section*{Aufgabe 2: Bruchgleichungen lösen.}
	
	\textbf{a)} \quad $\frac{3}{x} = - \frac{1}{2 -x}$
	\begin{align*}
		\frac{3}{x} &= - \frac{1}{2 - x} \\
		\text{Definitionsmenge: } & x \neq 0, \quad x \neq 2 \\
		\text{Kreuzmultiplikation: } & 3(2 - x) = - x \cdot 1 \\
		6 - 3x &= - x \quad | + 3x\\
		6 &= 2x \quad |: 2\\
		x &= 3
	\end{align*}
	
	\textbf{b)} \quad $\frac{2}{x} + 4 = \frac{3x +2}{x}$
	\begin{align*}
		\frac{2}{x} + 4 &= \frac{3x + 2}{x} \\
		\text{Definitionsmenge: } & x \neq 0 \\
		\text{Brüche auf gemeinsamen Nenner bringen:} & \frac{2 + 4x}{x} = \frac{3x + 2}{x} \\
		2 + 4x &= 3x + 2 \\
		4x - 3x &= 2 - 2 \\
		x &= 0\\
		x &\notin D
	\end{align*}

	
	\textbf{c)} \quad $\frac{3}{x + 1} = \frac{5}{x + 2}$
	\begin{align*}
		\frac{3}{x + 1} &= \frac{5}{x + 2} \\
		\text{Definitionsmenge: } & x \neq -1, \quad x \neq -2 \\
		\text{Kreuzmultiplikation: } & 3(x + 2) = 5(x + 1) \\
		3x + 6 &= 5x + 5 \\
		3x - 5x &= 5 - 6 \\
		-2x &= -1 \\
		x &= \frac{1}{2}
	\end{align*}

	
	
	\section*{Aufgabe 3: Quadratische Erg\"anzung.}
	
	\textbf{a)} \quad $x^2 + 8x$
	\begin{align*}
		x^2 + 8x &= \left(x^2 + 8x + \left(\frac{8}{2}\right)^2 \right) - \left(\frac{8}{2}\right)^2 \\
		&= \left(x + 4\right)^2 - 16
	\end{align*}
	
	\textbf{b)} \quad $x^2 - 12x + 9$
	\begin{align*}
		x^2 - 12x + 9 &= \left(x^2 - 12x + \left(\frac{12}{2}\right)^2 \right) - \left(\frac{12}{2}\right)^2 + 9 \\
		&= \left(x - 6\right)^2 - 36 + 9 \\
		&= (x - 6)^2 - 27
	\end{align*}

	
	\textbf{b)} \quad $3x^2 - 15x + 14$
	\begin{align*}
		3x^2 - 15x + 14 &= 3\left(x^2 - 5x\right) + 14 \\
		&= 3\left(x^2 - 5x + \left(\frac{5}{2}\right)^2 - \left(\frac{5}{2}\right)^2 \right) + 14 \\
		&= 3\left(\left(x - \frac{5}{2}\right)^2 - \frac{25}{4} \right) + 14 \\
		&= 3(x - \frac{5}{2})^2 - \frac{75}{4} + 14 \\
		&= 3(x - \frac{5}{2})^2 - \frac{19}{4}
	\end{align*}

	
	\textbf{Aufgabe 4} \hfill (6 Punkte)\\
Verkürzt man die Seite eines Quadrates um 3 cm und verlängert die andere um 4 cm , so entsteht ein Rechteck, das den gleichen Flächeninhalt hat wie das Quadrat. Wie lang ist die Quadratseite?

	\begin{align*}
		\text{Sei } x \text{ die Seitenlänge des Quadrats.} \\
		\text{Fläche des Quadrats:} & \quad x^2 \\
		\text{Fläche des neuen Rechtecks:} & \quad (x - 3)(x + 4) \\
		\text{Da beide Flächen gleich sind:} & \quad x^2 = (x - 3)(x + 4) \\
		\text{Ausmultiplizieren:} & \quad x^2 = x^2 + 4x - 3x - 12 \\
		& \quad x^2 = x^2 + x - 12 \\
		\text{Beide Seiten um } x^2 \text{ reduzieren:} & \quad 0 = x - 12 \\
		& \quad x = 12
	\end{align*}

	
	\textbf{Aufgabe 5} \hfill (15 Punkte)\\
	In einer Schachtel befinden sich 20 Kugeln: 8 rote, 6 blaue und 6 grüne. Es werden mehrere Kugeln nacheinander ohne Zurücklegen gezogen.
	\[
	\begin{array}{ll}
		\textbf{a)} & \text{Wie groß ist die Wahrscheinlichkeit, dass die erste gezogene Kugel rot ist?} \\
		\textbf{b)} & \text{Wie groß ist die Wahrscheinlichkeit, dass die erste Kugel rot und die zweite Kugel blau ist?} \\
		\textbf{c)} & \text{Es wird drei Mal gezogen. Wie groß ist die Wahrscheinlichkeit, dass genau zwei Kugeln dieselbe Farbe haben} \\ 
		& \text{und die dritte Kugel eine andere Farbe?} \\
		\textbf{d)} & \text{Wie groß ist die Wahrscheinlichkeit, dass alle drei Kugeln unterschiedliche Farben haben?} 
	\end{array}
	\]
	\textbf{a)} Wahrscheinlichkeit, dass die erste Kugel rot ist:
	\begin{align*}
		P(\text{erste Kugel rot}) &= \frac{8}{20} = \frac{2}{5} = 0.4
	\end{align*}
	
	\textbf{b)} Wahrscheinlichkeit, dass die erste Kugel rot und die zweite Kugel blau ist:
	\begin{align*}
		P(RB) &= P(R) \cdot P(B) \\
		&= \frac{8}{20} \times \frac{6}{19} = \frac{48}{380} = \frac{24}{190} \approx 0.2526
	\end{align*}
	
	\textbf{c)} Wahrscheinlichkeit, dass genau zwei Kugeln dieselbe Farbe haben und die dritte eine andere ist:
	\begin{align*}
		P(\text{genau zwei gleiche, eine andere}) &= P(RRB) + P(RRG) + P(BBR) + P(BBG) + P(GGR) + P(GGB) \\
		P(RRB) &= \frac{8}{20} \times \frac{7}{19} \times \frac{6}{18} = \frac{336}{6840} = \frac{28}{570} \\
		P(RRG) &= \frac{8}{20} \times \frac{7}{19} \times \frac{6}{18} = \frac{336}{6840} = \frac{28}{570} \\
		P(BBR) &= \frac{6}{20} \times \frac{5}{19} \times \frac{8}{18} = \frac{240}{6840} = \frac{20}{570} \\
		P(BBG) &= \frac{6}{20} \times \frac{5}{19} \times \frac{6}{18} = \frac{180}{6840} = \frac{15}{570} \\
		P(GGR) &= \frac{6}{20} \times \frac{5}{19} \times \frac{8}{18} = \frac{240}{6840} = \frac{20}{570} \\
		P(GGB) &= \frac{6}{20} \times \frac{5}{19} \times \frac{6}{18} = \frac{180}{6840} = \frac{15}{570} \\
		P(\text{genau zwei gleiche, eine andere}) &= \frac{28}{570} + \frac{28}{570} + \frac{20}{570} + \frac{15}{570} + \frac{20}{570} + \frac{15}{570} = \frac{126}{570} \approx 0.2211
	\end{align*}
	
	\textbf{d)} Wahrscheinlichkeit, dass alle drei Kugeln unterschiedliche Farben haben:
	\begin{align*}
		P(\text{alle unterschiedlich}) &= P(RBG) + P(RGB) + P(BRG) + P(BGR) + P(GRB) + P(GBR) \\
		P(RBG) &= \frac{8}{20} \times \frac{6}{19} \times \frac{6}{18} = \frac{288}{6840} = \frac{24}{570} \\
		P(RGB) &= \frac{8}{20} \times \frac{6}{19} \times \frac{6}{18} = \frac{288}{6840} = \frac{24}{570} \\
		P(BRG) &= \frac{6}{20} \times \frac{8}{19} \times \frac{6}{18} = \frac{288}{6840} = \frac{24}{570} \\
		P(BGR) &= \frac{6}{20} \times \frac{6}{19} \times \frac{8}{18} = \frac{288}{6840} = \frac{24}{570} \\
		P(GRB) &= \frac{6}{20} \times \frac{8}{19} \times \frac{6}{18} = \frac{288}{6840} = \frac{24}{570} \\
		P(GBR) &= \frac{6}{20} \times \frac{6}{19} \times \frac{8}{18} = \frac{288}{6840} = \frac{24}{570} \\
		P(\text{alle unterschiedlich}) &= \frac{24}{570} + \frac{24}{570} + \frac{24}{570} + \frac{24}{570} + \frac{24}{570} + \frac{24}{570} = \frac{144}{570} \approx 0.2526
	\end{align*}

	
\end{document}
