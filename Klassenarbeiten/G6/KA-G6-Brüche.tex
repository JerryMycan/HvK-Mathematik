\documentclass[a4paper,12pt]{article}
\usepackage{tabularx}
\usepackage{amsmath}
\usepackage[utf8]{inputenc}
\usepackage{amsmath, amssymb, amsthm}
\usepackage{graphicx}
\usepackage{array}
\usepackage[left=2cm, right=2cm, top=2cm, bottom=2cm]{geometry}
\usepackage{fancyhdr}
\usepackage{xfp} % Für mathematische Berechnungen in LaTeX
\usepackage{pgf} % Erlaubt Berechnungen mit \pgfmathparse

% Kopf- und Fußzeile
\pagestyle{fancy}
\lhead{6. März 2025\\Klassenarbeit 45min.}
\chead{Heinrich-von-Kleist-Schule}
\rhead{Mathematik - G6E}
\lfoot{}
%\cfoot{Seite \thepage}
\rfoot{}

% Makro für einzelne Punkteingaben (hier einfach Platzhalter)
\newcommand{\punkteA}{20}  % Eingetragene Punkte für Aufgabe 1
\newcommand{\punkteB}{15}  % Eingetragene Punkte für Aufgabe 2
\newcommand{\punkteC}{12}   % Eingetragene Punkte für Aufgabe 3
\newcommand{\punkteD}{6}   % Eingetragene Punkte für Aufgabe 4
\newcommand{\punkteE}{0}  % Eingetragene Punkte für Aufgabe 5

% Maximale Punktzahl insgesamt
\newcommand{\maxSumme}{53}  % Hier die maximale Punktzahl anpassen

% Automatische Berechnung der Notengrenzen
\newcommand{\noteEinsMin}{\fpeval{round(\maxSumme * 0.90,0)}}
\newcommand{\noteZweiMin}{\fpeval{round(\maxSumme * 0.75,0)}}
\newcommand{\noteDreiMin}{\fpeval{round(\maxSumme * 0.60,0)}}
\newcommand{\noteVierMin}{\fpeval{round(\maxSumme * 0.45,0)}}
\newcommand{\noteFunfMin}{\fpeval{round(\maxSumme * 0.20,0)}}
\newcommand{\noteSechsMin}{0}

% Automatische Berechnung der Summe
\newcommand{\summe}{%
	\pgfmathparse{\punkteA + \punkteB + \punkteC + \punkteD + \punkteE}%
	\pgfmathprintnumber{\pgfmathresult}
}


\begin{document}
	
	\begin{center}
		%\textbf{Klassenarbeit N°1 G8}\\[0.3cm]
		\textbf{Rechnen mit Brüchen}\\[0.2cm]
		%\textbf{© J. Mycan}\\[0.5cm]
	\end{center}
	
	\textbf{Vor- und Nachname:} \underline{\hspace{10cm}}\\%[0.5cm]
	
	%\vspace{1cm}

\textbf{Aufgabe 1}\hfill (8+12 = \punkteA{} Punkte)\
\\Löse folgende Aufgaben.
\begin{center}
	\renewcommand{\arraystretch}{1.5}
	\begin{tabular}{llll}
		a) $\frac{3}{4} + \frac{5}{6} =$ & \quad b) $\frac{7}{8} - \frac{2}{5} =$ & \quad  c) $\frac{2}{3} \cdot \frac{4}{9} =$ & \quad  d) $\frac{5}{7} \div \frac{3}{8} =$ \\
		e) $\frac{1}{2} + \frac{1}{3} - \frac{1}{4} =$ & \quad f) $\frac{5}{6} \cdot \frac{3}{4} \div \frac{2}{5} =$ & \quad g) $\left( \frac{3}{5} + \frac{2}{7} \right) \cdot \frac{4}{9} =$ & \quad h) $\frac{8}{11} \div \left( \frac{5}{6} - \frac{1}{4} \right) =$ \\
	\end{tabular}
\end{center}

	\textbf{Aufgabe 2} \hfill (6+9 = \punkteB{} Punkte)\\
	Wandle die folgenden Prozentangaben in Brüche und Dezimalzahlen um. Beispiel: \[50\% = \frac{1}{2} = 0,5\]
\begin{center}
	\renewcommand{\arraystretch}{1.5}
	\begin{tabular}{lll}
		a) $25\% =$ & \qquad b) $40\% =$ & \qquad c) $12.5\% =$ \\
		d) $66.6\% =$ & \qquad e) $5\% =$ & \qquad f) $83.33\% =$ \\
	\end{tabular}
\end{center}
	
	\textbf{Aufgabe 3} \hfill (\punkteC{} Punkte)\\
	Berechne die entsprechende  Anteile von Größen.
\begin{center}
	\renewcommand{\arraystretch}{1.5}
	\begin{tabular}{lll}
		a) $\frac{1}{12}$ von 200m = & b) $\frac{5}{6}$ von 4,2dm = & c) $11\%$ von 2m = \\
		d) $\frac{1}{5}$ von \dots min sind 40min &  e) $\frac{9}{10}$ von \dots sind 54h & f) $\frac{6}{13}$ von \dots sind 84€\\
	\end{tabular}
\end{center}
	
	
	\textbf{Aufgabe 4} \hfill (\punkteD{} Punkte)\\
	Kurt Klugschwätzer behauptet: "In der Klasse 6b spielen ein Drittel aller Schüler genau ein Instrument. Die Hälfte davon spielt Geige, zwei Schüler flöten und drei Schüler spielen Klavier."
	Wie viele Schüler hat die Klasse 6b? Begründe deine Antwort.\\

	\textbf{Aufgabe 5} \hfill (2+6=\punkteE{} Punkte)\\
	Ein Zwölftel aller Schüler einer Realschule fahren mit dem Fahrrad und mit dem Moped zur Schule. Ein Achtel der Schüler wird mit dem Schulbus gebracht. Die übrigen kommen zu Fuß.
\begin{enumerate}
	\item[a)] Welcher Anteil (Bruch) der Schüler kommt nicht zu Fuß?
	\item[b)] Wenn 27 Schüler ein Fahrrad oder Moped benutzen, wie viele Schüler hat dann diese Schule?
\end{enumerate}

	\vspace{0.5cm}
	\textbf{Auswertungstabelle:}
	\begin{center}
		\begin{tabular}{|c|c|c|c|c|c|c|c|}
			\hline
			Aufgabe & 1 & 2 & 3 & 4 & 5 & Summe & Note \\
			\hline
			Punkte  & \text{\quad / \punkteA } & \text{\quad / \punkteB } & \text{ \quad / \punkteC } & \text{\quad / \punkteD } & \text{\quad / \punkteE}  & \summe & \\
			\hline
		\end{tabular}
	\end{center}	
	%	\vspace{1cm}
	
	\textbf{Notenschlüssel:}
	
	\begin{center}
		\begin{tabular}{|c|c|c|c|c|c|c|}
			\hline
			Note & 1 & 2 & 3 & 4 & 5 & 6 \\
			\hline
			Prozent \% & 100--90 & 89--75 & 74--60 & 59--45 & 44--20 & 19--0 \\
			\hline
			Punkte & \maxSumme{}--\noteEinsMin{} & \fpeval{\noteEinsMin-1}--\noteZweiMin{} & \fpeval{\noteZweiMin-1}--\noteDreiMin{} & \fpeval{\noteDreiMin-1}--\noteVierMin{} & \fpeval{\noteVierMin-1}--\noteFunfMin{} & \fpeval{\noteFunfMin-1}--\noteSechsMin{} \\
			\hline
		\end{tabular}
	\end{center}
	
	\vspace{3cm}
	
	\textbf{Kenntnisnahme eines Elternteils:} \hrulefill \hfill \textbf{Note:} \hrulefill
	
\end{document}