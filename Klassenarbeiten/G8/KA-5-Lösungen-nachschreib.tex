\documentclass[a4paper,12pt]{article}
\usepackage{amsmath, amssymb}
\usepackage{graphicx}
\usepackage[utf8]{inputenc}
\usepackage[ngerman]{babel}
\usepackage{geometry}
\geometry{left=2cm,right=2cm,top=2cm,bottom=2cm}

\title{\textbf{L\"osungen zur Klassenarbeit -- Lineare Funktionen und LGS}}
\date{}
\begin{document}
	\maketitle
	
	\section*{Aufgabe 1 (10 Punkte)}
	\textbf{Gegeben:} Zwei lineare Funktionen \( f(x) \) und \( g(x) \) mit Graphen.
	
	\textbf{Ziel:} Schnittpunkt rechnerisch bestimmen.
	
	\textbf{Beispielhafte Funktionen:}
	\[
	f(x) = -\frac{3}{4}x + \frac{1}{2}, \quad g(x) = -\frac{5}{4}x - \frac{5}{2}
	\]
	
	\textbf{1. Gleichsetzen:}
	\[
	-\frac{3}{4}x + \frac{1}{2} = -\frac{5}{4}x - \frac{5}{2}
	\]
	
	\textbf{2. Zusammenfassen:}
	\[
	\frac{1}{2} + \frac{5}{2} = -\frac{5}{4}x + \frac{3}{4}x \\
	3 = -\frac{2}{4}x = -\frac{1}{2}x
	\]
	
	\textbf{3. L\"osen:} 
	\[
	x = -6
	\]
	
	\textbf{4. In f einsetzen:} 
	\[
	y = -\frac{3}{4} \cdot (-6) + \frac{1}{2} = \frac{18}{4} + \frac{1}{2} = 4.5 + 0.5 = 5
	\]
	
	\textbf{L\"osung:} Der Schnittpunkt ist \( S(-6|5) \).
	
	\section*{Aufgabe 2 (8 Punkte)}
	\textbf{Gegeben:} 22 Fahrr\"ader, insgesamt 64 R\"ader
	
	\textbf{Annahme:} \( x \): normale Fahrr\"ader (2 R\"ader), \( y \): Kinderr\"ader mit St\"utzr\"adern (4 R\"ader)
	
\begin{align*}
	x + y &= 22 \\
	2x + 4y &= 64
\end{align*}

	
	\textbf{1. Umformen:} Erste Gleichung umstellen: \( x = 22 - y \)
	
	\textbf{2. Einsetzen:}
	\[
	2(22 - y) + 4y = 64 \\
	44 - 2y + 4y = 64 \\
	2y = 20 \Rightarrow y = 10
	\]
	
	\textbf{3. Einsetzen:} \( x = 22 - 10 = 12 \)
	
	\textbf{L\"osung:} 12 normale Fahrr\"ader, 10 Kinderr\"ader mit St\"utzr\"adern
	
	\section*{Aufgabe 3 (7 Punkte)}
	\textbf{Gegeben:} Vater ist dreimal so alt wie Sohn. In 12 Jahren doppelt so alt.
	
	\textbf{Annahme:} \( x \): Sohn, \( y = 3x \): Vater
	
	\textbf{In 12 Jahren:} \( x + 12 \), \( y + 12 \)
	
	\[
	y + 12 = 2(x + 12) \Rightarrow 3x + 12 = 2x + 24 \Rightarrow x = 12 \Rightarrow y = 36
	\]
	
	\textbf{L\"osung:} Sohn ist 12 Jahre alt, Vater ist 36 Jahre alt.
	
	\section*{Aufgabe 4 (9 Punkte)}
	\textbf{Gegeben:} Start 1500 Liter. Jede Woche 50 Liter entnommen, 20 Liter aufgef\"ullt.
	
	\textbf{Nettomenge:} -30 Liter/Woche
	
	\textbf{Ziel:} Wann sind 1080 Liter im Tank?
	\[
	1500 - 30x = 1080 \Rightarrow 30x = 420 \Rightarrow x = 14
	\]
	
	\textbf{L\"osung:} Nach 14 Wochen sind 1080 Liter im Tank.
	
	\section*{Aufgabe 5 (11 Punkte)}
	\textbf{Gegeben:}
	\begin{itemize}
		\item Ausflugsschiff: Start 9:00 Uhr, Tempo: 20 km/h
		\item Schnellboot: Start 10:00 Uhr, Tempo: 30 km/h
	\end{itemize}
	
	\textbf{Zur\"uckgelegte Strecke des Schiffes nach \( t \) h:} \( s_1 = 20t \)
	
	\textbf{Zur\"uckgelegte Strecke des Schnellboots:} \( s_2 = 30(t - 1) \) 
	
	\textbf{Begegnungspunkt:} \( s_1 = s_2 \)
	\[
	20t = 30(t - 1) \\
	20t = 30t - 30 \Rightarrow 10t = 30 \Rightarrow t = 3
	\]
	
	\textbf{Ort:} \( s = 20 \cdot 3 = 60 \text{ km} \)
	
	\textbf{Zeitpunkt:} 9:00 Uhr + 3h = \textbf{12:00 Uhr}
	
	\textbf{L\"osung:} Nach 60 km, um 12:00 Uhr treffen sich beide Boote.
	
\end{document}
