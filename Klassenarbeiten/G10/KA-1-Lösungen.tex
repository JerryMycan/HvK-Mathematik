\documentclass[a4paper,12pt]{article}
\usepackage{tabularx}
\usepackage{amsmath}
\usepackage[utf8]{inputenc}
\usepackage{multicol}
\usepackage{cancel}
\usepackage{amsmath, amssymb, amsthm}
\usepackage{graphicx}
\usepackage{enumitem}
\usepackage{array}
\usepackage[left=2cm, right=2cm, top=2cm, bottom=2cm]{geometry}
\usepackage{fancyhdr}
\usepackage{xfp}
\usepackage{pgf}

\usepackage{graphicx}
\usepackage{fancyhdr}
\setlength{\headheight}{28pt} % genug Platz für das Logo
\pagestyle{fancy}
\fancyhf{} % alles leeren
\fancyhead[L]{\includegraphics[height=1.2cm]{logo.png}}
\fancyhead[C]{\small Klassenarbeit – Rechnen mit Potenzen \ (Kl. G10B)}
\fancyhead[R]{\small Name:\ \rule{2.8cm}{0.4pt}}
\fancyfoot[C]{\thepage}

\fancyfoot[C]{Seite \thepage \enspace\textbullet\enspace J.\,Mycan \textcopyright~2025 *Klassenarbeit 45 min.*}

\renewcommand{\footrulewidth}{0.4pt}




%\pagestyle{fancy}
%\lhead{Klassenarbeit 45min.}
%\chead{Heinrich-von-Kleist-Schule}
%\rhead{Mathematik - G8A}
%\lfoot{}
%\cfoot{Seite \thepage}
%\rfoot{}

\newcommand{\punkteA}{6}
\newcommand{\punkteB}{6}
\newcommand{\punkteC}{6}
\newcommand{\punkteD}{18}
\newcommand{\punkteE}{6}
%\newcommand{\punkteF}{12}

\newcommand{\maxSumme}{42}
\newcommand{\noteEinsMin}{\fpeval{round(\maxSumme * 0.95,0)}}
\newcommand{\noteZweiMin}{\fpeval{round(\maxSumme * 0.80,0)}}
\newcommand{\noteDreiMin}{\fpeval{round(\maxSumme * 0.60,0)}}
\newcommand{\noteVierMin}{\fpeval{round(\maxSumme * 0.45,0)}}
\newcommand{\noteFunfMin}{\fpeval{round(\maxSumme * 0.20,0)}}
\newcommand{\noteSechsMin}{0}

\newcommand{\summe}{%
	\pgfmathparse{\punkteA + \punkteB + \punkteC + \punkteD + \punkteE}%
	\pgfmathprintnumber{\pgfmathresult}}

\begin{document}
	
%	\begin{center}
%		\textbf{Klassenarbeit - Lineare Funktionen und LGS}
%	\end{center}
	
%	\textbf{Vor- und Nachname:} \underline{\hspace{10cm}}\\[0.1cm]
\vspace{2cm}
Lösungen! \\[1cm]
	
	% -------------------------------------------------
	% Aufgabe 1
	% -------------------------------------------------
\textbf{Aufgabe 1 (6 Punkte)}\\
Wandle \emph{ohne Rundung} in die Form \(a\cdot 10^{n}\) mit \(1\le a<10\) um.

\begin{multicols}{2}
	\begin{enumerate}[label=\alph*)]
		\item \(0{,}00052=5{,}2\cdot10^{-4}\)
		\item \(0{,}0063=6{,}3\cdot10^{-3}\)
		\item \(4\,700\,000=4{,}7\cdot10^{6}\)
		\item \(0{,}0000008=8\cdot10^{-7}\)
		\item \(125\,000=1{,}25\cdot10^{5}\)
		\item \(91{,}2=9{,}12\cdot10^{1}\)
	\end{enumerate}
\end{multicols}
	
	\vspace{2 cm}
	
	% -------------------------------------------------
	% Aufgabe 2
	% -------------------------------------------------
\textbf{Aufgabe 2 (6 Punkte)}\\
Schreibe in normale Dezimalschreibweise um (\emph{ohne} Rundung).
% Lösung zu Aufgabe 2
\begin{multicols}{2}
	\begin{enumerate}[label=\alph*)]
		\item \(4{,}5\cdot 10^{3}=4500\)
		\item \(6{,}02\cdot 10^{-4}=0{,}000602\)
		\item \(9\cdot 10^{1}=90\)
		\item \(1{,}8\cdot 10^{-2}=0{,}018\)
		\item \(2{,}75\cdot 10^{5}=275000\)
		\item \(3{,}6\cdot 10^{0}=3{,}6\)
	\end{enumerate}
\end{multicols}


	
	
	\vspace{2cm}
	% -------------------------------------------------
	% Aufgabe 3
	% -------------------------------------------------
\textbf{Aufgabe 3 (6 Punkte)}\\
Vereinfache und berechne vollständig.
% Lösung zu Aufgabe 3
\begin{multicols}{2}
	\begin{enumerate}[label=\alph*)]
		\item \(2^{5}\cdot 2^{-2}\cdot 2 = 2^{5-2+1}=2^{4}=16\)
		\item \(3^{-2}:3^{-5}=3^{-2-(-5)}=3^{3}=27\)
		\item \(\bigl(5^{2}\cdot 5^{-1}\bigr)^{2}=\bigl(5^{2-1}\bigr)^{2}=5^{2}=25\)
		\item \(10^{-4}\cdot 10^{6}:10=10^{-4+6-1}=10^{1}=10\)
		\item \(4^{-1}\cdot 2^{2}=(2^{2})^{-1}\cdot 2^{2}=2^{-2}\cdot 2^{2}=2^{0}=1\)
		\item \(16^{\tfrac{3}{4}}=(2^{4})^{\tfrac{3}{4}}=2^{3}=8\)
	\end{enumerate}
\end{multicols}


	
	\newpage
	% -------------------------------------------------
	% Aufgabe 4
	% -------------------------------------------------
\textbf{Aufgabe 4 (18 Punkte)}\\
Vereinfache vollständig; schreibe ohne negative Exponenten und rationalisiere ggf. den Nenner. Es gelte \(a,b,x,y>0\).

% Lösung zu Aufgabe 4
\begin{multicols}{2}
	\begin{enumerate}[label=\alph*)]
		\item \(\sqrt{32\,a^{2}b}
		=\sqrt{16\cdot2}\,\sqrt{a^{2}}\,\sqrt{b}
		=4a\sqrt{2b}\)
		
		\item \(\dfrac{x^{\tfrac12}\cdot\sqrt{18x^{3}}}{\sqrt{2}}
		=\dfrac{x^{\tfrac12}\cdot 3x\sqrt{2x}}{\sqrt{2}}
		=3x^{\tfrac12+1}\sqrt{x}
		=3x^{2}\)
		
		\item \(\dfrac{\sqrt{32\,x^{6}}\cdot x^{\tfrac{3}{2}}}{\sqrt{2x}\cdot x}
		=\dfrac{4x^{3}\sqrt{2}\cdot x^{\tfrac{3}{2}}}{x\cdot \sqrt{2}\,\sqrt{x}}
		=\dfrac{4x^{\tfrac{9}{2}}}{x^{\tfrac{3}{2}}}
		=4x^{3}\)
		
		\item \(\dfrac{\sqrt{12x}}{\sqrt{3}}\cdot \sqrt{x}
		=\dfrac{\sqrt{12x}\,\sqrt{x}}{\sqrt{3}}
		=\dfrac{\sqrt{12x^{2}}}{\sqrt{3}}
		=\sqrt{\tfrac{12x^{2}}{3}}
		=\sqrt{4x^{2}}
		=2x\)
		
		\item \(\left(\dfrac{\sqrt{9a}-\sqrt{4a}}{\sqrt{a}}\right)^{2}
		=\left(\dfrac{3\sqrt{a}-2\sqrt{a}}{\sqrt{a}}\right)^{2}
		=1^{2}=1\)
		
		\item \(\dfrac{a^{2n+1}}{b^{-3n}}\cdot \dfrac{b^{-3n}}{a^{2n+1}}
		=\dfrac{a^{2n+1}\,b^{3n}}{1}\cdot \dfrac{b^{-3n}}{a^{2n+1}}
		=1\)
	\end{enumerate}
\end{multicols}

	
	% -------------------------------------------------
	% Aufgabe 4
	% -------------------------------------------------
%	\textbf{Aufgabe 5 (15 Punkte)}\\
%	Vereinfache vollständig; schreibe ohne negative Exponenten und rationalisiere ggf. den Nenner. Es gelte \(a,b,x,y>0\).
%	\begin{multicols}{2}
%		\begin{enumerate}[label=\alph*)]
%			\item \( \dfrac{a^{2n+1}}{b^{-3n}}\cdot \dfrac{b^{-3n-5}}{a^{2n+6}} \)
%			\item \( \dfrac{y^{3n-5}}{x^{n+3}}\cdot \dfrac{x^{2n+5}}{y^{2n-4}} \)
%		\end{enumerate}
%	\end{multicols}
	
	
	\vspace{1.5cm}
	% -------------------------------------------------
	% Aufgabe 5
	% -------------------------------------------------
	\textbf{Aufgabe 5 (6 + 6* Punkte)}\\
% Lösungen zu Aufgabe 5 (6 + 6* Punkte)

\begin{enumerate}[label=\alph*)]
	
	\item \[
	\left(\frac{36-a^{2}}{(a+6)(6-a)}\right)^{2}
	\cdot
	\left(\frac{12+2a}{\,6-a\,}\right)^{2}
	\cdot
	\left(\frac{6-a}{\,2(6+a)\,}\right)^{2}
	\]

	\[
	\begin{aligned}
	&=\left(\frac{(6-a)(6+a)}{(6+a)(6-a)}\right)^{2}\!\cdot
	\left(\frac{2(6+a)}{6-a}\right)^{2}\!\cdot
	\left(\frac{6-a}{2(6+a)}\right)^{2}\\[2pt]
	&=1^{2}\cdot 1^{2}\cdot 1^{2}\\
	&=1
\end{aligned}
	\]
	
	\item[b$^{*}$)] 
% Lösung
\[
\left(\frac{x^{2}-2x+1}{x-3}\right)^{4}:
\left(\frac{x-1}{x^{2}-9}\right)^{4}
\cdot
\left(\frac{3}{(x+3)(x-1)}\right)^{4}
\]

\[
\begin{aligned}
	&=\left[
	\frac{x^{2}-2x+1}{x-3}\cdot
	\frac{x^{2}-9}{x-1}\cdot
	\frac{3}{(x+3)(x-1)}
	\right]^{\!4}
	\\[2pt]
	&=\left[
	\frac{(x-1)^{2}}{x-3}\cdot
	\frac{(x-3)(x+3)}{x-1}\cdot
	\frac{3}{(x+3)(x-1)}
	\right]^{\!4}
	\quad
	\\[2pt]
	&=\left[
	\cancel{\frac{(x-1)^{2}}{\,}}{\phantom{(x-3)}}\frac{\cancel{(x-3)}(x+3)}{\cancel{(x-1)}}\cdot
	\frac{3}{(x+3)\, \cancel{(x-1)}}
	\right]^{\!4}
	=\left[\,3\,\right]^{4}
	=81
\end{aligned}
\]

\end{enumerate}




%		\item \[
%		\left(\frac{a^{2}-8a+16}{a-4}\right)^{2}
%		\cdot
%		\left(\frac{2a+2}{a^{2}-1}\right)^{2}
%		\cdot
%		\left(\frac{a-1}{2\,(a-4)}\right)^{2}
%		\]
		% Ergebnis: 1, da a^2-8a+16=(a-4)^2 und a^2-1=(a-1)(a+1), (2a+2)=2(a+1).
		% DGM: a \neq 4,\, \pm 1
	%\vspace{5cm}
	
	
	
	\vspace{6cm}
	\textbf{Auswertungstabelle:}
	\begin{center}
		\begin{tabular}{|c|c|c|c|c|c|c|c|}
			\hline
			Aufgabe & 1 & 2 & 3 & 4 & 5 &  Summe\\
			\hline
			Punkte & \text{\ / \punkteA} & \text{\ / \punkteB} & \text{\ / \punkteC} & \text{\ / \punkteD} & \text{\ / \punkteE} & \text{\ / \summe}\\
			\hline
		\end{tabular}
	\end{center}
	
	\textbf{Notenschlüssel:}
	\begin{center}
		\begin{tabular}{|c|c|c|c|c|c|c|}
			\hline
			Note & 1 & 2 & 3 & 4 & 5 & 6 \\
			\hline
			Prozent \% & 100--95 & 94--80 & 79--60 & 59--45 & 44--16 & 15--0 \\
			\hline
			Punkte & \maxSumme{}--\noteEinsMin{} & \fpeval{\noteEinsMin-1}--\noteZweiMin{} & \fpeval{\noteZweiMin-1}--\noteDreiMin{} & \fpeval{\noteDreiMin-1}--\noteVierMin{} & \fpeval{\noteVierMin-1}--\noteFunfMin{} & \fpeval{\noteFunfMin-1}--\noteSechsMin{} \\
			\hline
		\end{tabular}
	\end{center}
	
	\vspace{2cm}
	\textbf{Kenntnisnahme eines Elternteils:} \hrulefill \hfill \textbf{Note:} \hrulefill
	
\end{document}
