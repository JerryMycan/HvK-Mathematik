\documentclass[a4paper,12pt]{article}
\usepackage{tabularx}
\usepackage{amsmath, amssymb, amsthm}
\usepackage[utf8]{inputenc}
\usepackage{geometry}
\usepackage{fancyhdr}
\usepackage{graphicx}

\geometry{left=2cm, right=2cm, top=2cm, bottom=2cm}
\pagestyle{fancy}
\lhead{L\"osungen - Klassenarbeit}
\chead{Heinrich-von-Kleist-Schule}
\rhead{Mathematik - G8A}
\lfoot{}
\cfoot{Seite \thepage}
\rfoot{}

\begin{document}
	
	\section*{Ausführliche Lösungen}
	
	\textbf{Aufgabe 1}\newline
	\textit{Die Graphen sind zeichnerisch zu lösen.}
	
	\textbf{Aufgabe 2}
	\begin{itemize}
		\item[a)] \textbf{Steigung berechnen:}
		\[m=\frac{8-4}{5-1}=\frac{4}{4}=1\]
		\textbf{Punkt einsetzen (A):}
		\[4=1\cdot1+b\quad\Rightarrow\quad b=3\]
		Funktionsgleichung: \(f(x)=x+3\).
		
		\item[b)] \textbf{Nullstelle berechnen:}
		\[0=x+3\quad\Rightarrow\quad x=-3\]
		Die Nullstelle liegt bei \(x=-3\).
		
		\item[c)] \textbf{Punktprobe für C(3|6):}
		\[f(3)=3+3=6\]
		Da der Punkt die Gleichung erfüllt, liegt er auf der Geraden.
	\end{itemize}
	
	\textbf{Aufgabe 3}
	\begin{align*}
		-0,5x+5 &= 1,5x-3\\
		5+3 &= 1,5x+0,5x\\
		8 &= 2x\\
		x &= 4\\[5pt]
		y &= 1,5\cdot 4-3= 6 - 3 =3\\
		oder\\
		y &= -0,5 \cdot 4 + 5 =-2 + 5 = 3
	\end{align*}
	Schnittpunkt: \(\left(4|3\right)\).
	
	\textbf{Aufgabe 4}
	\begin{itemize}
		\item[a)] \textbf{Parallel bedeutet gleiche Steigung \((-3)\), y-Achsenabschnitt \(b=6\):}
		\[0=-3x+6\quad\Rightarrow\quad x=2\]
		Die Nullstelle ist bei \(x=2\).
		
		\item[b)] \textbf{Steigung parallel zu \(f(x)=4x-1\) ist \(4\), Nullstelle \(-4\):}
		\[0=4\cdot(-4)+b\quad\Rightarrow\quad b=16\]
		Funktionsgleichung: \(g(x)=4x+16\).
		
		\item[c)] \textbf{Orthogonale Gerade:} Steigung \(m_2=-\frac{1}{m_1}=-\frac{1}{4}\), durch Punkt \((0|2)\):
		\[y=-\frac{1}{4}x+2\]
		Funktionsgleichung: \(h(x)=-\frac{1}{4}x+2\).
	\end{itemize}
	
	\textbf{Aufgabe 5}
	\begin{itemize}
		\item[a)] \textbf{Funktionsgleichung mit täglichem Verlust von 75 Litern:}
		\[f(t)=1500-75t\]
		
		\item[b)] \textbf{Wassermenge nach 12 Tagen berechnen:}
		\[f(12)=1500-75\cdot12=1500-900=600\]\newline
		Nach 12 Tagen sind noch 600 Liter im Behälter.
		
		\item[c)] \textbf{Zeitpunkt, wann der Behälter leer ist (Nullstelle):}
		\[0=1500-75t\quad\Rightarrow\quad 75t=1500\quad\Rightarrow\quad t=20\]
		Nach 20 Tagen ist der Behälter leer.
	\end{itemize}
	
\end{document}


