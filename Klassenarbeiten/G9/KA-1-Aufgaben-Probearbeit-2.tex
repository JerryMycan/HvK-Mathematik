\documentclass[a4paper,12pt]{article}
\usepackage{tabularx}
\usepackage{amsmath}
\usepackage[utf8]{inputenc}
\usepackage{amsmath, amssymb, amsthm}
\usepackage{graphicx}
\usepackage{enumitem}
\usepackage{multicol}
\usepackage{array}
\usepackage[left=2cm, right=2cm, top=2cm, bottom=2cm]{geometry}
\usepackage{fancyhdr}
\usepackage{xfp}
\usepackage{pgf}

\usepackage{graphicx}
\usepackage{fancyhdr}
\setlength{\headheight}{28pt} % genug Platz für das Logo
\pagestyle{fancy}
\fancyhf{} % alles leeren
\fancyhead[L]{\includegraphics[height=1.2cm]{logo.png}}
\fancyhead[C]{\small Probe-Klassenarbeit – LGS und Wurzelgesetze \ (Kl. G9A)}
%\fancyhead[R]{\small Name:\ \rule{2.8cm}{0.4pt}}
\fancyfoot[C]{\thepage}

\fancyfoot[C]{Seite \thepage \enspace\textbullet\enspace J.\,Mycan \textcopyright~2025}

\renewcommand{\footrulewidth}{0.4pt}
%\pagestyle{fancy}
%\lhead{Probe-Klassenarbeit 45min.}
%\chead{Heinrich-von-Kleist-Schule}
%\rhead{Mathematik - G9A}
%\lfoot{}
%\cfoot{Seite \thepage}
%\rfoot{}

\newcommand{\punkteA}{8}
\newcommand{\punkteB}{8}
\newcommand{\punkteC}{12}
\newcommand{\punkteD}{6}
\newcommand{\punkteE}{12}

\newcommand{\maxSumme}{44}
\newcommand{\noteEinsMin}{\fpeval{round(\maxSumme * 0.95,0)}}
\newcommand{\noteZweiMin}{\fpeval{round(\maxSumme * 0.80,0)}}
\newcommand{\noteDreiMin}{\fpeval{round(\maxSumme * 0.60,0)}}
\newcommand{\noteVierMin}{\fpeval{round(\maxSumme * 0.45,0)}}
\newcommand{\noteFunfMin}{\fpeval{round(\maxSumme * 0.20,0)}}
\newcommand{\noteSechsMin}{0}

\newcommand{\summe}{%
	\pgfmathparse{\punkteA + \punkteB + \punkteC + \punkteD + \punkteE}%
	\pgfmathprintnumber{\pgfmathresult}}

\begin{document}
	
%	\begin{center}
%		\textbf{Probe-Klassenarbeit - LGS und Wurzelgesetze}
%	\end{center}
	
	\textbf{Vor- und Nachname:} \underline{\hspace{10cm}}\\[0.1cm]
	Die Lösungen sowie Lösungswege sollten klar strukturiert und gut nachvollziehbar sein. \\[0.1cm]
	
	\textbf{Aufgabe 1 (8 Punkte)}\\
	Gegeben ist ein Koordinatensystem, in dem die Graphen der linearen Funktionen \(f\) und \(g\) eingezeichnet sind (siehe Abbildung~1).\\
	Bestimme den Schnittpunkt der beiden Funktionen \emph{rechnerisch}.
	Bestimme dazu zunächst aus der Abbildung die Funktionsgleichungen von \(f\) und \(g\).
	Stelle anschließend das lineare Gleichungssystem
%	\[
%	\begin{cases}
%		y = f(x)\\
%		y = g(x)
%	\end{cases}
%	\]
	auf und löse es.
	Gib das Ergebnis als Punkt \(S(x_S \mid y_S)\) an.

	\begin{figure}[h!]
		\centering
		\includegraphics[width=0.8\textwidth]{plot-2.png}
		\caption{Graph der linearen Funktionen \(f\) und \(g\)}
	\end{figure}

	
	\vspace{1cm}
	
\textbf{Aufgabe 2 (8 Punkte)}\\
Im \emph{Copyshop} kosten \emph{Farbausdrucke} und \emph{Schwarzweißkopien} unterschiedlich viel. 
Kundin A bezahlt für \emph{3 Farbausdrucke und 4 Schwarzweißkopien} insgesamt \(\,1{,}60\,€\).
Kunde B bezahlt für \emph{2 Farbausdrucke und 7 Schwarzweißkopien} insgesamt \(\,1{,}50\,€\).

\begin{enumerate}[label=\alph*)]
	\item Lege geeignete Variablen fest und stelle aus den Angaben ein lineares Gleichungssystem auf.
	\item Löse das Gleichungssystem \emph{rechnerisch} und gib die beiden Einzelpreise an.
\end{enumerate}

\vspace{1cm}
		
\textbf{Aufgabe 3 (12 Punkte)}\\
Gegeben ist das lineare Gleichungssystem in Matrixschreibweise
\[
A\vec{x}=\vec{b},\quad
A=\begin{pmatrix}
	1 & 1 & 0\\
	2 & -1 & 1\\
	1 & 2 & 1
\end{pmatrix},\quad
\vec{b}=\begin{pmatrix}
	3\\
	3\\
	8
\end{pmatrix}.
\]

	Löse das Gleichungssystem rechnerisch (z.\,B. mit dem Gauß-Verfahren).
	%\item Prüfe dein Ergebnis durch Einsetzen in \(A\vec{x}=\vec{b}\).

% Hinweis: Die Lösung besteht aus einstelligen ganzen Zahlen.



	\newpage


\textbf{Aufgabe 4 (6 Punkte)}\\
Vereinfache so weit wie möglich, indem du vollständige Quadrate aus der Wurzel ziehst.

\begin{multicols}{2}
	\begin{enumerate}[label=\alph*)]
		\item \(\sqrt{120}\)
		\item \(\sqrt{27}\)
		\item \(\sqrt{578}\)
		\item \(\sqrt{75}\)
		\item \(\sqrt{98}\)
		\item \(\sqrt{200}\)
	\end{enumerate}
\end{multicols}

\vspace{1cm}

\textbf{Aufgabe 5 (12 Punkte)}\\
Vereinfache vollständig \emph{(ggf. Nenner rationalisieren)}. Es gelte \(a,b,x,y>0\).

\begin{multicols}{2}
	\begin{enumerate}[label=\alph*)]
		% leicht
		\item \(\sqrt{32\,x^{2}}\)
		\item \(\sqrt{\dfrac{50\,a^{2}}{2}}\)
		
		% mittel
		\item \(\dfrac{\sqrt{24\,a^{5}}}{\sqrt{6a}}\)
		\item \(\sqrt{9x}\cdot\sqrt{8x^{3}}\)
		
		% anspruchsvoller
		\item \(4\sqrt{3x}-\sqrt{27x}+2\sqrt{12x}\)
		\item \(\dfrac{1}{\sqrt{a}+\sqrt{b}}+\dfrac{1}{\sqrt{a}-\sqrt{b}}\)
	\end{enumerate}
\end{multicols}


%
%\textbf{Aufgabe 6 (Punkte)}\\
%Vereinfache vollständig \emph{(ggf. Nenner rationalisieren)}. Es gelte \(a,b,x,y>0\).
%
%\begin{multicols}{2}
%	\begin{enumerate}[label=\alph*)]
%		\item \(\;\dfrac{\sqrt{72\,a^{3}b^{5}}}{\sqrt{8ab}}\cdot\sqrt{b}\)
%		\item \(\;\dfrac{\sqrt{27x^{3}}}{3\sqrt{x}}\cdot\dfrac{\sqrt{12x}}{\sqrt{3}}\)
%		\item \(\;\dfrac{\sqrt{18a}-\sqrt{8a}}{\sqrt{2a}}\)
%		\item \(\;\left(\dfrac{\sqrt{a}}{\sqrt[4]{a^{3}}}\right)^{3}\cdot\dfrac{1}{\sqrt{a}}\)
%		\item \(\;\dfrac{3}{\sqrt{a}+\sqrt{b}}-\dfrac{1}{\sqrt{a}-\sqrt{b}}\)
%		\item \(\;\dfrac{\sqrt{50\,x^{3}y^{5}}}{\sqrt{2xy}}\cdot\dfrac{\sqrt[4]{x^{2}}}{\sqrt{y}}\)
%	\end{enumerate}
%\end{multicols}
\vspace{5 cm}
	\textbf{Auswertungstabelle:}
\begin{center}
	\begin{tabular}{|c|c|c|c|c|c|c|c|}
		\hline
		Aufgabe & 1 & 2 & 3 & 4 & 5 &  Summe\\
		\hline
		Punkte & \text{\ / \punkteA} & \text{\ / \punkteB} & \text{\ / \punkteC} & \text{\ / \punkteD} & \text{\ / \punkteE} & \text{\ / \summe}\\
		\hline
	\end{tabular}
\end{center}

\textbf{Notenschlüssel:}
\begin{center}
	\begin{tabular}{|c|c|c|c|c|c|c|}
		\hline
		Note & 1 & 2 & 3 & 4 & 5 & 6 \\
		\hline
		Prozent \% & 100--95 & 94--80 & 79--60 & 59--45 & 44--16 & 15--0 \\
		\hline
		Punkte & \maxSumme{}--\noteEinsMin{} & \fpeval{\noteEinsMin-1}--\noteZweiMin{} & \fpeval{\noteZweiMin-1}--\noteDreiMin{} & \fpeval{\noteDreiMin-1}--\noteVierMin{} & \fpeval{\noteVierMin-1}--\noteFunfMin{} & \fpeval{\noteFunfMin-1}--\noteSechsMin{} \\
		\hline
	\end{tabular}
\end{center}

\vspace{2cm}
\textbf{Kenntnisnahme eines Elternteils:} \hrulefill \hfill \textbf{Note:} \hrulefill
	
	
\end{document}
