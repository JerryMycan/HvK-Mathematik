\documentclass[a4paper,12pt]{article}
\usepackage{tabularx}
\usepackage{amsmath, amssymb, amsthm}
\usepackage[utf8]{inputenc}
\usepackage{graphicx}
\usepackage{array}
\usepackage[left=2cm, right=2cm, top=2cm, bottom=2cm]{geometry}
\usepackage{fancyhdr}

% Kopf- und Fußzeile
\pagestyle{fancy}
\lhead{Musterlösung - Probeklassenarbeit}
\chead{Heinrich-von-Kleist-Schule}
\rhead{Mathematik - G8A}
\lfoot{}
\cfoot{Seite \thepage}
\rfoot{}

\begin{document}
	
	\begin{center}
		\textbf{Musterlösung zur Probeklassenarbeit:}\
		\textbf{Un- und Gleichungen, Bruchgleichungen, quadratische Ergänzung, Wahrscheinlichkeitsrechnung}\
		\vspace{0.2cm}
	\end{center}
	
	\section*{Aufgabe 1: Gleichungen und Ungleichungen lösen}
	
	\textbf{a)} \quad $(3x - 7)^2 = 9x^2 - 42x + 49$
	\begin{align*}
		9x^2 - 42x + 49 &= 9x^2 - 42x + 49 \quad \text{(Binomische Formel)} \\
		0 &= 0 \quad \text{(Wahr für alle $x$, unendlich viele Lösungen)}
	\end{align*}
	
	\textbf{b)} \quad $x(2x + 15) + 40 = (x + 4)^2$
	\begin{align*}
		2x^2 + 15x + 40 &= x^2 + 8x + 16 \quad \text{(Binomische Formel auflösen)} \\
		2x^2 + 15x + 40 - x^2 - 8x - 16 &= 0 \\
		x^2 + 7x + 24 &= 0 \\
		(x + 3)(x + 8) &= 0 \\
		x_1 = -3, \quad x_2 = -8
	\end{align*}
	
	\textbf{c)} \quad $\frac{x + 2}{3} \leq \frac{3x - 4}{5}$
	\begin{align*}
		5(x + 2) &\leq 3(3x - 4) \quad \text{(Hauptnenner 15)} \\
		5x + 10 &\leq 9x - 12 \\
		10 + 12 &\leq 9x - 5x \\
		22 &\leq 4x \\
		\frac{22}{4} &\leq x \\
		\frac{11}{2} &\leq x
	\end{align*}
	
	\section*{Aufgabe 2: Bruchgleichungen lösen}
	
	\textbf{a)} \quad $\frac{x}{5} + \frac{4}{x} = 3$
	\begin{align*}
		x^2 + 20 &= 15x \quad \text{(Hauptnenner $5x$)} \\
		x^2 - 15x + 20 &= 0 \\
		(x - 5)(x - 4) &= 0 \\
		x_1 = 5, \quad x_2 = 4
	\end{align*}
	
	\textbf{b)} und \textbf{c)} analog gelöst.
	
	\section*{Aufgabe 3: Quadratische Ergänzung}
	
	\textbf{a)} \quad $x^2 + 8x$
	\begin{align*}
		x^2 + 8x + 16 - 16 &= (x+4)^2 - 16
	\end{align*}
	
	Weitere Teile analog.
	
	\section*{Aufgabe 4: Wahrscheinlichkeitsrechnung mit Kugeln}
	
	\textbf{Baumdiagramm für Aufgabe 4a:} (Baumdiagramm als Grafik einfügen.)
	
	\textbf{b)} Wahrscheinlichkeit für drei gleiche Farben:
	\begin{align*}
		P(\text{alle rot}) &= \frac{3}{10} \times \frac{2}{9} \times \frac{1}{8} = \frac{6}{720} = \frac{1}{120}
	\end{align*}
	
	Weitere Berechnungen analog.
	
	\section*{Aufgabe 5: Wahrscheinlichkeitsrechnung mit Bonbons}
	
	\textbf{b)} Wahrscheinlichkeit für zwei Schokoladen:
	\begin{align*}
		P(\text{Schoko, Schoko}) &= \frac{7}{18} \times \frac{6}{17} = \frac{42}{306} = \frac{7}{51}
	\end{align*}
	
	Weitere Berechnungen analog.
	
\end{document}
