\documentclass[a4paper,12pt]{article}
\usepackage{tabularx}
\usepackage{amsmath}
\usepackage[utf8]{inputenc}
\usepackage{amsmath, amssymb, amsthm}
\usepackage{graphicx}
\usepackage{array}
\usepackage[left=2cm, right=2cm, top=2cm, bottom=2cm]{geometry}
\usepackage{fancyhdr}

% Kopf- und Fußzeile
\pagestyle{fancy}
\lhead{Klassenarbeit 45min.}
\chead{Heinrich-von-Kleist-Schule}
\rhead{Mathematik - G8A}
\lfoot{}
\cfoot{Seite \thepage}
\rfoot{}

\begin{document}
	
	\begin{center}
		%\textbf{Klassenarbeit N°1 G8}\\[0.3cm]
		\textbf{Un- und Gleichungen, Bruchgleichungen, quadratische Ergänzung, Wahrscheinlichkeitsrechnung}\\[0.2cm]
		%\textbf{© J. Mycan}\\[0.5cm]
	\end{center}
	
	\textbf{Vor- und Nachname:} \underline{\hspace{10cm}}\\%[0.5cm]
	
	%\vspace{1cm}
	
	\textbf{Aufgabe 1} \hfill (15 Punkte)\\
	Löse folgende Gleichungen und Ungleichung nach \( x \):
\[
\renewcommand{\arraystretch}{1.5} % Erhöht den Zeilenabstand für bessere Lesbarkeit
\begin{tabularx}{\textwidth}{>{\centering\arraybackslash}X 
		>{\centering\arraybackslash}X 
		>{\centering\arraybackslash}X}
	\textbf{a)} \( (2x - 3)^2 = 4x^2 - 12x + 9 \) &
	\textbf{b)} \( (x + 5)^2 = 61 \) &
	\textbf{c)} \( \frac{x - 3}{4} \geq \frac{2x + 1}{6} \)
\end{tabularx}
\]
	
	\textbf{Aufgabe 2} \hfill (2,3,2 Punkte)\\
	Löse folgende Bruchgleichungen.
\[
\renewcommand{\arraystretch}{1.5}
\begin{tabularx}{\textwidth}{>{\centering\arraybackslash}X 
		>{\centering\arraybackslash}X 
		>{\centering\arraybackslash}X}
	\textbf{a)} \quad \( \frac{x}{4} + \frac{3}{x} = 2 \) &
	\textbf{b)} \quad \( \frac{2x}{x+1} - \frac{3}{x-1} = \frac{1}{x^2 - 1} \) &
	\textbf{c)} \quad \( \frac{2}{x} + \frac{3}{x-1} = \frac{5}{x^2 - x} \)
\end{tabularx}
\]
	
	
	\textbf{Aufgabe 3} \hfill (6 Punkte)\\
	Fasse die folgenden Terme mithilfe der quadratischen Ergänzung zu einem Binom zusammen.
	
\[
\renewcommand{\arraystretch}{1.5}
\begin{tabularx}{\textwidth}{>{\centering\arraybackslash}X 
		>{\centering\arraybackslash}X 
		>{\centering\arraybackslash}X
		>{\centering\arraybackslash}X}
	\textbf{a)} \quad \( x^2 + 6x \) &
	\textbf{b)} \quad \( x^2 - 10x + 7 \) &
	\textbf{c)} \quad \( 2x^2 + 8x - 5 \) &
	\textbf{d)} \quad \( 3x^2 - 12x + 11 \)
\end{tabularx}
\]
	
	\textbf{Aufgabe 4} \hfill (9 Punkte)\\
Ein Beutel enthält **jeweils 2 rote, 3 blaue und 5 grüne Kugeln**. Es werden **nacheinander drei Kugeln ohne Zurücklegen** gezogen.
\[
\begin{aligned}
	\textbf{a)} & \quad \text{Zeichne ein Baumdiagramm, das alle möglichen Ziehungen darstellt.} \\
	\textbf{b)} & \quad \text{Wie groß ist die Wahrscheinlichkeit, dass alle drei Kugeln dieselbe Farbe haben?} \\
	\textbf{c)} & \quad \text{Wie groß ist die Wahrscheinlichkeit, dass genau zwei Kugeln dieselbe Farbe haben und die dritte Kugel eine andere Farbe?} \\
	\textbf{d)} & \quad \text{Wie groß ist die Wahrscheinlichkeit, dass alle drei Kugeln unterschiedliche Farben haben?} 
\end{aligned}
\]
	
	\textbf{Aufgabe 5} \hfill (9 Punkte)\\
In einer Kiste befinden sich **8 Schokoladen, 5 Karamell- und 7 Erdbeerbonbons**. Ein Kind zieht **nacheinander zwei Bonbons ohne Zurücklegen**.
\[
\begin{aligned}
	\textbf{a)} & \quad \text{Zeichne ein Baumdiagramm für die möglichen Ziehungen.} \\
	\textbf{b)} & \quad \text{Wie groß ist die Wahrscheinlichkeit, dass beide Bonbons Schokolade sind?} \\
	\textbf{c)} & \quad \text{Wie groß ist die Wahrscheinlichkeit, dass das erste Bonbon Schokolade und das zweite Erdbeere ist?} \\
	\textbf{d)} & \quad \text{Wie groß ist die Wahrscheinlichkeit, dass mindestens ein Bonbon Karamell ist?} 
\end{aligned}
\]

	
	\textbf{Auswertungstabelle:}
	
	\begin{center}
		\begin{tabular}{|c|c|c|c|c|c|c|c|c|c|c|}
			\hline
			Aufgabe & 1 & 2 & 3 & 4 & 5 & 6 & 7 & 8 & Summe & Note \\
			\hline
			Punkte  &  &  &  &  &  &  &  &  &  &  \\
			\hline
		\end{tabular}
	\end{center}
	
	\vspace{1cm}
	
	\textbf{Notenschlüssel:}
	
	\begin{center}
		\begin{tabular}{|c|c|c|c|c|c|c|}
			\hline
			Note & 1 & 2 & 3 & 4 & 5 & 6 \\
			\hline
			Punktzahl & 62--65 & 52--61 & 42--51 & 29--41 & 15--28 & 0--14 \\
			\hline
		\end{tabular}
	\end{center}
	
	\vspace{1cm}
	
	\textbf{Kenntnisnahme eines Elternteils:} \hrulefill \hfill \textbf{Note:} \hrulefill
	
\end{document}