\documentclass[a4paper,12pt]{article}
\usepackage{tabularx}
\usepackage{amsmath}
\usepackage[utf8]{inputenc}
\usepackage{multicol}
\usepackage{amsmath, amssymb, amsthm}
\usepackage{graphicx}
\usepackage{enumitem}
\usepackage{array}
\usepackage[left=2cm, right=2cm, top=2cm, bottom=2cm]{geometry}
\usepackage{fancyhdr}
\usepackage{xfp}
\usepackage{pgf}

\usepackage{graphicx}
\usepackage{fancyhdr}
\setlength{\headheight}{28pt} % genug Platz für das Logo
\pagestyle{fancy}
\fancyhf{} % alles leeren
\fancyhead[L]{\includegraphics[height=1.2cm]{logo.png}}
\fancyhead[C]{\small Klassenarbeit – Funktionale Zusammenhänge\\ Potenzfunktionen \ (Kl. G9A)}
\fancyhead[R]{\small Name:\ \rule{2.8cm}{0.4pt}}
\fancyfoot[C]{\thepage}

\fancyfoot[C]{Seite \thepage \enspace\textbullet\enspace J.\,Mycan \textcopyright~2025}

\renewcommand{\footrulewidth}{0.4pt}




%\pagestyle{fancy}
%\lhead{Klassenarbeit 45min.}
%\chead{Heinrich-von-Kleist-Schule}
%\rhead{Mathematik - G8A}
%\lfoot{}
%\cfoot{Seite \thepage}
%\rfoot{}

\newcommand{\punkteA}{6}
\newcommand{\punkteB}{6}
\newcommand{\punkteC}{6}
\newcommand{\punkteD}{18}
\newcommand{\punkteE}{6}
%\newcommand{\punkteF}{12}

\newcommand{\maxSumme}{42}
\newcommand{\noteEinsMin}{\fpeval{round(\maxSumme * 0.95,0)}}
\newcommand{\noteZweiMin}{\fpeval{round(\maxSumme * 0.80,0)}}
\newcommand{\noteDreiMin}{\fpeval{round(\maxSumme * 0.60,0)}}
\newcommand{\noteVierMin}{\fpeval{round(\maxSumme * 0.45,0)}}
\newcommand{\noteFunfMin}{\fpeval{round(\maxSumme * 0.20,0)}}
\newcommand{\noteSechsMin}{0}

\newcommand{\summe}{%
	\pgfmathparse{\punkteA + \punkteB + \punkteC + \punkteD + \punkteE}%
	\pgfmathprintnumber{\pgfmathresult}}

\begin{document}
	
%	\begin{center}
%		\textbf{Klassenarbeit - Lineare Funktionen und LGS}
%	\end{center}
	
%	\textbf{Vor- und Nachname:} \underline{\hspace{10cm}}\\[0.1cm]
	Die Lösungen sowie Lösungswege sollten klar strukturiert und gut nachvollziehbar sein.\\[0.1cm]
	
	% -------------------------------------------------
	% Aufgabe 1
	% -------------------------------------------------
	\textbf{Aufgabe 1 ( Punkte)}\\
	Gegeben ist eine quadratische Funktion der Form
	\[
	f(x) = x^2 + bx + c.
	\]
	Der Graph dieser Funktion verläuft durch die Punkte
	\[
	P(1 \mid -6) \quad \text{und} \quad Q(3 \mid -4).
	\]
	
	\begin{enumerate}
		\item Bestimme den Funktionsterm von \(f\), also die Werte von \(b\) und \(c\).
		
		\item Berechne die Nullstellen der Funktion aus Teilaufgabe a).
		
		\item Gib die gefundene Funktion in Scheitelpunktform an.
		
		\textbf{Hinweis:} Solltest du den Funktionsterm in Teilaufgabe a) nicht bestimmen können, 
		rechne in den Teilaufgaben b) und c) mit der Funktion
		\[
		f(x) = 2x^2 - 4x - 6
		\]
		weiter.
	\end{enumerate}
	
	% -------------------------------------------------
	% Aufgabe 2
	% -------------------------------------------------
	\textbf{Aufgabe 2 ( Punkte)}\\
	Ein Baum ist bei einem Drittel seiner Höhe abgeknickt. Die Spitze berührt den Boden in einem Abstand von
	\(10\,\text{m}\) vom Stamm (siehe Abbildung~\ref{fig:baum}).
	
	\begin{figure}[h]
		\centering
		\includegraphics[width=0.4\textwidth]{baum.png}
		\caption{Abgeknickter Baum}
		\label{fig:baum}
	\end{figure}
	
	\noindent
	Berechne, wie hoch der Baum ursprünglich war.\\
	
	% -------------------------------------------------
	% Aufgabe 3
	% -------------------------------------------------
	\textbf{Aufgabe 3 ( Punkte)}\\
	Ein Rechteck soll so konstruiert werden, dass seine Breite \(x\) Meter beträgt und seine Länge
	durch die Funktion
	\[
	L(x) = 12 - x
	\]
	gegeben ist. Die Fläche des Rechtecks ergibt sich damit durch
	\[
	A(x) = x \cdot L(x).
	\]
	
	\begin{enumerate}
		\item Stelle die Flächenfunktion \(A(x)\) explizit als quadratische Funktion auf.
		\item Bestimme durch geeignete Verfahren den Wert von \(x\), bei dem die Fläche maximal wird.
		\item Berechne die maximale Fläche.
	\end{enumerate}
	\vspace{1 cm}
	%\newpage
	% -------------------------------------------------
	% Aufgabe 4
	% -------------------------------------------------
	\textbf{Aufgabe 4 ( Punkte)}\\
	Gegeben sind die Funktionen
	\[
	f(x) = 5x^6 - 3x^2 - 4
	\]
	und
	\[
	g(x) = 2x^5+ 3x^3 - 3x.
	\]
	
	\begin{enumerate}
		\item[(a)] Untersuche die Graphen von \(f\) und \(g\) auf Symmetrie.
		\begin{itemize}
			\item Entscheide für jede der beiden Funktionen, ob ihr Graph \textbf{achsensymmetrisch zur \(y\)-Achse}, \textbf{punktsymmetrisch zum Ursprung} oder \textbf{keine der beiden Symmetrien} besitzt.
			\item Begründe deine Entscheidung jeweils mit einem passenden Merkmal der Funktionsgleichung oder mit einer Skizze.
		\end{itemize}
		
		\item[(b)] Beweise deine Vermutungen aus Teilaufgabe (a) rechnerisch:

	\end{enumerate}
	
	% -------------------------------------------------
	% Aufgabe 5
	% -------------------------------------------------
	\textbf{Aufgabe 5 ( Punkte)}\\
	Löse die folgenden quadratischen Gleichungen. Begründe jeden Schritt kurz (z.\,B. quadratische Ergänzung, Mitternachtsformel, Ausklammern).
	
	\begin{enumerate}
		\item[(a)] 
		\[
		x^2 - 6x + 5 = 0
		\]
		Löse die Gleichung und gib beide Lösungen an.
		
		\item[(b)]
		\[
		3x^2 + 12x = 9
		\]
		Bringe die Gleichung zunächst auf eine Seite und löse sie anschließend mit einem Verfahren deiner Wahl.
		
		\item[(c)] 
		\[
		(x - 4)(x + 2) = 10
		\]
		Multipliziere aus, forme eine quadratische Gleichung und löse sie.
		
		\item[(d)] Zusatz (optional):  
		Prüfe für jede Gleichung, ob die Lösungen reell sind oder ob keine reellen Lösungen existieren.
	\end{enumerate}

	% -------------------------------------------------
	% Aufgabe 6
	% -------------------------------------------------
	\textbf{Aufgabe 6 ( Punkte)}\\
	Gegeben ist eine quadratische Funktion, deren Nullstellen bei
	\[
	x_1 = -2 \quad \text{und} \quad x_2 = 4
	\]
	liegen. Außerdem verläuft der Graph der Funktion durch den Punkt
	\[
	P(1 \mid -9).
	\]
	
	\begin{enumerate}
		\item Bestimme den Funktionsterm der quadratischen Funktion in der in der normal Form.
		
		\item Gib anschließend die Funktionsgleichung in der Produktform

		an und überprüfe, ob sie mit deinem Ergebnis aus Teil (1) übereinstimmt.
	\end{enumerate}
	\vspace{6cm}

	
	\vspace{2cm}
	\textbf{Kenntnisnahme eines Elternteils:} \hrulefill \hfill \textbf{Note:} \hrulefill
	
\end{document}
