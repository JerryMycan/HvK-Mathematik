\documentclass[a4paper,12pt]{article}
\usepackage{amsmath, amssymb}
\usepackage[utf8]{inputenc}
\usepackage{geometry}
\geometry{a4paper, margin=2cm}
\usepackage{array}

\title{\textbf{Klassenarbeit Mathematik - Lösungen}}
\date{\today}
\author{Lösungen zu Bruchrechnung und Prozentrechnung}

\begin{document}
	
	\maketitle
	
	\section*{Lösungen zu Aufgabe 1}
	
	\begin{enumerate}
		\item[a)] Berechnung von $\frac{4}{7} + \frac{3}{5}$:
		\begin{align*}
			\frac{4}{7} + \frac{3}{5} &= \frac{4 \cdot 5}{7 \cdot 5} + \frac{3 \cdot 7}{5 \cdot 7} \\
			&= \frac{20}{35} + \frac{21}{35} \\
			&= \frac{41}{35}
		\end{align*}
		
		\item[b)] Berechnung von $\frac{11}{12} - \frac{5}{8}$:
		\begin{align*}
			\frac{11}{12} - \frac{5}{8} &= \frac{11 \cdot 2}{12 \cdot 2} - \frac{5 \cdot 3}{8 \cdot 3} \\
			&= \frac{22}{24} - \frac{15}{24} \\
			&= \frac{7}{24}
		\end{align*}
		
		\item[c)] Berechnung von $\frac{3}{4} \times \frac{7}{9}$:
		\begin{align*}
			\frac{3}{4} \times \frac{7}{9} &= \frac{3 \cdot 7}{4 \cdot 9} \\
			&= \frac{21}{36} \\
			&= \frac{7}{12}
		\end{align*}
		
		\item[d)] Berechnung von $\frac{6}{11} \div \frac{2}{5}$:
		\begin{align*}
			\frac{6}{11} \div \frac{2}{5} &= \frac{6}{11} \times \frac{5}{2} \\
			&= \frac{30}{22} \\
			&= \frac{15}{11}
		\end{align*}
	\end{enumerate}
	
	\section*{Lösungen zu Aufgabe 2}
	\begin{center}
		\renewcommand{\arraystretch}{1.5}
		\begin{tabular}{lll}
			a) $35\% = \frac{7}{20} = 0.35$ & b) $60\% = \frac{3}{5} = 0.6$ & c) $15.5\% = \frac{31}{200} = 0.155$ \\
			d) $82.25\% = \frac{329}{400} = 0.8225$ & e) $9\% = \frac{9}{100} = 0.09$ & f) $95.4\% = \frac{477}{500} = 0.954$ \\
		\end{tabular}
	\end{center}
	
	\section*{Lösungen zu Aufgabe 3}
	\begin{center}
		\renewcommand{\arraystretch}{1.5}
		\begin{tabular}{lll}
			a) $\frac{3}{8}$ von 240m = \underline{$90$}m & b) $\frac{4}{7}$ von 6,3dm = \underline{$3.6$}dm & \\ 
			c) $9\%$ von 2,8m = \underline{$0.252$}m & d) $\frac{1}{6}$ von \underline{$270$}min = 45min & \\ 
			e) $\frac{5}{9}$ von \underline{$145,8$}h sind 81h & f) $\frac{7}{15}$ von \underline{300}€ sind 140€€
		\end{tabular}
	\end{center}
	
	\section*{Lösungen zu Aufgabe 4}
	\begin{align*}
		\text{Mathematikliebhaber: } & \quad \frac{1}{4} \times 32 = 8 \\
		\text{Sportliebhaber: } & \quad \frac{1}{3} \times 32 = 10.67 \approx 11 \\
		\text{Kunstliebhaber: } & \quad 8 \\
		\text{Mindestens eine Vorliebe: } & \quad 8 + 11 + 8 = 27 \text{ Schüler}
	\end{align*}
	
	\section*{Lösungen zu Aufgabe 5}
	\begin{enumerate}
		\item[a)] \text{Anteil der Bewohner, die nicht Fahrrad oder ÖPNV nutzen:}
		\begin{align*}
			1 - \left( \frac{1}{10} + \frac{40}{100} \right) &= 1 - \left( \frac{1}{10} + \frac{2}{5} \right) \\
			&= 1 - \left( \frac{1}{10} + \frac{4}{10} \right) \\
			&= 1 - \frac{5}{10} = \frac{1}{2}
		\end{align*}
		
		\item[b)] \text{Berechnung der Gesamtbewohnerzahl:}
		\begin{align*}
			\frac{40}{100} x &= 3200 \\
			x &= 3200 \times \frac{100}{40} \\
			x &= 8000
		\end{align*}
		Also hat die Stadt insgesamt \textbf{8000 Bewohner}.
	\end{enumerate}
	
\end{document}