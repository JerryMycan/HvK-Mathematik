\documentclass[11pt,a4paper]{article}
\usepackage[margin=2.2cm]{geometry}
\usepackage[T1]{fontenc}
\usepackage[utf8]{inputenc}
\usepackage[ngerman]{babel}
\usepackage{amsmath,amssymb}
\usepackage{enumitem}
\usepackage{multicol}
\setlength\parindent{0pt}
\setlist[enumerate]{itemsep=4pt,topsep=4pt}

\begin{document}
	\begin{center}
		{\Large \textbf{Lösungsblatt – Vorbereitung auf die Klassenarbeit: Potenzen}}\\
		\small(Kurzlösungen mit Rechenschritten; Ergebnisse ohne negative Exponenten.)
	\end{center}
	
	% =================== Aufgabe 1 ===================
	\textbf{Aufgabe 1 – Wissenschaftliche Schreibweise}\\
	\begin{multicols}{3}
		\begin{enumerate}[label=\alph*)]
			\item \(3{,}45\cdot10^{6}\)
			\item \(7{,}8\cdot10^{-5}\)
			\item \(1{,}2\cdot10^{11}\)
			\item \(4{,}5\cdot10^{-1}\)
			\item \(9{,}006\cdot10^{6}\)
			\item \(5\cdot10^{-7}\)
			\item \(3{,}2\cdot10^{-3}\)
			\item \(5{,}6\cdot10^{12}\)
			\item \(1{,}2\cdot10^{-4}\)
			\item \(6{,}78\cdot10^{4}\)
			\item \(9{,}8\cdot10^{7}\)
			\item \(3{,}1\cdot10^{-10}\)
		\end{enumerate}
	\end{multicols}
	
	% =================== Aufgabe 2 ===================
	\textbf{Aufgabe 2 – Dezimalschreibweise}\\
	\begin{multicols}{3}
		\begin{enumerate}[label=\alph*)]
			\item \(4500\)
			\item \(0{,}0007\)
			\item \(120{,}3\)
			\item \(0{,}0908\)
			\item \(3\)
			\item \(275000\)
			\item \(0{,}006007\)
			\item \(81\)
			\item \(0{,}000005\)
			\item \(99990\)
			\item \(0{,}125\)
			\item \(3{,}402\)
		\end{enumerate}
	\end{multicols}
	
	% =================== Aufgabe 3 ===================
	\textbf{Aufgabe 3 – Potenzen (nur Zahlen)}\\
	\begin{multicols}{3}
		\begin{enumerate}[label=\alph*)]
			\item \(2^{3}\cdot2^{5}=2^{8}=256\)
			\item \(5^{7}:5^{2}=5^{5}=3125\)
			\item \((3^{4})^{2}=3^{8}=6561\)
			\item \(10^{-3}\cdot10^{5}=10^{2}=100\)
			\item \(4^{-2}=\dfrac{1}{16}\)
			\item \(27^{2/3}=(\sqrt[3]{27})^{2}=3^{2}=9\)
			\item \(32^{-3/5}=(\sqrt[5]{32})^{-3}=2^{-3}=\dfrac{1}{8}\)
			\item \(16^{1/4}\cdot 8^{2/3}=2\cdot4=8\)
			\item \(81^{3/4}:3^{1/2}=\dfrac{3^{3}}{3^{1/2}}=3^{5/2}=9\sqrt{3}\)
			\item \(\bigl(2^{5}-2^{3}\bigr):2^{2}=(32-8):4=6\)
			\item \(\bigl(5^{3}\cdot4^{3}\bigr):10^{3}=20^{3}:10^{3}=8\)
			\item \(\left(\tfrac{1}{8}\right)^{-2/3}=8^{2/3}=4\)
		\end{enumerate}
	\end{multicols}
	
	% =================== Aufgabe 4 ===================
	\textbf{Aufgabe 4 – Potenzen (mit Variablen)}\\
	\begin{multicols}{3}
		\begin{enumerate}[label=\alph*)]
			\item \(a^{3}\cdot a^{5}=a^{8}\)
			\item \(x^{7}:x^{2}=x^{5}\)
			\item \(\bigl(y^{4}\bigr)^{2}=y^{8}\)
			\item \(b^{-3}\cdot b^{5}=b^{2}\)
			\item \(c^{-2}=\dfrac{1}{c^{2}}\)
			\item \(\bigl(x^{6}y^{3}\bigr)^{1/2}=x^{3}y^{3/2}=x^{3}\sqrt{y^{3}}\)
			\item \(\bigl(a^{1/2}b^{3/2}\bigr):a^{1/4}=a^{1/4}b^{3/2}\)
			\item \(x^{2/3}\cdot x^{4/3}=x^{2}\)
			\item \((mn)^{3}:m^{2}n=m\,n^{2}\)
			\item \(\bigl(p^{5}-p^{3}\bigr):p^{2}=p^{3}-p\)
			\item \(\dfrac{a^{3}b^{-2}}{a^{-1}b^{4}}=a^{4}b^{-6}=\dfrac{a^{4}}{b^{6}}\)
			\item \(\dfrac{x^{-3/2}y}{x^{-1/2}y^{-2}}=x^{-1}y^{3}=\dfrac{y^{3}}{x}\)
		\end{enumerate}
	\end{multicols}
	
	% =================== Aufgabe 5 ===================
	\textbf{Aufgabe 5 – Wurzeln}\\
	\emph{a–f (nur Zahlen):}\quad
	\(2\sqrt{5},\;3\sqrt{5},\;6\sqrt{2},\;7\sqrt{3},\;6\sqrt{3},\;10\sqrt{3}\).\\[2pt]
	\emph{g–l (mit Variablen):}\quad
	\(3x\sqrt{2},\;5a\sqrt{2},\;2x^{2}\sqrt{3},\;6x,\;3a\sqrt{3b},\;2xy^{2}\sqrt{2xy}\).
	
	% =================== Aufgabe 6 ===================
	\textbf{Aufgabe 6 – Wurzelgesetze (rationalisieren, falls nötig)}\\
	\begin{enumerate}[label=\alph*)]
		\item \(\sqrt{18a^{2}b}=3a\sqrt{2b}\)
		\item \(\sqrt{12x}\cdot\sqrt{27x^{3}}=18x^{2}\)
		\item \(\dfrac{\sqrt{48a^{5}}}{\sqrt{3a}}=\sqrt{16a^{4}}=4a^{2}\)
		\item \(5\sqrt{2x}-2\sqrt{8x}+3\sqrt{18x}=10\sqrt{2x}\)
		\item \(\sqrt{\dfrac{9a^{3}b}{4a}}\cdot \dfrac{\sqrt{b}}{\sqrt{a^{2}}}
		=\dfrac{3}{2}\,b\)
		\item \(\dfrac{2}{\sqrt{5x}}+\dfrac{3\sqrt{x}}{\sqrt{20}}
		=\dfrac{(3x+4)\sqrt{5x}}{10x}\)
	\end{enumerate}
	
	% =================== Aufgabe 6 (irrational) ===================
	\textbf{Aufgabe 6 – Potenzen mit irrationalen Exponenten}\\
	\begin{multicols}{2}
		\begin{enumerate}[label=\alph*)]
			\item \(a^{\sqrt{2}}\cdot a^{3\sqrt{2}}=a^{4\sqrt{2}}\)
			\item \(\dfrac{x^{\pi}}{x^{2\pi}}=x^{-\pi}=\dfrac{1}{x^{\pi}}\)
			\item \(\bigl(b^{\sqrt{5}}\bigr)^{2}=b^{2\sqrt{5}}\)
			\item \(\bigl(\sqrt{a}\bigr)^{\pi}\cdot a^{\pi/2}=a^{\pi}\)
			\item \(\dfrac{(m^{\sqrt{3}})^{4}}{m^{2\sqrt{3}}}=m^{2\sqrt{3}}\)
			\item \(\dfrac{(pq)^{\sqrt{2}}}{p^{\sqrt{2}}}=q^{\sqrt{2}}\)
			\item \(9^{\sqrt{2}}\cdot 3^{\sqrt{2}}=(27)^{\sqrt{2}}\)
			\item \(\dfrac{16^{\sqrt{2}}}{2^{\sqrt{2}}}=8^{\sqrt{2}}\)
			\item \(\bigl(x^{-\sqrt{7}}\bigr)^{-1}=x^{\sqrt{7}}\)
			\item \(\sqrt[3]{a^{2\sqrt{2}}}\cdot\sqrt[3]{a^{\sqrt{2}}}=a^{\sqrt{2}}\)
			\item \(\left(\dfrac{\sqrt[4]{b}}{\sqrt{b}}\right)^{\sqrt{2}}=\dfrac{1}{b^{\sqrt{2}/4}}\)
			\item \(\dfrac{(10^{\sqrt{2}})^{3}}{(\sqrt{10})^{\sqrt{2}}}=10^{5\sqrt{2}/2}\)
		\end{enumerate}
	\end{multicols}
	
	% =================== Aufgabe 7 ===================
	\textbf{Aufgabe 7 – Lange Terme (gemischt)}\\
	\begin{enumerate}[label=\alph*)]
		\item \(\sqrt{75a^{3}b^{5}}\cdot\dfrac{\sqrt{12ab}}{3\sqrt{3a}}
		=\dfrac{10\,a\,b^{3}\sqrt{3a}}{3}\)
		\item \(\dfrac{\sqrt{32x^{5}}}{4\sqrt{2x}}+\dfrac{3\sqrt{18x^{3}}}{2\sqrt{8x}}
		=x^{2}+\dfrac{3}{4}x\)
		\item \(\dfrac{5}{\sqrt{a}+\sqrt{b}}-\dfrac{2}{\sqrt{a}-\sqrt{b}}
		=\dfrac{3\sqrt{a}-7\sqrt{b}}{a-b}\)
		\item \(\left(\dfrac{\sqrt{45x^{4}y}}{\sqrt{5xy}}\right):\left(\dfrac{\sqrt{9x^{2}}}{\sqrt{x}}\right)=x\)
		\item \(\sqrt{\dfrac{(12a^{3}b^{2})(27ab^{5})}{3a^{2}b}}\cdot\dfrac{1}{\sqrt{6ab}}
		=3b^{2}\sqrt{2ab}\)
		\item \(\sqrt{50x^{3}y^{5}}-2\sqrt{2xy}\cdot\sqrt{8x^{2}y^{3}}+\sqrt{200x^{3}y^{5}}
		=x y^{2}\sqrt{x}\,\bigl(15\sqrt{2y}-8\bigr)\)
	\end{enumerate}
	
	% =================== Aufgabe 8 ===================
	\textbf{Aufgabe 8 – „Sieht schwer aus – ist leicht“}\\
	\begin{enumerate}[label=\alph*)]
		\item \(\displaystyle
		\left(\frac{(3x+6)^{\,n-1}(3x+6)^{\,n+2}}{(x+2)^{\,2n+1}}\right):3^{\,2n}
		=\frac{[3(x+2)]^{2n+1}}{(x+2)^{2n+1}}:3^{2n}=3\)
		\item \(\displaystyle
		\left(\frac{9-a^{2}}{a+3}\right)^{3}\!\cdot\!\left(\frac{15+5a}{3-a}\right)^{3}\!\cdot\!\left(\frac{a+3}{a^{2}+6a+9}\right)^{3}
		=\bigl(5\bigr)^{3}=125\)
		\item \(\displaystyle
		\left(\frac{x^{2}-10x+25}{x-4}\right)^{3}\!:\!\left(\frac{x-5}{x^{2}-16}\right)^{3}\!\cdot\!\left(\frac{4}{(x-5)(x+4)}\right)^{3}
		=4^{3}=64\)
	\end{enumerate}
	
	\vspace{0.5em}
	\footnotesize
	Hinweis: Die Aufgabenstellung stammt aus deinem Anhang (Abschnitte „Wissenschaftliche Schreibweise“, „Potenzen/Wurzeln“ etc.). :contentReference[oaicite:2]{index=2} :contentReference[oaicite:3]{index=3}
	
\end{document}
