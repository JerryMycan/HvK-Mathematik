\documentclass[a4paper,12pt]{article}
\usepackage{tabularx}
\usepackage{amsmath}
\usepackage[utf8]{inputenc}
\usepackage{amsmath, amssymb, amsthm}
\usepackage{graphicx}
\usepackage{array}
\usepackage[left=2cm, right=2cm, top=2cm, bottom=2cm]{geometry}
\usepackage{fancyhdr}
\usepackage{xfp} % Für mathematische Berechnungen in LaTeX
\usepackage{pgf} % Erlaubt Berechnungen mit \pgfmathparse

% Kopf- und Fußzeile
\pagestyle{fancy}
\lhead{Klassenarbeit 45min.}
\chead{Heinrich-von-Kleist-Schule}
\rhead{Mathematik - G8A}
\lfoot{}
\cfoot{Seite \thepage}
\rfoot{}

% Makro für einzelne Punkteingaben (hier einfach Platzhalter)
\newcommand{\punkteA}{14}  % Eingetragene Punkte für Aufgabe 1
\newcommand{\punkteB}{17}  % Eingetragene Punkte für Aufgabe 2
\newcommand{\punkteC}{9}   % Eingetragene Punkte für Aufgabe 3
\newcommand{\punkteD}{6}   % Eingetragene Punkte für Aufgabe 4
\newcommand{\punkteE}{15}  % Eingetragene Punkte für Aufgabe 5

% Maximale Punktzahl insgesamt
\newcommand{\maxSumme}{61}  % Hier die maximale Punktzahl anpassen

% Automatische Berechnung der Notengrenzen
\newcommand{\noteEinsMin}{\fpeval{round(\maxSumme * 0.90,0)}}
\newcommand{\noteZweiMin}{\fpeval{round(\maxSumme * 0.75,0)}}
\newcommand{\noteDreiMin}{\fpeval{round(\maxSumme * 0.60,0)}}
\newcommand{\noteVierMin}{\fpeval{round(\maxSumme * 0.45,0)}}
\newcommand{\noteFunfMin}{\fpeval{round(\maxSumme * 0.20,0)}}
\newcommand{\noteSechsMin}{0}

% Automatische Berechnung der Summe
\newcommand{\summe}{%
	\pgfmathparse{\punkteA + \punkteB + \punkteC + \punkteD + \punkteE}%
	\pgfmathprintnumber{\pgfmathresult}
}


\begin{document}
	
	\begin{center}
		%\textbf{Klassenarbeit N°1 G8}\\[0.3cm]
		\textbf{Themen der Klassenarbeit}\\[0.2cm]
		%\textbf{© J. Mycan}\\[0.5cm]
	\end{center}
	
	\textbf{Vor- und Nachname:} \underline{\hspace{10cm}}\\%[0.5cm]
	
	%\vspace{1cm}
	
	\textbf{Aufgabe 1} \hfill (4+6+4 = \punkteA{} Punkte)\\

	
	\textbf{Aufgabe 2} \hfill (5+7+5 = \punkteB{} Punkte)\\

	
	\textbf{Aufgabe 3} \hfill (3+3+3 = \punkteC{} Punkte)\\

	
	\textbf{Aufgabe 4} \hfill (\punkteD{} Punkte)\\

	
	\textbf{Aufgabe 5} \hfill (2+3+8+4 = \punkteE{} Punkte)\\

	
	
	\textbf{Auswertungstabelle:}
	\begin{center}
		\begin{tabular}{|c|c|c|c|c|c|c|c|}
			\hline
			Aufgabe & 1 & 2 & 3 & 4 & 5 & Summe & Note \\
			\hline
			Punkte  & \text{/ \punkteA } & \text{/ \punkteB } & \text{/ \punkteC } & \text{/ \punkteD } & \text{/ \punkteE}  & \summe & \\
			\hline
		\end{tabular}
	\end{center}	
	%	\vspace{1cm}
	
	\textbf{Notenschlüssel:}
	
	\begin{center}
		\begin{tabular}{|c|c|c|c|c|c|c|}
			\hline
			Note & 1 & 2 & 3 & 4 & 5 & 6 \\
			\hline
			Prozent \% & 100--90 & 89--75 & 74--60 & 59--45 & 44--20 & 19--0 \\
			\hline
			Punkte & \maxSumme{}--\noteEinsMin{} & \fpeval{\noteEinsMin-1}--\noteZweiMin{} & \fpeval{\noteZweiMin-1}--\noteDreiMin{} & \fpeval{\noteDreiMin-1}--\noteVierMin{} & \fpeval{\noteVierMin-1}--\noteFunfMin{} & \fpeval{\noteFunfMin-1}--\noteSechsMin{} \\
			\hline
		\end{tabular}
	\end{center}
	
	\vspace{3cm}
	
	\textbf{Kenntnisnahme eines Elternteils:} \hrulefill \hfill \textbf{Note:} \hrulefill
	
\end{document}