\documentclass[a4paper,12pt]{article}
\usepackage{tabularx}
\usepackage{amsmath}
\usepackage[utf8]{inputenc}
\usepackage{amsmath, amssymb, amsthm}
\usepackage{graphicx}
\usepackage{array}
\usepackage[left=2cm, right=2cm, top=2cm, bottom=2cm]{geometry}
\usepackage{fancyhdr}
\usepackage{xfp}
\usepackage{pgf}

\pagestyle{fancy}
\lhead{Probe-Klassenarbeit 45min.}
\chead{Heinrich-von-Kleist-Schule}
\rhead{Mathematik - G8A}
\lfoot{}
\cfoot{Seite \thepage}
\rfoot{}

\newcommand{\punkteA}{6}
\newcommand{\punkteB}{9}

\newcommand{\punkteC}{5}
\newcommand{\punkteD}{12}
\newcommand{\punkteE}{8}

\newcommand{\maxSumme}{40}
\newcommand{\noteEinsMin}{\fpeval{round(\maxSumme * 0.96,0)}}
\newcommand{\noteZweiMin}{\fpeval{round(\maxSumme * 0.80,0)}}
\newcommand{\noteDreiMin}{\fpeval{round(\maxSumme * 0.60,0)}}
\newcommand{\noteVierMin}{\fpeval{round(\maxSumme * 0.45,0)}}
\newcommand{\noteFunfMin}{\fpeval{round(\maxSumme * 0.20,0)}}
\newcommand{\noteSechsMin}{0}

\newcommand{\summe}{%
	\pgfmathparse{\punkteA + \punkteB + \punkteC + \punkteD + \punkteE}%
	\pgfmathprintnumber{\pgfmathresult}}

\begin{document}
	
	\begin{center}
		\textbf{Probe-Klassenarbeit - Lineare Funktionen und Gleichungen}
	\end{center}
	
	\textbf{Vor- und Nachname:} \underline{\hspace{10cm}}\\[0.2cm]
	
	Die Lösungen sowie Lösungswege sollten klar strukturiert und gut nachvollziehbar sein. Jeder einzelne Berechnungsschritt ist mit einer kurzen, prägnanten Überschrift zu versehen, die verdeutlicht, welcher Teil der Aufgabe bearbeitet wird. \\[0.2cm]
	
	\textbf{Aufgabe 1 (6 Punkte)}\newline
	Zeichne die folgenden Funktionen in ein Koordinatensystem:\newline
	a) \( f(x) = \frac{3}{2}x - 3 \)\newline
	b) \( g(x) = -\frac{1}{2}x + 2 \)\newline
	c) \( h(x) = 3 - x \)\newline
	
	\textbf{Aufgabe 2 (9 Punkte)}\newline
	a) Bestimme die Funktionsgleichung der linearen Funktion, die durch die Punkte \(A(1|3)\) und \(B(3|7)\) verläuft.\newline
	b) Berechne die Nullstelle dieser Funktion.\newline
	c) Prüfe rechnerisch, ob der Punkt \(C(2|5)\) auf der Geraden liegt.\newline
	
	\textbf{Aufgabe 3 (5 Punkte)}\newline
	Berechne den Schnittpunkt folgender linearer Funktionen:\newline
	\[ f(x) = 4x - 5 \quad \text{und} \quad g(x) = 7 - 2x \]
	

	\textbf{Aufgabe 4 (12 Punkte)}\newline
	a) Ermittle die Gleichung der Geraden mit Nullstelle bei \( x = 65 \) und Steigung \( m = -2,7 \).\newline
	b) Bestimme eine zur Geraden \( g(x) = 2x - 1 \) parallele Gerade, die bei \( x = 5 \) die Nullstelle hat.\newline
	c) Bestimme eine Gerade \(s(x)\), die orthogonal zu \( g(x)\) aus dem Teil b) liegt.
	
	\textbf{Aufgabe 5 (8 Punkte)}\newline
	Ein Swimmingpool enthält zu Beginn des Sommers 24.000 Liter Wasser. Durch ein kleines Leck verliert der Pool täglich gleichmäßig 150 Liter Wasser.
	
	a) Stelle die zugehörige lineare Funktionsgleichung auf, die das Sachverhalt beschreibt.
	
	b) Wie viel Wasser befindet sich nach 30 Tagen noch im Swimmingpool?
	
	c) Nach wie vielen Tagen ist der Pool vollständig leer?
	
	\vspace{1cm}
	\textbf{Auswertungstabelle:}
	\begin{center}
		\begin{tabular}{|c|c|c|c|c|c|c|c|}
			\hline
			Aufgabe & 1 & 2 & 3 & 4 & 5 & Summe\\
			\hline
			Punkte & \text{\ \ / \punkteA } & \text{\ \ / \punkteB } & \text{\ \ / \punkteC } & \text{\ \ / \punkteD } & \text{\ \ / \punkteE} & \text{\ \ / \summe}\\
			\hline
		\end{tabular}
	\end{center}
	
	\textbf{Notenschlüssel:}
	\begin{center}
		\begin{tabular}{|c|c|c|c|c|c|c|}
			\hline
			Note & 1 & 2 & 3 & 4 & 5 & 6 \\
			\hline
			Prozent \% & 100--96 & 95--80 & 79--60 & 59--45 & 44--16 & 15--0 \\
			\hline
			Punkte & \maxSumme{}--\noteEinsMin{} & \fpeval{\noteEinsMin-1}--\noteZweiMin{} & \fpeval{\noteZweiMin-1}--\noteDreiMin{} & \fpeval{\noteDreiMin-1}--\noteVierMin{} & \fpeval{\noteVierMin-1}--\noteFunfMin{} & \fpeval{\noteFunfMin-1}--\noteSechsMin{} \\
			\hline
		\end{tabular}
	\end{center}
	
	\vspace{2cm}
	\textbf{Kenntnisnahme eines Elternteils:} \hrulefill \hfill \textbf{Note:} \hrulefill
	
\end{document}

