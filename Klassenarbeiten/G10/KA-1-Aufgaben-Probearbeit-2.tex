\documentclass[a4paper,12pt]{article}
\usepackage{tabularx}
\usepackage{amsmath}
\usepackage[utf8]{inputenc}
\usepackage{multicol}
\usepackage{amsmath, amssymb, amsthm}
\usepackage{graphicx}
\usepackage{enumitem}
\usepackage{array}
\usepackage[left=2cm, right=2cm, top=2cm, bottom=2cm]{geometry}
\usepackage{fancyhdr}
\usepackage{xfp}
\usepackage{pgf}

\usepackage{graphicx}
\usepackage{fancyhdr}
\setlength{\headheight}{28pt} % genug Platz für das Logo
\pagestyle{fancy}
\fancyhf{} % alles leeren
\fancyhead[L]{\includegraphics[height=1.2cm]{logo.png}}
\fancyhead[C]{\small Klassenarbeit – Rechnen mit Potenzen \ (Kl. G10B)}
\fancyhead[R]{\small Name:\ \rule{2.8cm}{0.4pt}}
\fancyfoot[C]{\thepage}

\fancyfoot[C]{Seite \thepage \enspace\textbullet\enspace J.\,Mycan \textcopyright~2025}

\renewcommand{\footrulewidth}{0.4pt}




%\pagestyle{fancy}
%\lhead{Klassenarbeit 45min.}
%\chead{Heinrich-von-Kleist-Schule}
%\rhead{Mathematik - G8A}
%\lfoot{}
%\cfoot{Seite \thepage}
%\rfoot{}

\newcommand{\punkteA}{6}
\newcommand{\punkteB}{6}
\newcommand{\punkteC}{6}
\newcommand{\punkteD}{18}
\newcommand{\punkteE}{6}
%\newcommand{\punkteF}{12}

\newcommand{\maxSumme}{42}
\newcommand{\noteEinsMin}{\fpeval{round(\maxSumme * 0.95,0)}}
\newcommand{\noteZweiMin}{\fpeval{round(\maxSumme * 0.80,0)}}
\newcommand{\noteDreiMin}{\fpeval{round(\maxSumme * 0.60,0)}}
\newcommand{\noteVierMin}{\fpeval{round(\maxSumme * 0.45,0)}}
\newcommand{\noteFunfMin}{\fpeval{round(\maxSumme * 0.20,0)}}
\newcommand{\noteSechsMin}{0}

\newcommand{\summe}{%
	\pgfmathparse{\punkteA + \punkteB + \punkteC + \punkteD + \punkteE}%
	\pgfmathprintnumber{\pgfmathresult}}

\begin{document}
	
%	\begin{center}
%		\textbf{Klassenarbeit - Lineare Funktionen und LGS}
%	\end{center}
	
%	\textbf{Vor- und Nachname:} \underline{\hspace{10cm}}\\[0.1cm]
	Die Lösungen sowie Lösungswege sollten klar strukturiert und gut nachvollziehbar sein.\\[0.1cm]
	
	% -------------------------------------------------
	% Aufgabe 1
	% -------------------------------------------------
	\textbf{Aufgabe 1 (6 Punkte)}\\
	Wandle \emph{ohne Rundung} in die Form \(a\cdot 10^{n}\) mit \(1\le a<10\) um.
	\begin{multicols}{2}
		\begin{enumerate}[label=\alph*)]
			\item \(0{,}00034\)
			%\item \(456\,000\)
			%\item \(12\,300\,000\,000\)
			\item \(0{,}045\)
			\item \(7\,020\,000\)
			\item \(0{,}00000092\)
			\item \(68\,000\)
			\item \(101{,}9\)
		\end{enumerate}
	\end{multicols}
	
	
	\vspace{1.5cm}
	
	% -------------------------------------------------
	% Aufgabe 2
	% -------------------------------------------------
	\textbf{Aufgabe 2 (6 Punkte)}\\
	Schreibe in normale Dezimalschreibweise um (\emph{ohne} Rundung).
	\begin{multicols}{2}
		\begin{enumerate}[label=\alph*)]
			%\item \(2{,}5\cdot 10^{3}\)
			\item \(7{,}1\cdot 10^{-3}\)
			\item \(3{,}02\cdot 10^{5}\)
			\item \(8\cdot 10^{0}\)
			\item \(4{,}09\cdot 10^{-2}\)
			%\item \(6{,}4\cdot 10^{-6}\)
			\item \(1{,}234\cdot 10^{2}\)
			\item \(5\cdot 10^{-1}\)
		\end{enumerate}
	\end{multicols}
	
	
	\vspace{1cm}
	% -------------------------------------------------
	% Aufgabe 3
	% -------------------------------------------------
	\textbf{Aufgabe 3 (6 Punkte)}\\
	Vereinfache und berechne vollständig.
	\begin{multicols}{2}
		\begin{enumerate}[label=\alph*)]
			\item \(2^{5}\cdot 2^{-2}\)
			\item \(3^{2} : 3^{-2}\)
			\item \(\bigl(5^{2}\bigr)^{3}\)
			\item \(10^{-3}\cdot 10^{5}\)
			\item \(4^{-1}\)
			\item \(9^{\tfrac{3}{2}}\)
		\end{enumerate}
	\end{multicols}
	
	\vspace{1 cm}
	%\newpage
	% -------------------------------------------------
	% Aufgabe 4
	% -------------------------------------------------
	\textbf{Aufgabe 4 (18 Punkte)}\\
	Vereinfache vollständig; schreibe ohne negative Exponenten und rationalisiere ggf. den Nenner. Es gelte \(a,b,x,y>0\).
	\begin{multicols}{2}
		\begin{enumerate}[label=\alph*)]
			\item \(\sqrt{18a^{2}b}\)
			\item \(\dfrac{x^{\tfrac32}\cdot \sqrt{8x}}{\sqrt{2}}\)
			\item \(\dfrac{\sqrt{45\,x^{5}}}{x^{\tfrac32}\sqrt{5x}}\)
			%\item \(\left(\dfrac{\sqrt{a}}{\sqrt[4]{a^{3}}}\right)^{2}\)
			\item \(\dfrac{2}{\sqrt{x}}+\dfrac{\sqrt{x}}{2}\)
			\item \(\left(\dfrac{\sqrt{a}-\sqrt{b}}{\sqrt{a}+\sqrt{b}}\right)^{2}\)
			\item \( \dfrac{a^{2n+1}}{b^{-3n}}\cdot \dfrac{b^{-3n-5}}{a^{2n+6}} \)
			%\item \( \dfrac{y^{3n-5}}{x^{n+3}}\cdot \dfrac{x^{2n+5}}{y^{2n-4}} \)
		\end{enumerate}
	\end{multicols}
	
	\newpage
	% -------------------------------------------------
	% Aufgabe 4
	% -------------------------------------------------
%	\textbf{Aufgabe 5 (15 Punkte)}\\
%	Vereinfache vollständig; schreibe ohne negative Exponenten und rationalisiere ggf. den Nenner. Es gelte \(a,b,x,y>0\).
%	\begin{multicols}{2}
%		\begin{enumerate}[label=\alph*)]
%			\item \( \dfrac{a^{2n+1}}{b^{-3n}}\cdot \dfrac{b^{-3n-5}}{a^{2n+6}} \)
%			\item \( \dfrac{y^{3n-5}}{x^{n+3}}\cdot \dfrac{x^{2n+5}}{y^{2n-4}} \)
%		\end{enumerate}
%	\end{multicols}
	
	
	%\vspace{1.5cm}
	% -------------------------------------------------
	% Aufgabe 5
	% -------------------------------------------------
	\textbf{Aufgabe 5 (12 Punkte)}\\
	Vereinfache vollständig (Definitionsmenge beachten).
	\begin{enumerate}[label=\alph*)]
%		\item \[
%		\left(\frac{9-a^{2}}{a+3}\right)^{2}
%		\cdot
%		\left(\frac{15+5a}{\,3-a\,}\right)^{2}
%		\cdot
%		\left(\frac{3-a}{\,5(3+a)\,}\right)^{2}
%		\]
		% Ergebnis: 1, da 9-a^2=(3-a)(3+a); alle Faktoren kürzen sich.
		% DGM: a \neq \pm 3
		
		\item \[
		\left(\frac{16-a^{2}}{a+4}\right)^{2}
		\cdot
		\left(\frac{8+2a}{\,4-a\,}\right)^{2}
		\cdot
		\left(\frac{4-a}{\,2(4+a)\,}\right)^{2}
		\]
		% Ergebnis: 1, da 16-a^2=(4-a)(4+a); alle Faktoren kürzen sich.
		% DGM: a \neq \pm 4
		
				\item \[
		\left(\frac{a^{2}-6a+9}{a-3}\right)^{2}
		\cdot
		\left(\frac{a-3}{a^{2}-9}\right)^{2}
		\cdot
		\left(\frac{a+3}{a-3}\right)^{2}
		\]
		% Ergebnis: 1, da a^2-6a+9=(a-3)^2 und a^2-9=(a-3)(a+3).
		% DGM: a \neq \pm 3
		
	\end{enumerate}



%		\item \[
%		\left(\frac{a^{2}-8a+16}{a-4}\right)^{2}
%		\cdot
%		\left(\frac{2a+2}{a^{2}-1}\right)^{2}
%		\cdot
%		\left(\frac{a-1}{2\,(a-4)}\right)^{2}
%		\]
		% Ergebnis: 1, da a^2-8a+16=(a-4)^2 und a^2-1=(a-1)(a+1), (2a+2)=2(a+1).
		% DGM: a \neq 4,\, \pm 1
	%\vspace{5cm}
	
	
	
	\vspace{6cm}
	\textbf{Auswertungstabelle:}
	\begin{center}
		\begin{tabular}{|c|c|c|c|c|c|c|c|}
			\hline
			Aufgabe & 1 & 2 & 3 & 4 & 5 &  Summe\\
			\hline
			Punkte & \text{\ / \punkteA} & \text{\ / \punkteB} & \text{\ / \punkteC} & \text{\ / \punkteD} & \text{\ / \punkteE} & \text{\ / \summe}\\
			\hline
		\end{tabular}
	\end{center}
	
	\textbf{Notenschlüssel:}
	\begin{center}
		\begin{tabular}{|c|c|c|c|c|c|c|}
			\hline
			Note & 1 & 2 & 3 & 4 & 5 & 6 \\
			\hline
			Prozent \% & 100--95 & 94--80 & 79--60 & 59--45 & 44--16 & 15--0 \\
			\hline
			Punkte & \maxSumme{}--\noteEinsMin{} & \fpeval{\noteEinsMin-1}--\noteZweiMin{} & \fpeval{\noteZweiMin-1}--\noteDreiMin{} & \fpeval{\noteDreiMin-1}--\noteVierMin{} & \fpeval{\noteVierMin-1}--\noteFunfMin{} & \fpeval{\noteFunfMin-1}--\noteSechsMin{} \\
			\hline
		\end{tabular}
	\end{center}
	
	\vspace{2cm}
	\textbf{Kenntnisnahme eines Elternteils:} \hrulefill \hfill \textbf{Note:} \hrulefill
	
\end{document}
