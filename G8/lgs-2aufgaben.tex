\documentclass[a4paper,12pt]{article}
\usepackage[utf8]{inputenc}
\usepackage[ngerman]{babel}
\usepackage{amsmath}
\usepackage{geometry}
\geometry{left=2.5cm, right=2.5cm, top=2.5cm, bottom=2.5cm}
\usepackage{parskip}
\usepackage{graphicx}

\title{Arbeitsblatt: Lineare Gleichungssysteme}
\author{Heinrich-von-Kleist-Schule \\ Mathematik – Jahrgang 8}
\date{}

\begin{document}

\maketitle

\section*{Anwendungsaufgaben mit linearen Gleichungssystemen}

Löse die folgenden Textaufgaben mithilfe eines linearen Gleichungssystems. Gib die Gleichungen an und löse sie rechnerisch.

\begin{enumerate}
    \item Aus 80\%-igem und 30\%-igem Alkohol sollen durch Mischung 40 Liter hergestellt werden, die 50\% Alkohol enthalten. Wie viel Liter jeder Sorte werden benötigt?

    \item Aus zwei Apfelsäften mit einem Fruchtanteil von 70\% und 45\% sollen 80 Liter Apfelsaft mit einem Fruchtanteil von 60\% hergestellt werden. Wie viel Liter jeder Sorte müssen bereitgestellt werden?

    \item Wie viel Liter Traubensaft mit einem Fruchtanteil von 75\% muss man zu 20 Liter Traubensaft mit einem Fruchtanteil von 90\% gießen, um einen Traubensaft mit 80\% Fruchtanteil zu erhalten?

    \item Es sollen 80 g Gold vom Feingehalt 750/1000 (d.\,h.\ der Anteil an reinem Gold beträgt 750/1000) hergestellt werden. Es stehen Gold vom Feingehalt 800 und vom Feingehalt 650 zur Verfügung. Wie viel Gramm jeder Sorte werden benötigt?

    \item Zu 63 kg Messing mit 42\% Kupfer soll Messing mit 70\% Kupfer zugeschmolzen werden, um Messing mit 52\% Kupfer zu erhalten. Wie viel Messing von 70\% Kupfer ist notwendig?

    \item Ein Großhändler will eine gute Teesorte, das Kilogramm zu 40\ €, mit einer billigen Teesorte, das Kilogramm zu 34\ €, mischen. Es soll nach seinem Rezept 2{,}5-mal so viel von der billigeren Sorte verwendet werden wie von der teureren Sorte. Der Händler will für insgesamt 1000\ € von der Mischung herstellen. Wie viel Kilogramm jeder Sorte werden benötigt?

    \item Aus zwei Orangensäften mit einem Fruchtanteil von 55\% und 3\% soll ein halber Liter Saft mit einem Fruchtanteil von 23\% gemischt werden. Wie viel Liter jeder Sorte werden benötigt?
\end{enumerate}

\vspace{1cm}
\hrule
\vspace{0.5cm}
\textbf{Hinweis:} Überlege dir, welche Größen du als Variablen benennen möchtest. Stelle ein Gleichungssystem auf und löse es z.\,B. durch das Additions-, Einsetzungs- oder Gleichsetzungsverfahren.

\end{document}
