\documentclass[a4paper,12pt]{article}
%\documentclass[a4paper,fontsize=13pt]{scrartcl}

\usepackage{tabularx}
\usepackage{amsmath}
\usepackage[utf8]{inputenc}
\usepackage{multicol}
\usepackage{amsmath, amssymb, amsthm}
\usepackage{graphicx}
\usepackage{enumitem}
\usepackage{array}
\usepackage[left=2cm, right=2cm, top=2cm, bottom=2cm]{geometry}
\usepackage{fancyhdr}
\usepackage{xfp}
\usepackage{pgf}
\usepackage[T1]{fontenc}
\usepackage[utf8]{inputenc} % hast du schon, ist ok


\usepackage{graphicx}
\usepackage{fancyhdr}
\setlength{\headheight}{28pt} % genug Platz für das Logo
\pagestyle{fancy}
\fancyhf{} % alles leeren
\fancyhead[L]{\includegraphics[height=1.2cm]{logo.png}}
\fancyhead[C]{\small Klassenarbeit – Rechnen mit Potenzen \ (Kl. G10B)}
\fancyhead[R]{\small Name:\ \rule{2.8cm}{0.4pt}}
\fancyfoot[C]{\thepage}

\fancyfoot[C]{Seite \thepage \enspace\textbullet\enspace J.\,Mycan \textcopyright~2025 *Klassenarbeit 45 min.*}

\renewcommand{\footrulewidth}{0.4pt}



\begin{document}
\textbf{Arbeitsauftrag: Potenzfunktionen}\\[0.4em]
Erstelle selbstständig ein kleines Aufgabenset zum Thema \emph{Potenzfunktionen}.  
Dein Aufgabenset soll aus \textbf{drei} Aufgaben bestehen, die folgende Bereiche abdecken:

\begin{enumerate}
	\item \textbf{Aufgabe 1 – Funktionsgraphen zuordnen}\\
	Entwickle eine Zuordnungsaufgabe:
	\begin{itemize}
		\item Zeichne in ein Koordinatensystem \textbf{mindestens vier} verschiedene Graphen von Potenzfunktionen
		(z.B. \(x^2, x^3, \tfrac{1}{x}, \sqrt{x}, \tfrac{1}{\sqrt{x}}, x^{1/3}\)\, …).
		\item Stelle daneben \textbf{mindestens vier} Funktionsgleichungen auf, von denen genau vier zu den gezeichneten
		Graphen passen (eine Funktionsgleichung bleibt „übrig“).
		\item Formuliere als Arbeitsauftrag: „Ordne jedem Graphen eine Funktionsgleichung zu. Begründe kurz.“
		\item Erstelle eine Musterlösung mit richtiger Zuordnung.
	\end{itemize}
	
	\item \textbf{Aufgabe 2 – Symmetrie von Funktionen (4-Teil-Aufgabe)}\\
	Entwickle eine Aufgabe mit \textbf{vier Teilaufgaben (a–d)}, in der die Symmetrie von Funktionen untersucht wird.
	\begin{itemize}
		\item Wähle vier verschiedene Funktionen (z.B. Potenzfunktionen wie \(x^2, x^3, \tfrac{1}{x^2}, -x^3\) oder
		Mischungen mit Vorfaktoren).
		\item Formuliere zu jeder Funktion eine Teilaufgabe, z.B.:
		\begin{itemize}
			\item „Untersuche, ob die Funktion achsensymmetrisch, punktsymmetrisch oder ohne Symmetrie ist.“
			\item „Begründe deine Entscheidung rechnerisch (mit \(f(-x)\) und ggf. \(-f(x)\)).“
		\end{itemize}
		\item Schreibe eine vollständige Lösung zu jeder Teilaufgabe mit Symmetrieart und kurzer Begründung.
	\end{itemize}
	
	\item \textbf{Aufgabe 3 – Potenzgleichungen (4-Teil-Aufgabe)}\\
	Entwickle eine Aufgabe mit \textbf{vier Potenzgleichungen (a–d)}, die gelöst werden sollen.
	\begin{itemize}
		\item Nutze verschiedene Basen (z.B. \(2, 3, 5, \tfrac{1}{2}, \tfrac{1}{3}\)) und Exponenten mit \(x\)
		(z.B. \(2^{x+1}, 3^{2x-1}, \bigl(\tfrac{1}{2}\bigr)^{x-3}\) …).
		\item Mindestens zwei Gleichungen sollen Terme mit Potenzen auf \emph{beiden} Seiten haben.
		\item Formuliere als Arbeitsauftrag: „Löse die Gleichung. Vereinfache sinnvoll und gib alle Lösungen an.“
		\item Erstelle zu jeder Teilaufgabe eine Musterlösung mit klaren Zwischenschritten
		(z.B. Umformen auf gleiche Basis, Exponenten gleichsetzen, Äquivalenzumformungen).
	\end{itemize}
\end{enumerate}

\medskip
\textbf{Abgabe:}\\
Gib dein Aufgabenset sauber geschrieben oder gedruckt ab.  
Zu jeder von dir erstellten Aufgabe muss eine vollständige \emph{Musterlösung} mit Rechenwegen vorhanden sein, sodass
eine Mitschülerin / ein Mitschüler deine Aufgaben problemlos lösen und kontrollieren kann.

	
\end{document}
