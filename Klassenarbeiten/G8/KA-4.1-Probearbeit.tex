\documentclass[a4paper,12pt]{article}
\usepackage{tabularx}
\usepackage{amsmath}
\usepackage[utf8]{inputenc}
\usepackage{amsmath, amssymb, amsthm}
\usepackage{graphicx}
\usepackage{array}
\usepackage[left=2cm, right=2cm, top=2cm, bottom=2cm]{geometry}
\usepackage{fancyhdr}
\usepackage{xfp}
\usepackage{pgf}

\pagestyle{fancy}
\lhead{Probe-Klassenarbeit 45min.}
\chead{Heinrich-von-Kleist-Schule}
\rhead{Mathematik - G8A}
\lfoot{}
\cfoot{Seite \thepage}
\rfoot{}

\newcommand{\punkteA}{6}
\newcommand{\punkteB}{9}

\newcommand{\punkteC}{5}
\newcommand{\punkteD}{12}
\newcommand{\punkteE}{8}

\newcommand{\maxSumme}{40}
\newcommand{\noteEinsMin}{\fpeval{round(\maxSumme * 0.96,0)}}
\newcommand{\noteZweiMin}{\fpeval{round(\maxSumme * 0.80,0)}}
\newcommand{\noteDreiMin}{\fpeval{round(\maxSumme * 0.60,0)}}
\newcommand{\noteVierMin}{\fpeval{round(\maxSumme * 0.45,0)}}
\newcommand{\noteFunfMin}{\fpeval{round(\maxSumme * 0.20,0)}}
\newcommand{\noteSechsMin}{0}

\newcommand{\summe}{%
	\pgfmathparse{\punkteA + \punkteB + \punkteC + \punkteD + \punkteE}%
	\pgfmathprintnumber{\pgfmathresult}}

\begin{document}
	
	\begin{center}
		\textbf{Probe-Klassenarbeit - Lineare Funktionen und Gleichungen}
	\end{center}
	
	\textbf{Vor- und Nachname:} \underline{\hspace{10cm}}\\[0.1cm]
	Die Lösungen sowie Lösungswege sollten klar strukturiert und gut nachvollziehbar sein. Jeder einzelne Berechnungsschritt ist mit einer kurzen, prägnanten Überschrift zu versehen, die verdeutlicht, welcher Teil der Aufgabe bearbeitet wird. \\[0.1cm]
	
\textbf{Aufgabe 1 (6 Punkte)}\newline
Zeichne die Graphen folgender Funktionen in ein gemeinsames Koordinatensystem:\newline
	a) $f(x) = \frac{1}{2}x - 4$\newline
	b) $g(x) = -1,5x + 3$\newline
	c) $h(x) = - 1 - x$\newline

\textbf{Aufgabe 2 (9 Punkte)}\newline
Gegeben sind die Punkte $A(2|1)$ und $B(4|5).$\newline
	a) Ermittle die Funktionsgleichung der Geraden durch diese Punkte.\newline
	b) Berechne die Nullstelle der Funktion.\newline
	c) Prüfe rechnerisch, ob der Punkt $C(3|3)$ auf dieser Geraden liegt.\newline

\textbf{Aufgabe 3 (5 Punkte)}\newline
Berechne den Schnittpunkt der Geraden:
\[f(x) = -3x + 6 \quad\text{und}\quad g(x) = 2x - 4\]

\textbf{Aufgabe 4 (12 Punkte)}\newline
	a) Ermittle die Gleichung einer Geraden mit Nullstelle bei $x=20$ und der Steigung $m=-1,5$.\newline
	b) Ermittle eine zur Geraden $f(x)=3x-2$ parallele Gerade mit Nullstelle bei $x=-3$.\newline
	c) Ermittle eine Gerade, die orthogonal zur Geraden aus b) durch den Punkt $(0|0)$ verläuft.\newline

\textbf{Aufgabe 5 (8 Punkte)}\newline
Ein Wassertank enthält 10.000 Liter Wasser. Aufgrund eines technischen Problems verliert der Tank jeden Tag 200 Liter Wasser.\newline
	a) Erstelle eine Funktionsgleichung, die diese Situation beschreibt.\newline
	b) Wie viel Wasser befindet sich nach 15 Tagen noch im Tank?\newline
	c) Nach wie vielen Tagen ist der Tank leer?\newline

	
	\vspace{1cm}
	\textbf{Auswertungstabelle:}
	\begin{center}
		\begin{tabular}{|c|c|c|c|c|c|c|c|}
			\hline
			Aufgabe & 1 & 2 & 3 & 4 & 5 & Summe\\
			\hline
			Punkte & \text{\ \ / \punkteA } & \text{\ \ / \punkteB } & \text{\ \ / \punkteC } & \text{\ \ / \punkteD } & \text{\ \ / \punkteE} & \text{\ \ / \summe}\\
			\hline
		\end{tabular}
	\end{center}
	
	\textbf{Notenschlüssel:}
	\begin{center}
		\begin{tabular}{|c|c|c|c|c|c|c|}
			\hline
			Note & 1 & 2 & 3 & 4 & 5 & 6 \\
			\hline
			Prozent \% & 100--96 & 95--80 & 79--60 & 59--45 & 44--16 & 15--0 \\
			\hline
			Punkte & \maxSumme{}--\noteEinsMin{} & \fpeval{\noteEinsMin-1}--\noteZweiMin{} & \fpeval{\noteZweiMin-1}--\noteDreiMin{} & \fpeval{\noteDreiMin-1}--\noteVierMin{} & \fpeval{\noteVierMin-1}--\noteFunfMin{} & \fpeval{\noteFunfMin-1}--\noteSechsMin{} \\
			\hline
		\end{tabular}
	\end{center}
	
	\vspace{2cm}
	\textbf{Kenntnisnahme eines Elternteils:} \hrulefill \hfill \textbf{Note:} \hrulefill
	
\end{document}

