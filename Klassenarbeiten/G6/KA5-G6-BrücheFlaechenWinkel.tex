\documentclass[a4paper,12pt]{article}
\usepackage{tabularx}
\usepackage{amsmath}
\usepackage[utf8]{inputenc}
\usepackage{amsmath, amssymb, amsthm}
\usepackage{graphicx}
\usepackage{array}
\usepackage[left=2cm, right=2cm, top=2cm, bottom=2cm]{geometry}
\usepackage{fancyhdr}
\usepackage{xfp} % Für mathematische Berechnungen in LaTeX
\usepackage{pgf} % Erlaubt Berechnungen mit \pgfmathparse

% Kopf- und Fußzeile
\pagestyle{fancy}
\lhead{10. Juni 2025\\Klassenarbeit 45min.}
\chead{Heinrich-von-Kleist-Schule}
\rhead{Mathematik - G6E}
\lfoot{}
%\cfoot{Seite \thepage}
\rfoot{}

% Makro für einzelne Punkteingaben (hier einfach Platzhalter)
\newcommand{\punkteA}{10}  % Eingetragene Punkte für Aufgabe 1
\newcommand{\punkteB}{6}  % Eingetragene Punkte für Aufgabe 2
\newcommand{\punkteC}{7}   % Eingetragene Punkte für Aufgabe 3
\newcommand{\punkteD}{5}   % Eingetragene Punkte für Aufgabe 4
\newcommand{\punkteE}{7}  % Eingetragene Punkte für Aufgabe 5
\newcommand{\punkteF}{10}  % Eingetragene Punkte für Aufgabe 6

% Maximale Punktzahl insgesamt
\newcommand{\maxSumme}{45}  % Hier die maximale Punktzahl anpassen

% Automatische Berechnung der Notengrenzen
\newcommand{\noteEinsMin}{\fpeval{round(\maxSumme * 0.92,0)}}
\newcommand{\noteZweiMin}{\fpeval{round(\maxSumme * 0.77,0)}}
\newcommand{\noteDreiMin}{\fpeval{round(\maxSumme * 0.62,0)}}
\newcommand{\noteVierMin}{\fpeval{round(\maxSumme * 0.46,0)}}
\newcommand{\noteFunfMin}{\fpeval{round(\maxSumme * 0.20,0)}}
\newcommand{\noteSechsMin}{0}

% Automatische Berechnung der Summe
\newcommand{\summe}{%
	\pgfmathparse{\punkteA + \punkteB + \punkteC + \punkteD + \punkteE + \punkteF}%
	\pgfmathprintnumber{\pgfmathresult}
}


\begin{document}
	
	\begin{center}
		%\textbf{Klassenarbeit N°1 G8}\\[0.3cm]
		\textbf{Rechnen mit Brüchen}\\[0.2cm]
		%\textbf{© J. Mycan}\\[0.5cm]
	\end{center}
	
	\textbf{Vor- und Nachname:} \underline{\hspace{10cm}}\\%[0.5cm]
	
	%\vspace{1cm}

\textbf{Aufgabe 1}\hfill ( \punkteA{} Punkte)

Berechne die folgenden Bruchterme. Achte auf die Klammern und kürze, wenn möglich.  
Zeige alle Rechenschritte übersichtlich!

\begin{center}
	\renewcommand{\arraystretch}{2}
	\begin{tabular}{ll}
		a) $\left( \frac{3}{4} + \frac{1}{2} \right) \cdot \frac{2}{5} =$ & 
		b) $\frac{7}{8} : \left( \frac{3}{4} - \frac{1}{8} \right) =$ \\
		
		c) $\left( \frac{5}{6} - \frac{2}{3} \right) \cdot \left( \frac{3}{4} + \frac{1}{2} \right) =$ &
		d) $\frac{2}{3} + \left[ \frac{4}{9} : \frac{2}{3} \right] =$ \\
	\end{tabular}
\end{center}


\textbf{Aufgabe 2} \hfill ( \punkteB{} Punkte)\\
Ein Kino möchte seine Sitzplätze erneuern. 
Geplant ist, \(\frac{3}{5}\) aller Sitzplätze mit neuen Polstern auszustatten.  
Davon sollen \(\frac{3}{10}\) rote Bezüge erhalten, \(\frac{1}{4}\) blaue.

Wie groß ist der Anteil der Sitze mit roten Bezügen an allen Sitzplätzen?  
Und wie groß ist der Anteil der Sitze mit blauen Bezügen an allen Sitzplätzen?  
Gib beide Anteile vollständig gekürzt als Bruch an.\\

\textbf{Aufgabe 3} \hfill ( \punkteC{} Punkte)\\
Ein Brotlaib wiegt \( \frac{7}{8} \) kg.
Davon werden \( \frac{1}{2} \) kg an die Tafel gespendet.
Der Rest wird in 4 gleich große Stücke geteilt.
Wie viel wiegt eines dieser Stücke? Gib dein Ergebnis als Bruch in kg an.\\


\textbf{Aufgabe 4} \hfill ( \punkteD{} Punkte)\\
Ein Wasserbecken im Botanischen Garten hat die Maße: 25 m lang, 12 m breit und 1{,}8 m tief.
a) Berechne das Volumen des Beckens in Kubikmetern.\\[0.2cm]
b) Wie viele Liter Wasser passen in das Becken?\\


\textbf{Aufgabe 5} \hfill ( \punkteE{} Punkte)\\
Gib jeweils die fehlenden Winkel an.

\begin{center}
		\includegraphics[width=0.50\textwidth]{winkel1.png}\\
	Berechne die Winkel \(\alpha\), \(\beta\), \(\gamma\), \(\epsilon\), \(\delta\), \(\phi\) und \(\eta\). Begründe jeweils mit einer Winkelbeziehung.
\end{center}




\textbf{Aufgabe 6} \hfill (\punkteF{} Punkte)\\
Berechne die Flächeninhalte der folgenden Figuren. Zerlege die Flächen in bekannte geometrische Formen (z.\,B.\ Rechtecke, Dreiecke, Trapeze). Notiere alle Zwischenschritte und gib die Lösung jeweils in  an.\\

\begin{minipage}{0.38\textwidth}
	\includegraphics[width=0.9\textwidth]{flaeche1.png}\\
\end{minipage}
\hfill
\begin{minipage}{0.48\textwidth}
	\textbf{b)}\\
	\includegraphics[width=0.9\textwidth]{flaeche2.png}\\
\end{minipage}




	\vspace{0.5cm}
	\textbf{Auswertungstabelle:}
	\begin{center}
		\begin{tabular}{|c|c|c|c|c|c|c|c|c|}
			\hline
			Aufgabe & 1 & 2 & 3 & 4 & 5 & 6 & Summe & Note \\
			\hline
			Punkte  & \text{\quad / \punkteA } & \text{\quad / \punkteB } & \text{ \quad / \punkteC } & \text{\quad / \punkteD } & \text{\quad / \punkteE} & \text{\quad / \punkteF} & \summe & \\
			\hline
		\end{tabular}
	\end{center}	
	%	\vspace{1cm}
	
	\textbf{Notenschlüssel:}
	
	\begin{center}
		\begin{tabular}{|c|c|c|c|c|c|c|}
			\hline
			Note & 1 & 2 & 3 & 4 & 5 & 6 \\
			\hline
			Prozent \% & 100--92 & 91--77 & 76--62 & 61--46 & 45--20 & 19--0 \\
			\hline
			Punkte & \maxSumme{}--\noteEinsMin{} & \fpeval{\noteEinsMin-1}--\noteZweiMin{} & \fpeval{\noteZweiMin-1}--\noteDreiMin{} & \fpeval{\noteDreiMin-1}--\noteVierMin{} & \fpeval{\noteVierMin-1}--\noteFunfMin{} & \fpeval{\noteFunfMin-1}--\noteSechsMin{} \\
			\hline
		\end{tabular}
	\end{center}
	
	\vspace{3cm}
	
	\textbf{Kenntnisnahme eines Elternteils:} \hrulefill \hfill \textbf{Note:} \hrulefill
	
\end{document}