\documentclass[a4paper,12pt]{article}
\usepackage[utf8]{inputenc}
\usepackage[ngerman]{babel}
\usepackage{amsmath}
\usepackage{geometry}
\geometry{a4paper,left=2.5cm,right=2.5cm,top=2.5cm,bottom=3cm}

\begin{document}
	
	\begin{center}
		\textbf{Heinrich-von-Kleist-Schule Eschborn}\\
		\textbf{Arbeitsblatt Mathematik, Klasse G6E}\\
		\textbf{Thema: Große Zahlen und Maßeinheiten}\\
		\vspace{0.5cm}
	\end{center}
	
	\textbf{Aufgabe 1 – Staatsverschuldung Deutschlands}\\
	
	In der politischen Entwicklung der letzten Tage ist sehr viel passiert. Deutschland möchte Kredite aufnehmen, um verschiedene Reformen anzustoßen. Die gesamte geplante Neuverschuldung beträgt 1,5 Billionen Euro (in Zahlen: 1.500.000.000.000 €).\\
	
	\textbf{a)} Stell dir vor, du könntest jede Sekunde genau eine 1€-Münze auf einen Stapel legen. Wie lange würdest du brauchen, bis alle 1,5 Billionen Münzen gestapelt sind? Gib die Zeitdauer in Jahren an.\\
	
	\textbf{b)} Eine 1€-Münze hat eine Dicke von 1 mm. Wie hoch wäre der Stapel aus 1,5 Billionen 1€-Münzen? Gib die Höhe in Kilometern (km) an.\\
	
	\textbf{c)} Ein 50€-Schein hat eine Dicke von etwa 0,1 mm. Wenn der gesamte Betrag von 1,5 Billionen Euro in Form von 50€-Scheinen gestapelt würde, wie hoch wäre dann der Stapel? Gib die Höhe ebenfalls in Kilometern (km) an.\\
	
	\textbf{d)} Der Umfang unserer Erde entlang des Äquators beträgt ca. 40.000 km. Wie oft könnte man den Stapel aus 1€-Münzen bzw. den Stapel aus 50€-Scheinen um die Erde legen?\\
	
	\vspace{1cm}
	\textit{Viel Erfolg beim Rechnen!}
	
\end{document}
