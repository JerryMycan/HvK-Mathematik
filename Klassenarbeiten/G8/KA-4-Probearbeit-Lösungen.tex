\documentclass[a4paper,12pt]{article}
\usepackage{tabularx}
\usepackage{amsmath}
\usepackage[utf8]{inputenc}
\usepackage{amsmath, amssymb, amsthm}
\usepackage{graphicx}
\usepackage{array}
\usepackage[left=2cm, right=2cm, top=2cm, bottom=2cm]{geometry}
\usepackage{fancyhdr}
\usepackage{xfp}
\usepackage{pgf}

\pagestyle{fancy}
\lhead{Klassenarbeit 45min.}
\chead{Heinrich-von-Kleist-Schule}
\rhead{Mathematik - G8A}
\lfoot{}
\cfoot{Seite \thepage}
\rfoot{}

\newcommand{\punkteA}{6}
\newcommand{\punkteB}{11}
\newcommand{\punkteC}{5}
\newcommand{\punkteD}{4}
\newcommand{\punkteE}{15}

\newcommand{\maxSumme}{61}
\newcommand{\noteEinsMin}{\fpeval{round(\maxSumme * 0.90,0)}}
\newcommand{\noteZweiMin}{\fpeval{round(\maxSumme * 0.75,0)}}
\newcommand{\noteDreiMin}{\fpeval{round(\maxSumme * 0.60,0)}}
\newcommand{\noteVierMin}{\fpeval{round(\maxSumme * 0.45,0)}}
\newcommand{\noteFunfMin}{\fpeval{round(\maxSumme * 0.20,0)}}
\newcommand{\noteSechsMin}{0}

\newcommand{\summe}{%
	\pgfmathparse{\punkteA + \punkteB + \punkteC + \punkteD + \punkteE}%
	\pgfmathprintnumber{\pgfmathresult}}

\begin{document}
	
	\begin{center}
		\textbf{Klassenarbeit - Lineare Funktionen und Gleichungen}
	\end{center}
	
	\textbf{Vor- und Nachname:} \underline{\hspace{10cm}}\\[0.5cm]
	
	\textbf{Aufgabe 1 (6 Punkte)}\newline
	Zeichne die folgenden Funktionen in ein Koordinatensystem:\newline
	a) \( f(x) = 2x - 3 \)\newline
	b) \( g(x) = -\frac{1}{2}x + 2 \)\newline
	c) \( h(x) = 3 - x \)\newline
	
	\textbf{Aufgabe 2 (17 Punkte)}\newline
	a) Bestimme die Funktionsgleichung der linearen Funktion, die durch die Punkte \(A(1|3)\) und \(B(3|7)\) verläuft.\newline
	b) Berechne die Nullstelle dieser Funktion.\newline
	c) Prüfe rechnerisch, ob der Punkt \(C(2|5)\) auf der Geraden liegt.\newline
	
	\textbf{Aufgabe 3 (9 Punkte)}\newline
	Berechne den Schnittpunkt folgender linearer Funktionen:\newline
	\[ f(x) = 4x - 5 \quad \text{und} \quad g(x) = 7 - 2x \]
	
	\textbf{Aufgabe 4 (6 Punkte)}\newline
	Begründe, ob der Punkt \(P(3|10)\) auf, über oder unter der Geraden \( f(x) = 3x - 5 \) liegt.
	
	\textbf{Aufgabe 5 (15 Punkte)}\newline
	a) Ermittle die Gleichung der Geraden mit Nullstelle bei \( x = 65 \) und Steigung \( m = -2,7 \).\newline
	b) Bestimme eine zur Geraden \( g(x) = 2x - 1 \) parallele Gerade, die bei \( x = 5 \) die Nullstelle hat.\newline
	c) Bestimme eine Gerade \(s(x)\), die orthogonal zu \( g(x)\) aus dem Teil b) liegt.
	
	
	\textbf{Auswertungstabelle:}
	\begin{center}
		\begin{tabular}{|c|c|c|c|c|c|c|c|}
			\hline
			Aufgabe & 1 & 2 & 3 & 4 & 5 & Summe & Note \\
			\hline
			Punkte & \text{/ \punkteA } & \text{/ \punkteB } & \text{/ \punkteC } & \text{/ \punkteD } & \text{/ \punkteE} & \summe & \\
			\hline
		\end{tabular}
	\end{center}
	
	\textbf{Notenschlüssel:}
	\begin{center}
		\begin{tabular}{|c|c|c|c|c|c|c|}
			\hline
			Note & 1 & 2 & 3 & 4 & 5 & 6 \\
			\hline
			Prozent \% & 100--90 & 89--75 & 74--60 & 59--45 & 44--20 & 19--0 \\
			\hline
			Punkte & \maxSumme{}--\noteEinsMin{} & \fpeval{\noteEinsMin-1}--\noteZweiMin{} & \fpeval{\noteZweiMin-1}--\noteDreiMin{} & \fpeval{\noteDreiMin-1}--\noteVierMin{} & \fpeval{\noteVierMin-1}--\noteFunfMin{} & \fpeval{\noteFunfMin-1}--\noteSechsMin{} \\
			\hline
		\end{tabular}
	\end{center}
	
	\vspace{3cm}
	\textbf{Kenntnisnahme eines Elternteils:} \hrulefill \hfill \textbf{Note:} \hrulefill
	
	\newpage
	\section*{Lösungen zur Klassenarbeit}
	
	\textbf{Aufgabe 1}
	\begin{itemize}
		\item[a)] Gerade durch $(0|-3)$, Steigung $2$.
		\item[b)] Gerade durch $(0|2)$, fallend mit Steigung $-\frac{1}{2}$.
		\item[c)] Gerade durch $(0|3)$, fallend mit Steigung $-1$.
	\end{itemize}
	
	\textbf{Aufgabe 2}
	\begin{itemize}
		\item[a)] Steigung: $m=\frac{7-3}{3-1}=2$, somit: $f(x)=2x+b$. Punkt A einsetzen: $3=2\cdot1+b$, ergibt $b=1$. \newline Funktionsgleichung: $f(x)=2x+1$.
		\item[b)] Nullstelle berechnen: $0=2x+1 \Rightarrow x=-\frac{1}{2}$.
		\item[c)] Punktprobe für $C(2|5)$: $f(2)=2\cdot2+1=5$. Der Punkt liegt auf der Geraden.
	\end{itemize}
	
	\textbf{Aufgabe 3}
	\begin{align*}
		4x-5 &= 7-2x \\ 6x &= 12 \\ x &= 2 \\ 
		\text{Einsetzen in } f(x):\quad y &=4\cdot2-5=3 \\ 
		\text{Schnittpunkt}&:(2|3)
	\end{align*}
	
	\textbf{Aufgabe 4}
	\begin{align*}
		f(3)&=3\cdot3-5=4 \quad\text{(Funktionswert)}\\
		P_y&=10 \quad\text{(Punktkoordinate)}\\
		\text{Da }10 &> 4, \text{liegt der Punkt }P(3|10) \text{ oberhalb der Geraden.}
	\end{align*}
	
	\textbf{Aufgabe 5}
	\begin{itemize}
		\item[a)] Nullstelle bei $x=65$, also $(65|0)$. Steigung $m=-2,7$:
		\[0=-2,7\cdot65+b \quad\Rightarrow\quad b=175,5\]
		Funktionsgleichung: $f(x)=-2,7x+175,5$
		\item[b)] Parallel bedeutet gleiche Steigung $m=2$, Nullstelle bei $(5|0)$:
		\[0=2\cdot5+b \quad\Rightarrow\quad b=-10\]
		Funktionsgleichung: $g(x)=2x-10$
		\item[c)] Zeichnung der beiden Geraden ins Koordinatensystem:
		\begin{itemize}
			\item $f(x)=-2,7x+175,5$
			\item $g(x)=2x-10$
		\end{itemize}
	\end{itemize}
	
\end{document}

