\documentclass[a4paper,12pt]{article}
\usepackage{amsmath, amssymb}
\usepackage[utf8]{inputenc}
\usepackage{geometry}
\geometry{a4paper, margin=2cm}
\usepackage{array}

\title{\textbf{Klassenarbeit Mathematik - Lösungen}}
\date{\today}
\author{Lösungen zu Rechnen mit Brüchen und Dezimalzahlen}

\begin{document}
	
	\maketitle
	
	\section*{Lösungen zu Aufgabe 1}
	
	\begin{enumerate}
		\item[a)] Berechnung von $\frac{2}{5} + \frac{3}{7}$:
		\begin{align*}
			\frac{2}{5} + \frac{3}{7} &= \frac{2 \cdot 7}{5 \cdot 7} + \frac{3 \cdot 5}{7 \cdot 5} \\
			&= \frac{14}{35} + \frac{15}{35} \\
			&= \frac{29}{35}
		\end{align*}
		
		\item[b)] Berechnung von $\frac{9}{10} - \frac{2}{3}$:
		\begin{align*}
			\frac{9}{10} - \frac{2}{3} &= \frac{9 \cdot 3}{10 \cdot 3} - \frac{2 \cdot 10}{3 \cdot 10} \\
			&= \frac{27}{30} - \frac{20}{30} \\
			&= \frac{7}{30}
		\end{align*}
		
		\item[c)] Berechnung von $\frac{4}{9} \times \frac{5}{6}$:
		\begin{align*}
			\frac{4}{9} \times \frac{5}{6} &= \frac{4 \cdot 5}{9 \cdot 6} \\
			&= \frac{20}{54} \\
			&= \frac{10}{27}
		\end{align*}
		
		\item[d)] Berechnung von $\frac{7}{8} \div \frac{3}{5}$:
		\begin{align*}
			\frac{7}{8} \div \frac{3}{5} &= \frac{7}{8} \times \frac{5}{3} \\
			&= \frac{35}{24}
		\end{align*}
		
		\item[e)] Berechnung von $\frac{5}{6} + \frac{2}{9} - \frac{1}{4}$:
		\begin{align*}
			\frac{5}{6} + \frac{2}{9} - \frac{1}{4} &= \frac{30}{36} + \frac{8}{36} - \frac{9}{36} \\
			&= \frac{29}{36}
		\end{align*}
	\end{enumerate}
	
	\section*{Lösungen zu Aufgabe 2}
	\begin{center}
		\renewcommand{\arraystretch}{1.5}
		\begin{tabular}{lll}
			a) $30\% = \frac{3}{10} = 0.3$ & b) $55\% = \frac{11}{20} = 0.55$ & c) $22.5\% = \frac{9}{40} = 0.225$ \\
			d) $75.5\% = \frac{151}{200} = 0.755$ & e) $12\% = \frac{3}{25} = 0.12$ & f) $90.8\% = \frac{227}{250} = 0.908$ \\
		\end{tabular}
	\end{center}
	
	\section*{Lösungen zu Aufgabe 3}
	\begin{center}
		\renewcommand{\arraystretch}{1.5}
		\begin{tabular}{lll}
			a) $\frac{2}{9}$ von 180m = \underline{$40$}m & b) $\frac{3}{5}$ von 5,5dm = \underline{$3.3$}dm & \\ c) $8\%$ von 3,2m = \underline{$0.256$}m & 
			d) $\frac{1}{4}$ von \underline{200} min sind 50min &
			\\ e) $\frac{7}{8}$ von \underline{$\approx82,28571429$} sind 72h & f) $\frac{5}{12}$ von \underline{240} sind 100€ \\
		\end{tabular}
	\end{center}
	
	\section*{Lösungen zu Aufgabe 4}
	\begin{align*}
		\text{Fußballspieler: } & \quad \frac{1}{4} \times 240 = 60 \\
		\text{Basketballspieler: } & \quad \frac{1}{3} \times 240 = 80 \\
		\text{Leichtathleten: } & \quad 20 \\
		\text{Mindestens eine Sportart: } & \quad 60 + 80 + 20 = 160 \text{ Schüler}
	\end{align*}
	
	\section*{Lösungen zu Aufgabe 5}
	\begin{enumerate}
		\item[a)] \text{Anteil der Bewohner, die nicht Fahrrad oder ÖPNV nutzen:}
		\begin{align*}
			1 - \left( \frac{1}{5} + \frac{1}{3} \right) &= 1 - \left( \frac{3}{15} + \frac{5}{15} \right) \\
			&= 1 - \frac{8}{15} = \frac{7}{15}
		\end{align*}
		
		\item[b)] \text{Berechnung der Gesamtbewohnerzahl:}
		\begin{align*}
			\frac{1}{3} x &= 1500 \\
			x &= 1500 \times 3 = 4500
		\end{align*}
		Also hat die Stadt insgesamt \textbf{4500 Bewohner}.
	\end{enumerate}
	
\end{document}
