\documentclass[a4paper,12pt]{article}
\usepackage{tabularx}
\usepackage{amsmath}
\usepackage[utf8]{inputenc}
\usepackage{amsmath, amssymb, amsthm}
\usepackage{graphicx}
\usepackage{array}
\usepackage[left=2cm, right=2cm, top=2cm, bottom=2cm]{geometry}
\usepackage{fancyhdr}
\usepackage{xfp} % Für mathematische Berechnungen in LaTeX
\usepackage{pgf} % Erlaubt Berechnungen mit \pgfmathparse

% Kopf- und Fußzeile
\pagestyle{fancy}
\lhead{Klassenarbeit 45min.}
\chead{Heinrich-von-Kleist-Schule}
\rhead{Mathematik - G6E}
\lfoot{}
\rfoot{}

% Makro für Punkte
\newcommand{\punkteA}{20}
\newcommand{\punkteB}{15}
\newcommand{\punkteC}{12}
\newcommand{\punkteD}{6}
\newcommand{\punkteE}{8}
\newcommand{\maxSumme}{61}

\begin{document}
	
	\begin{center}
		\textbf{Rechnen mit Brüchen und Dezimalzahlen}\\[0.2cm]
	\end{center}
	
	\textbf{Vor- und Nachname:} \underline{\hspace{10cm}}\\
	
	\textbf{Aufgabe 1} \hfill (8+12 = \punkteA{} Punkte)\\
	Berechne die folgenden Brüche.
	\begin{center}
		\renewcommand{\arraystretch}{1.5}
		\begin{tabular}{llll}
			a) $\frac{2}{5} + \frac{3}{7} =$ & b) $\frac{9}{10} - \frac{2}{3} =$ & c) $\frac{4}{9} \times \frac{5}{6} =$ & d) $\frac{7}{8} \div \frac{3}{5} =$ \\
			e) $\frac{5}{6} + \frac{2}{9} - \frac{1}{4} =$ & f) $\frac{4}{7} \times \frac{3}{5} \div \frac{2}{9} =$ & g) $\left( \frac{5}{8} + \frac{3}{10} \right) \times \frac{6}{11} =$ & h) $\frac{9}{13} \div \left( \frac{4}{7} - \frac{1}{5} \right) =$ \\
		\end{tabular}
	\end{center}
	
	\textbf{Aufgabe 2} \hfill (6+9 = \punkteB{} Punkte)\\
	Wandle die folgenden Prozentangaben in Brüche und Dezimalzahlen um.
	\begin{center}
		\renewcommand{\arraystretch}{1.5}
		\begin{tabular}{lll}
			a) $30\% =$ & b) $55\% =$ & c) $22.5\% =$ \\
			d) $75.5\% =$ & e) $12\% =$ & f) $90.8\% =$ \\
		\end{tabular}
	\end{center}
	
	\textbf{Aufgabe 3} \hfill (\punkteC{} Punkte)\\
	Berechne die entsprechenden Anteile.
	\begin{center}
		\renewcommand{\arraystretch}{1.5}
		\begin{tabular}{lll}
			a) $\frac{2}{9}$ von 180m = & b) $\frac{3}{5}$ von 5,5dm = & c) $8\%$ von 3,2m = \\
			d) $\frac{1}{4}$ von \dots min sind 50min & e) $\frac{7}{8}$ von \dots sind 72h = & f) $\frac{5}{12}$ von \dots sind 100€ \\
		\end{tabular}
	\end{center}
	
	\textbf{Aufgabe 4} \hfill (\punkteD{} Punkte)\\
	Eine Schule hat 240 Schüler. Ein Viertel davon spielt Fußball, ein Drittel spielt Basketball und 20 Schüler sind im Leichtathletik-Team. Wie viele Schüler üben eine dieser Sportarten aus? Begründe deine Antwort.
	
	\textbf{Aufgabe 5} \hfill (2+6 =\punkteE{} Punkte)\\
	In einer Stadt benutzen ein Fünftel der Bewohner das Fahrrad als Hauptverkehrsmittel, während ein Drittel öffentliche Verkehrsmittel nutzt. Der Rest fährt mit dem Auto oder geht zu Fuß.
	\begin{enumerate}
		\item[a)] Welcher Anteil (Bruch) der Bewohner nutzt nicht das Fahrrad oder öffentliche Verkehrsmittel?
		\item[b)] Wenn 1500 Bewohner öffentliche Verkehrsmittel nutzen, wie viele Bewohner hat die Stadt insgesamt?
	\end{enumerate}
	
\end{document}
