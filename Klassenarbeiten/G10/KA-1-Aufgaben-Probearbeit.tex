\documentclass[a4paper,12pt]{article}
\usepackage{tabularx}
\usepackage{amsmath}
\usepackage[utf8]{inputenc}
\usepackage{multicol}
\usepackage{amsmath, amssymb, amsthm}
\usepackage{graphicx}
\usepackage{enumitem}
\usepackage{array}
\usepackage[left=2cm, right=2cm, top=2cm, bottom=2cm]{geometry}
\usepackage{fancyhdr}
\usepackage{xfp}
\usepackage{pgf}

\pagestyle{fancy}
%\lhead{Probe-Klassenarbeit 45min.}
\lhead{Beispielaufgaben-Klassenarbeit}
\chead{Heinrich-von-Kleist-Schule}
\rhead{Mathematik - G10A}
\lfoot{}
\cfoot{Seite \thepage}
\rfoot{}

\newcommand{\punkteA}{0}
\newcommand{\punkteB}{0}
\newcommand{\punkteC}{0}
\newcommand{\punkteD}{0}
\newcommand{\punkteE}{0}

\newcommand{\maxSumme}{45}
\newcommand{\noteEinsMin}{\fpeval{round(\maxSumme * 0.95,0)}}
\newcommand{\noteZweiMin}{\fpeval{round(\maxSumme * 0.80,0)}}
\newcommand{\noteDreiMin}{\fpeval{round(\maxSumme * 0.60,0)}}
\newcommand{\noteVierMin}{\fpeval{round(\maxSumme * 0.45,0)}}
\newcommand{\noteFunfMin}{\fpeval{round(\maxSumme * 0.20,0)}}
\newcommand{\noteSechsMin}{0}

\newcommand{\summe}{%
	\pgfmathparse{\punkteA + \punkteB + \punkteC + \punkteD + \punkteE}%
	\pgfmathprintnumber{\pgfmathresult}}

\begin{document}
	
	\begin{center}
		\textbf{Vorbereitung auf die Klassenarbeit - Potenzen}
	\end{center}
	
	\textbf{Vor- und Nachname:} \underline{\hspace{10cm}}\\[0.1cm]
	Die Lösungen sowie Lösungswege sollten klar strukturiert und gut nachvollziehbar sein.\\[0.1cm]
	
	\textbf{Aufgabe 1 (Punkte)}\\
% Aufgabe: Zehnerpotenzen – Umwandlung in wissenschaftliche Schreibweise
Wandle die folgenden Zahlen \emph{ohne Rundung} in die wissenschaftliche Schreibweise \(a\cdot 10^{n}\) mit \(1\le a<10\) um.

\begin{multicols}{3}
	\begin{enumerate}[label=\alph*)]
		\item \(3\,450\,000\)
		\item \(0{,}000078\)
		\item \(120\,000\,000\,000\)
		\item \(0{,}45\)
		\item \(9\,006\,000\)
		\item \(0{,}0000005\)
		\item \(0{,}0032\)
		\item \(5\,600\,000\,000\,000\)
		\item \(0{,}00012\)
		\item \(67\,800\)
		\item \(98\,000\,000\)
		\item \(0{,}00000000031\)
	\end{enumerate}
\end{multicols}

% 3-spaltiges Layout – Von Zehnerpotenzen zur Dezimalschreibweise
% (Voraussetzungen in der Präambel:)
% \usepackage{multicol}
% \usepackage{enumitem}

\textbf{Aufgabe 2 (Punkte)}\\
Wandle die folgenden Zahlen \emph{ohne Rundung} aus der Form \(a\cdot 10^{n}\)Je in die normale Dezimalschreibweise um.

\begin{multicols}{3}
	\begin{enumerate}[label=\alph*)]
		\item \(4{,}5\cdot 10^{3}\)
		\item \(7\cdot 10^{-4}\)
		\item \(1{,}203\cdot 10^{2}\)
		\item \(9{,}08\cdot 10^{-2}\)
		\item \(3\cdot 10^{0}\)
		\item \(2{,}75\cdot 10^{5}\)
		\item \(6{,}007\cdot 10^{-3}\)
		\item \(8{,}1\cdot 10^{1}\)
		\item \(5\cdot 10^{-6}\)
		\item \(9{,}999\cdot 10^{4}\)
		\item \(1{,}25\cdot 10^{-1}\)
		\item \(3{,}402\cdot 10^{0}\)
	\end{enumerate}
\end{multicols}

% Potenzen nur mit Zahlen – alle Arten (gleiche Basis/Exponenten, ganzzahlige & rationale Exponenten, + − · :)
% (Voraussetzungen in der Präambel:)
% \usepackage{multicol}
% \usepackage{enumitem}

\textbf{Aufgabe 3 (Punkte))}\\
Berechne und vereinfache vollständig ohne Taschenrechner.

\begin{multicols}{3}
	\begin{enumerate}[label=\alph*)]
		\item \(2^{3}\cdot 2^{5}\)
		\item \(5^{7} : 5^{2}\)
		\item \(\bigl(3^{4}\bigr)^{2}\)
		\item \(10^{-3}\cdot 10^{5}\)
		\item \(4^{-2}\)
		\item \(27^{\tfrac{2}{3}}\)
		\item \(32^{-\tfrac{3}{5}}\)
		\item \(16^{\tfrac{1}{4}}\cdot 8^{\tfrac{2}{3}}\)
		\item \(81^{\tfrac{3}{4}} : 3^{\tfrac{1}{2}}\)
		\item \(\bigl(2^{5}-2^{3}\bigr) : 2^{2}\)
		\item \(\bigl(5^{3}\cdot 4^{3}\bigr) : 10^{3}\)
		\item \(\left(\tfrac{1}{8}\right)^{-\tfrac{2}{3}}\)
	\end{enumerate}
\end{multicols}

% Potenzen mit Variablen – alle Arten (gleiche Basis/Exponenten, ganz-/rationale Exponenten, + − · :)
% (Voraussetzungen in der Präambel:)
% \usepackage{multicol}
% \usepackage{enumitem}

% Potenzen mit Variablen – alle Arten (korrigiert)
% \usepackage{multicol}
% \usepackage{enumitem}

\textbf{Aufgabe 4 (Punkte)}\\
Vereinfache vollständig. \emph{Hinweis:} Gehe von \(a,b,c,m,n,p,x,y>0\) aus und gib das Ergebnis ohne negative Exponenten an.

\begin{multicols}{3}
	\begin{enumerate}[label=\alph*)]
		\item \(a^{3}\cdot a^{5}\)
		\item \(x^{7} : x^{2}\)
		\item \(\bigl(y^{4}\bigr)^{2}\)
		\item \(b^{-3}\cdot b^{5}\)
		\item \(c^{-2}\)
		\item \(\bigl(x^{6}y^{3}\bigr)^{\tfrac12}\)
		\item \(\bigl(a^{\tfrac12}b^{\tfrac32}\bigr) : a^{\tfrac14}\)
		\item \(x^{\tfrac{2}{3}}\cdot x^{\tfrac{4}{3}}\)
		\item \((mn)^{3} : m^{2}n\)
		\item \(\bigl(p^{5}-p^{3}\bigr) : p^{2}\)
		\item \(\dfrac{a^{3}b^{-2}}{a^{-1}b^{4}}\)
		\item \(\dfrac{x^{-\tfrac{3}{2}}y}{x^{-\tfrac{1}{2}}y^{-2}}\)
	\end{enumerate}
\end{multicols}

% Potenzen & Wurzeln gemischt (3-spaltig)
% Voraussetzungen in der Präambel:
% \usepackage{multicol}
% \usepackage{enumitem}
\newpage
\textbf{Aufgabe 5 (Punkte)}\\
Vereinfache vollständig. Schreibe das Ergebnis ohne negative Exponenten; rationalisiere ggf. den Nenner.
\emph{Hinweis:} Es gelte \(a,b,x,y>0\).

\begin{multicols}{3}
	\begin{enumerate}[label=\alph*)]
		\item \(\sqrt{a^{3}b}\cdot a^{\tfrac12}\)
		\item \(\dfrac{x^{\tfrac32}\cdot\sqrt{8x}}{\sqrt{2}}\)
		\item \(\dfrac{\sqrt[3]{27\,a^{6}b^{3}}}{a}\)
		\item \(\dfrac{\sqrt{45\,x^{5}}}{x^{\tfrac32}\sqrt{5x}}\)
		\item \(\bigl(\sqrt{12}+\sqrt{3}\bigr)^{2}\)
		\item \(\dfrac{2}{\sqrt{x}}+\dfrac{\sqrt{x}}{2}\)
		\item \(\left(\dfrac{\sqrt{a}}{\sqrt[4]{a^{3}}}\right)^{2}\)
		\item \(\dfrac{\sqrt{x}\cdot x^{\tfrac34}}{\sqrt[4]{x}}\)
		\item \(\dfrac{\sqrt{18\,a^{3}}}{\sqrt{2a}}\cdot\sqrt{a}\)
		\item \(\sqrt{\dfrac{a}{b}}\cdot\sqrt{\dfrac{b}{a^{3}}}\)
		\item \(\dfrac{\sqrt{5}-\sqrt{2}}{\sqrt{5}+\sqrt{2}}\)
		\item \(\sqrt[4]{16\,x^{2}y^{6}}\cdot\sqrt{\dfrac{y}{x}}\)
	\end{enumerate}
\end{multicols}

% Potenzen mit irrationalen Exponenten (3-spaltig)
% Voraussetzungen in der Präambel:
% \usepackage{multicol}
% \usepackage{enumitem}

\textbf{Aufgabe 6 (Punkte)}\\
Vereinfache vollständig und gib das Ergebnis \emph{ohne negative Exponenten} an.
\emph{Hinweis:} Es gelte \(a,b,m,p,x,y>0\).

\begin{multicols}{3}
	\begin{enumerate}[label=\alph*)]
		\item \(a^{\sqrt{2}}\cdot a^{3\sqrt{2}}\)
		\item \(\dfrac{x^{\pi}}{x^{2\pi}}\)
		\item \(\bigl(b^{\sqrt{5}}\bigr)^{2}\)
		\item \(\bigl(\sqrt{a}\bigr)^{\pi}\cdot a^{\tfrac{\pi}{2}}\)
		\item \(\dfrac{\bigl(m^{\sqrt{3}}\bigr)^{4}}{m^{2\sqrt{3}}}\)
		\item \(\dfrac{(pq)^{\sqrt{2}}}{p^{\sqrt{2}}}\)
		\item \(9^{\sqrt{2}}\cdot 3^{\sqrt{2}}\)
		\item \(\dfrac{16^{\sqrt{2}}}{2^{\sqrt{2}}}\)
		\item \(\bigl(x^{-\sqrt{7}}\bigr)^{-1}\)
		\item \(\sqrt[3]{a^{2\sqrt{2}}}\cdot \sqrt[3]{a^{\sqrt{2}}}\)
		\item \(\left(\dfrac{\sqrt[4]{\,b\,}}{\sqrt{b}}\right)^{\sqrt{2}}\)
		\item \(\dfrac{\bigl(10^{\sqrt{2}}\bigr)^{3}}{\bigl(\sqrt{10}\bigr)^{\sqrt{2}}}\)
	\end{enumerate}
\end{multicols}

% Lange Terme – alle Arten der Potenzen gemischt (3-spaltig)
% Voraussetzungen in der Präambel:
% \usepackage{multicol}
% \usepackage{enumitem}

\textbf{Aufgabe 7 (Punkte)}\\
Vereinfache vollständig. Schreibe ohne negative Exponenten und rationalisiere ggf. den Nenner.
\emph{Hinweis:} Es gelte \(a,b,x,y>0\).

% Präambel-Ergänzung:
% \usepackage{graphicx}

\setlength{\columnsep}{10pt} % etwas schmalerer Spaltenabstand
\begin{multicols}{3}
	\small
	\begin{enumerate}[label=\alph*) , leftmargin=*, itemsep=2pt]
		\item \(\displaystyle \frac{(2a^{\tfrac{3}{2}}b^{-1})^{2}\cdot \sqrt{8ab^{5}}}{(ab^{2})^{\tfrac{3}{2}}}\)
		\item \(\displaystyle \left(\frac{\sqrt{18x^{5}}}{3x^{\tfrac12}}\right)^{2}\cdot \frac{x^{-\tfrac{3}{2}}}{\sqrt[3]{x^{-3}}}\)
		\item \(\displaystyle \frac{(a^{\sqrt{2}}b^{\tfrac12})^{2}\cdot \sqrt{ab^{3}}}{a^{1+\sqrt{2}}\,b}\)
		\item \(\displaystyle \left(\frac{\sqrt[3]{27\,y^{6}}}{\sqrt{y}}\right)\cdot\left(\frac{y^{\tfrac{3}{2}}}{\sqrt[4]{y^{3}}}\right)^{-1}\)
		\item \(\displaystyle \left(\frac{5^{3}\cdot 4^{3}}{10^{3}}\right)^{\tfrac12}\cdot \frac{2^{-\tfrac{3}{2}}}{\sqrt[3]{8}}\)
		\item \(\displaystyle \left(\frac{a^{-2}b^{3}}{\sqrt{ab}}\right)^{\tfrac32}\cdot\left(\frac{\sqrt{a}}{b}\right)^{2}\)
		\item \(\displaystyle \frac{(x^{\tfrac13}y^{-\tfrac12})^{-3}}{\sqrt[4]{(xy^{3})^{2}}}\)
		\item \(\displaystyle \left(\frac{\sqrt{a}-\sqrt{b}}{\sqrt{a}+\sqrt{b}}\right)^{2}\)
		\item \(\displaystyle \frac{3^{\sqrt{2}}\cdot 9^{\sqrt{2}}}{(\sqrt{3})^{\sqrt{2}}\cdot 3^{2\sqrt{2}}}\)
		\item \(\displaystyle \frac{(\sqrt{a}+\sqrt{b})^{2}-(a+b)}{\sqrt{ab}}\)
		\item \(\displaystyle \frac{\sqrt[3]{64\,a^{6}}\cdot \sqrt{\frac{a}{b^{3}}}}{(a^{2}b^{-1})^{\tfrac32}}\)
		% l) skalieren, damit es in die Spalte passt
		\item \resizebox{\linewidth}{!}{$
			\displaystyle
			\frac{\sqrt{50\,x^{3}y^{5}}-2\sqrt{2xy}\cdot\sqrt{8x^{2}y^{3}}+\sqrt{200\,x^{3}y^{5}}}{x^{\tfrac12}y}
			$}
	\end{enumerate}
\end{multicols}

% Schwierige Aufgaben – einfache Ergebnisse
% (Hinweis: Werte ausschließen, für die Nenner = 0.)
\newpage
\textbf{Aufgabe 8 (Punkte)}.
\begin{enumerate}[label=\alph*)]
	\item \[
	\left(\frac{(3x+6)^{\,n-1}\cdot(3x+6)^{\,n+2}}{(x+2)^{\,2n+1}}\right) : 3^{\,2n}
	\]
	% Ergebnis: 3  \; (weil 3x+6=3(x+2))
	
	\item \[
	\left(\frac{9-a^{2}}{a+3}\right)^{3}
	\cdot
	\left(\frac{15+5a}{\,3-a\,}\right)^{3}
	\cdot
	\left(\frac{a+3}{a^{2}+6a+9}\right)^{3}
	\]
	% Ergebnis: 5^{3}=125 \; (vollständige Kürzung)
	
	\item \[
	\left(\frac{x^{2}-10x+25}{x-4}\right)^{3}
	:\left(\frac{x-5}{x^{2}-16}\right)^{3}
	\cdot
	\left(\frac{4}{(x-5)(x+4)}\right)^{3}
	\]
	% Ergebnis: 4^{3}=64 \; (alles kürzt sich weg)
\end{enumerate}

% Definitionsmengen:
% a) x \ne -2 \qquad
% b) a \ne -3,\, a \ne 3 \qquad
% c) x \ne 5,\, x \ne \pm4


	
\end{document}
