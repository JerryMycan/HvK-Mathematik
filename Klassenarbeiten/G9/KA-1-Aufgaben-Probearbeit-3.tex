\documentclass[a4paper,12pt]{article}
\usepackage{tabularx}
\usepackage{amsmath}
\usepackage[utf8]{inputenc}
\usepackage{amsmath, amssymb, amsthm}
\usepackage{graphicx}
\usepackage{enumitem}
\usepackage{multicol}
\usepackage{array}
\usepackage[left=2cm, right=2cm, top=2cm, bottom=2cm]{geometry}
\usepackage{fancyhdr}
\usepackage{xfp}
\usepackage{pgf}

\usepackage{graphicx}
\usepackage{fancyhdr}
\setlength{\headheight}{28pt} % genug Platz für das Logo
\pagestyle{fancy}
\fancyhf{} % alles leeren
\fancyhead[L]{\includegraphics[height=1.2cm]{logo.png}}
\fancyhead[C]{\small Probe-Klassenarbeit – LGS und Wurzelgesetze \ (Kl. G9A)}
%\fancyhead[R]{\small Name:\ \rule{2.8cm}{0.4pt}}
\fancyfoot[C]{\thepage}

\fancyfoot[C]{Seite \thepage \enspace\textbullet\enspace J.\,Mycan \textcopyright~2025}

\renewcommand{\footrulewidth}{0.4pt}
%\pagestyle{fancy}
%\lhead{Probe-Klassenarbeit 45min.}
%\chead{Heinrich-von-Kleist-Schule}
%\rhead{Mathematik - G9A}
%\lfoot{}
%\cfoot{Seite \thepage}
%\rfoot{}

\newcommand{\punkteA}{8}
\newcommand{\punkteB}{8}
\newcommand{\punkteC}{12}
\newcommand{\punkteD}{6}
\newcommand{\punkteE}{12}

\newcommand{\maxSumme}{44}
\newcommand{\noteEinsMin}{\fpeval{round(\maxSumme * 0.95,0)}}
\newcommand{\noteZweiMin}{\fpeval{round(\maxSumme * 0.80,0)}}
\newcommand{\noteDreiMin}{\fpeval{round(\maxSumme * 0.60,0)}}
\newcommand{\noteVierMin}{\fpeval{round(\maxSumme * 0.45,0)}}
\newcommand{\noteFunfMin}{\fpeval{round(\maxSumme * 0.20,0)}}
\newcommand{\noteSechsMin}{0}

\newcommand{\summe}{%
	\pgfmathparse{\punkteA + \punkteB + \punkteC + \punkteD + \punkteE}%
	\pgfmathprintnumber{\pgfmathresult}}
	
	
\begin{document}
	

	\textbf{Vor- und Nachname:} \underline{\hspace{10cm}}\\[0.1cm]
Die Lösungen sowie Lösungswege sollten klar strukturiert und gut nachvollziehbar sein. \\[0.1cm]
	
	% ===================== A1 =====================
	\textbf{Aufgabe 1 (8 Punkte)}\\
	Gegeben ist ein Koordinatensystem, in dem die Graphen der linearen Funktionen \(f\) und \(g\) eingezeichnet sind (siehe Abbildung~1).\\
	Bestimme den Schnittpunkt der beiden Funktionen \emph{rechnerisch}.
	Bestimme dazu zunächst aus der Abbildung die Funktionsgleichungen von \(f\) und \(g\).
	Stelle anschließend das lineare Gleichungssystem
	%	\[
	%	\begin{cases}
		%		y = f(x)\\
		%		y = g(x)
		%	\end{cases}
	%	\]
	auf und löse es.
	Gib das Ergebnis als Punkt \(S(x_S \mid y_S)\) an.
	
	\begin{figure}[h!]
		\centering
		\includegraphics[width=0.7\textwidth]{plot-3.png}
		\caption{Graph der linearen Funktionen \(f\) und \(g\)}
	\end{figure}

	
	% ===================== A2 =====================
	\bigskip
	\textbf{Aufgabe 2 (8 Punkte)}\\
	In der \emph{Bäckerei} kosten \emph{Brötchen} und \emph{Brezeln} unterschiedlich viel.
	Kundin A bezahlt für \emph{3 Brötchen und 4 Brezeln} insgesamt \(\,2{,}50\,€\).
	Kunde B bezahlt für \emph{2 Brötchen und 7 Brezeln} insgesamt \(\,3{,}40\,€\).

	
	\begin{enumerate}[label=\alph*)]
		\item Lege Variablen fest (Preis pro Farbdruck, Preis pro Schwarzweißkopie) und stelle ein LGS auf. \hfill (3P)
		\item Löse das LGS rechnerisch mit einem geeigneten Verfahren und gib beide Einzelpreise an. \hfill (5P)
	\end{enumerate}
	
	% (Hinweis für die Lehrkraft: Lösung ist ganzzahlig und einstellig.)
	
	% ===================== A3 =====================
	\newpage
	
	\textbf{Aufgabe 3 (12 Punkte)}\\
	Gegeben ist das lineare Gleichungssystem in Matrixschreibweise
	\[
	A\vec{x}=\vec{b},\quad
	A=\begin{pmatrix}
		1 & 1 & 0\\
		0 & 1 & 1\\
		1 & 0 & 1
	\end{pmatrix},\quad
	\vec{b}=\begin{pmatrix}
		2\\
		2\\
		2
	\end{pmatrix}.
	\]
	Löse das Gleichungssystem rechnerisch (z.\,B. mit dem Gauß-Verfahren).
	% Hinweis: Die Lösung ist klein und ganzzahlig.


	
	% ===================== A4 =====================
	\bigskip
	\textbf{Aufgabe 4 (6 Punkte)}\\
	Vereinfache so weit wie möglich, indem du vollständige Quadrate aus der Wurzel ziehst.
	\begin{multicols}{2}
		\begin{enumerate}[label=\alph*)]
			\item \(\sqrt{9000}\)
			\item \(\sqrt{176}\)
			\item \(\sqrt{0,512}\)
			\item \(\sqrt{605}\)
			\item \(\sqrt{864}\)
			\item \(\sqrt{245}\)
		\end{enumerate}
	\end{multicols}
	
	% ===================== A5 =====================
	\bigskip
	\textbf{Aufgabe 5 (12 Punkte)}\\
	Vereinfache vollständig; rationalisiere ggf. den Nenner. Es gelte \(a,b,x,y>0\).
	\begin{multicols}{2}
		\begin{enumerate}[label=\alph*)]
			\item \(\dfrac{\sqrt{48\,a^{5}}}{\sqrt{6a}}\)
			\item \(\sqrt{9x}\cdot\sqrt{8x^{3}}\)
			\item \(\dfrac{2}{\sqrt{x}}+\dfrac{\sqrt{x}}{\sqrt{12}}\)
			\item \(\left(\dfrac{\sqrt{a}}{\sqrt{a^{3}}}\right)^{2}\cdot\dfrac{a^{2}}{\sqrt{a}}\)
			\item \(\dfrac{\sqrt{27a^{2}b}}{\sqrt{3b}}\cdot \sqrt{b}\)
			\item \(\displaystyle \frac{2}{\sqrt{5x}}+\frac{3\sqrt{x}}{\sqrt{20}}\)
		\end{enumerate}
	\end{multicols}
	
	% ===================== A6 =====================
%	\textbf{Aufgabe 6 (Punkte)}\\
%	Vereinfache mithilfe der Wurzelgesetze. Rationalisiere ggf. den Nenner.
%	\emph{Hinweis:} Es gelte \(a,b,x,y>0\).
%	
%	\begin{multicols}{2}
%		\begin{enumerate}[label=\alph*)]
%			\item \(\sqrt{18a^{2}b}\)
%			\item \(\sqrt{12x}\cdot\sqrt{27x^{3}}\)
%			\item \(\displaystyle \frac{\sqrt{48a^{5}}}{\sqrt{3a}}\)
%			\item \(5\sqrt{2x}-2\sqrt{8x}+3\sqrt{18x}\)
%			\item \(\sqrt{\dfrac{9a^{3}b}{4a}}\cdot \dfrac{\sqrt{b}}{\sqrt{a^{2}}}\)
%			\item \(\displaystyle \frac{2}{\sqrt{5x}}+\frac{3\sqrt{x}}{\sqrt{20}}\)
%		\end{enumerate}
%	\end{multicols}

	
	
	\vspace{3 cm}
	\textbf{Auswertungstabelle:}
	\begin{center}
		\begin{tabular}{|c|c|c|c|c|c|c|c|}
			\hline
			Aufgabe & 1 & 2 & 3 & 4 & 5 &  Summe\\
			\hline
		\hline
	Punkte & \text{\ / \punkteA} & \text{\ / \punkteB} & \text{\ / \punkteC} & \text{\ / \punkteD} & \text{\ / \punkteE} & \text{\ / \summe}\\
	\hline
		\end{tabular}
	\end{center}
	
	\textbf{Notenschlüssel:}
	\begin{center}
		\begin{tabular}{|c|c|c|c|c|c|c|}
			\hline
			Note & 1 & 2 & 3 & 4 & 5 & 6 \\
			\hline
			Prozent \% & 100--95 & 94--80 & 79--60 & 59--45 & 44--16 & 15--0 \\
			\hline
			Punkte & \maxSumme{}--\noteEinsMin{} & \fpeval{\noteEinsMin-1}--\noteZweiMin{} & \fpeval{\noteZweiMin-1}--\noteDreiMin{} & \fpeval{\noteDreiMin-1}--\noteVierMin{} & \fpeval{\noteVierMin-1}--\noteFunfMin{} & \fpeval{\noteFunfMin-1}--\noteSechsMin{} \\
			\hline
		\end{tabular}
	\end{center}
	
	\vspace{1cm}
	\textbf{Kenntnisnahme eines Elternteils:} \hrulefill \hfill \textbf{Note:} \hrulefill
	
\end{document}
