\documentclass[a4paper,12pt]{article}
\usepackage{tabularx}
\usepackage{amsmath, amssymb, amsthm}
\usepackage[utf8]{inputenc}
\usepackage{graphicx}
\usepackage{array}
\usepackage[left=2cm, right=2cm, top=2cm, bottom=2cm]{geometry}
\usepackage{fancyhdr}

% Kopf- und Fußzeile
\pagestyle{fancy}
\lhead{Probeklassenarbeit 45min.}
\chead{Heinrich-von-Kleist-Schule}
\rhead{Mathematik - G8A}
\lfoot{}
\cfoot{Seite \thepage}
\rfoot{}

\begin{document}
	
	\begin{center}
		\textbf{Un- und Gleichungen, Bruchgleichungen, quadratische Ergänzung, Wahrscheinlichkeitsrechnung}\
		\vspace{0.2cm}
	\end{center}
	
	\textbf{Vor- und Nachname:} \underline{\hspace{10cm}}\\
	
	\textbf{Aufgabe 1} \hfill (15 Punkte)\\
	Löse folgende Gleichungen und Ungleichung nach \( x \):
	\[
	\renewcommand{\arraystretch}{1.5} 
	\begin{tabularx}{\textwidth}{>{\centering\arraybackslash}X 
			>{\centering\arraybackslash}X 
			>{\centering\arraybackslash}X}
		\textbf{a)} \( (3x - 7)^2 = 9x^2 - 42x + 49 \) &
		\textbf{b)} \( x(2x - 14) + 49 = (x - 7)^2 \) &
		\textbf{c)} \( \frac{x + 2}{3} \leq \frac{3x - 4}{5} \)
	\end{tabularx}
	\]
	
	\textbf{Aufgabe 2} \hfill (2,3,2 Punkte)\\
	Löse folgende Bruchgleichungen.
	\[
	\renewcommand{\arraystretch}{1.5}
	\begin{tabularx}{\textwidth}{>{\centering\arraybackslash}X 
			>{\centering\arraybackslash}X 
			>{\centering\arraybackslash}X}
		\textbf{a)} \quad \( \frac{x}{5} + \frac{4}{x} = 3 \) &
		\textbf{b)} \quad \( \frac{3x}{x+2} - \frac{5}{x-2} = \frac{2}{x^2 - 4} \) &
		\textbf{c)} \quad \( \frac{3}{x} + \frac{4}{x-2} = \frac{7}{x^2 - 2x} \)
	\end{tabularx}
	\]
	
	\textbf{Aufgabe 3} \hfill (6 Punkte)\\
	Fasse die folgenden Terme mithilfe der quadratischen Ergänzung zu einem Binom zusammen.
	\[
	\renewcommand{\arraystretch}{1.5}
	\begin{tabularx}{\textwidth}{>{\centering\arraybackslash}X 
			>{\centering\arraybackslash}X 
			>{\centering\arraybackslash}X
			>{\centering\arraybackslash}X}
		\textbf{a)} \quad \( x^2 + 8x \) &
		\textbf{b)} \quad \( x^2 - 12x + 9 \) &
		\textbf{c)} \quad \( 3x^2 + 12x - 7 \) &
		\textbf{d)} \quad \( 4x^2 - 16x + 15 \)
	\end{tabularx}
	\]
	
	\textbf{Aufgabe 4} \hfill (9 Punkte)\\
	Ein Beutel enthält **jeweils 3 rote, 4 blaue und 3 grüne Kugeln**. Es werden **nacheinander drei Kugeln ohne Zurücklegen** gezogen.
	\begin{itemize}
		\item \textbf{a)} Zeichne ein Baumdiagramm, das alle möglichen Ziehungen darstellt.
		\item \textbf{b)} Berechne die Wahrscheinlichkeit, dass alle drei Kugeln dieselbe Farbe haben.
		\item \textbf{c)} Berechne die Wahrscheinlichkeit, dass genau zwei Kugeln dieselbe Farbe haben und die dritte Kugel eine andere Farbe.
		\item \textbf{d)} Berechne die Wahrscheinlichkeit, dass alle drei Kugeln unterschiedliche Farben haben.
	\end{itemize}
	
	\textbf{Aufgabe 5} \hfill (9 Punkte)\\
	In einer Kiste befinden sich **7 Schokoladen, 6 Karamell- und 5 Erdbeerbonbons**. Ein Kind zieht **nacheinander zwei Bonbons ohne Zurücklegen**.
	\begin{itemize}
		\item \textbf{a)} Zeichne ein Baumdiagramm für die möglichen Ziehungen.
		\item \textbf{b)} Berechne die Wahrscheinlichkeit, dass beide Bonbons Schokolade sind.
		\item \textbf{c)} Berechne die Wahrscheinlichkeit, dass das erste Bonbon Schokolade und das zweite Erdbeere ist.
		\item \textbf{d)} Berechne die Wahrscheinlichkeit, dass mindestens ein Bonbon Karamell ist.
	\end{itemize}
	
\end{document}
