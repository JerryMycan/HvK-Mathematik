\documentclass[a4paper,12pt]{article}
\usepackage{tabularx}
\usepackage{amsmath}
\usepackage[utf8]{inputenc}
\usepackage{amsmath, amssymb, amsthm}
\usepackage{graphicx}
\usepackage{array}
\usepackage[left=2cm, right=2cm, top=2cm, bottom=2cm]{geometry}
\usepackage{fancyhdr}
\usepackage{xfp}
\usepackage{pgf}

\pagestyle{fancy}
\lhead{Probe-Klassenarbeit 45min.}
\chead{Heinrich-von-Kleist-Schule}
\rhead{Mathematik - G8A}
\lfoot{}
\cfoot{Seite \thepage}
\rfoot{}

\newcommand{\punkteA}{6}
\newcommand{\punkteB}{9}

\newcommand{\punkteC}{5}
\newcommand{\punkteD}{12}
\newcommand{\punkteE}{8}

\newcommand{\maxSumme}{40}
\newcommand{\noteEinsMin}{\fpeval{round(\maxSumme * 0.96,0)}}
\newcommand{\noteZweiMin}{\fpeval{round(\maxSumme * 0.80,0)}}
\newcommand{\noteDreiMin}{\fpeval{round(\maxSumme * 0.60,0)}}
\newcommand{\noteVierMin}{\fpeval{round(\maxSumme * 0.45,0)}}
\newcommand{\noteFunfMin}{\fpeval{round(\maxSumme * 0.20,0)}}
\newcommand{\noteSechsMin}{0}

\newcommand{\summe}{%
	\pgfmathparse{\punkteA + \punkteB + \punkteC + \punkteD + \punkteE}%
	\pgfmathprintnumber{\pgfmathresult}}

\begin{document}
	
	\begin{center}
		\textbf{Probe-Klassenarbeit - Lineare Funktionen und Gleichungen}
	\end{center}
\textbf{Aufgabe 1 (Realitätsbezug - Kauf und Verkauf)}\\
Ein Händler kauft Äpfel für 1,50 € pro kg und verkauft sie mit 0,70 € Gewinn pro kg. Stelle die lineare Funktion auf und berechne den Gewinn bei 40 kg.\\
\\
\textbf{Aufgabe 2 (Modellierung und Interpretation)}\\
Ein Baum ist nach 2 Jahren 1,5 m hoch, nach 5 Jahren 3 m. Ermittle die Funktionsgleichung und erkläre, was Steigung und y-Achsenabschnitt bedeuten.\\
\\
\textbf{Aufgabe 3 (Fehleranalyse und Korrektur)}\\
Ein Schüler gibt an, die Gerade durch (1|4) und (2|8) habe die Gleichung $y=4x$. Prüfe, ob dies stimmt. Begründe und korrigiere gegebenenfalls.\\
\\
\textbf{Aufgabe 4 (Grafische Interpretation)}\\
Zeichne den Graphen von $y=-3x+1$ und bestimme grafisch die neue Funktionsgleichung, wenn der Graph um 2 Einheiten nach oben verschoben wird.\\
\\
\textbf{Aufgabe 5 (Vergleichende Analyse)}\\
Tarif A kostet 10 € Grundgebühr und 0,15 € pro Minute, Tarif B hat keine Grundgebühr, dafür 0,35 € pro Minute. Stelle für beide Tarife die Gleichungen auf und berechne, ab wie vielen Minuten Tarif A günstiger ist.\\
\\
\textbf{Aufgabe 6 (Besondere Lagebeziehungen)}\\
Die Gerade $g$ hat die Gleichung $y=3x+5$. Ermittle eine zu $g$ senkrechte Gerade $h$, die durch den Punkt (0|2) verläuft.\\
\\
\textbf{Aufgabe 7 (Rückwärtsrechnung)}\\
Eine lineare Funktion hat die Steigung 2,5 und verläuft durch den Punkt (4|0). Ermittle den y-Achsenabschnitt und gib die Funktionsgleichung an.\\
\\
\textbf{Aufgabe 8 (Erstellen eigener Beispiele)}\\
Erstelle eine lineare Funktion, deren Graph die y-Achse oberhalb des Ursprungs schneidet und eine negative Steigung hat. Begründe deine Wahl.\\

\end{document}

