\documentclass[a4paper,12pt]{article}
\usepackage[utf8]{inputenc}
\usepackage[ngerman]{babel}
\usepackage{amsmath}
\usepackage{geometry}
\geometry{a4paper,left=2.5cm,right=2.5cm,top=2.5cm,bottom=3cm}

\begin{document}
	
	\begin{center}
		\textbf{Heinrich-von-Kleist-Schule Eschborn}\\
		\textbf{Arbeitsblatt Mathematik, Klasse G6E}\\
		\textbf{Thema: Große Zahlen und Maßeinheiten}\\
		\vspace{0.5cm}
	\end{center}
	
	\textbf{Aufgabe 1 – Staatsverschuldung Deutschlands}\\
	
	In der politischen Entwicklung der letzten Tage ist sehr viel passiert. Deutschland möchte Kredite aufnehmen, um verschiedene Reformen anzustoßen. Die gesamte geplante Neuverschuldung beträgt 1,5 Billionen Euro (in Zahlen: 1.500.000.000.000 €).\\
	
	\textbf{a)} Stell dir vor, du könntest jede Sekunde genau eine 1€-Münze auf einen Stapel legen. Wie lange würdest du brauchen, bis alle 1,5 Billionen Münzen gestapelt sind? Gib die Zeitdauer in Jahren an.\\
	
	\textbf{Lösung:}
	
	Es gibt 60 Sekunden pro Minute, 60 Minuten pro Stunde, 24 Stunden pro Tag und etwa 365 Tage pro Jahr.
	
	\begin{align*}
		1.500.000.000.000\text{ s} &: (60\text{ s/min} \cdot 60\text{ min/h} \cdot 24\text{ h/d} \cdot 365\text{ d/Jahr}) \\
		&= \frac{1.500.000.000.000}{31.536.000} \approx 47.564,69 \text{ Jahre}
	\end{align*}
	
	\textbf{Antwort:} Etwa 47.565 Jahre.\\
	
	\textbf{b)} Eine 1€-Münze hat eine Dicke von 1 mm. Wie hoch wäre der Stapel aus 1,5 Billionen 1€-Münzen? Gib die Höhe in Kilometern (km) an.\\
	
	\textbf{Lösung:}
	
	1 mm = 0,001 m, 1 km = 1.000 m
	
	\begin{align*}
		1.500.000.000.000 \text{ mm} &= 1.500.000.000 \text{ m} \\
		&= \frac{1.500.000.000}{1.000} \text{ km} \\
		&= 1.500.000 \text{ km}
	\end{align*}
	
	\textbf{Antwort:} 1.500.000 km.\\
	
	\textbf{c)} Ein 50€-Schein hat eine Dicke von etwa 0,1 mm. Wenn der gesamte Betrag von 1,5 Billionen Euro in Form von 50€-Scheinen gestapelt würde, wie hoch wäre dann der Stapel? Gib die Höhe ebenfalls in Kilometern (km) an.\\
	
	\textbf{Lösung:}
	
	Anzahl der Scheine:
	\[
	\frac{1.500.000.000.000 \text{ €}}{50 \text{ € pro Schein}} = 30.000.000.000 \text{ Scheine}
	\]
	Höhe in mm:
	\[
	30.000.000.000 \text{ Scheine} \cdot 0,1 \text{ mm/Schein} = 3.000.000.000 \text{ mm}
	\]
	Höhe in km:
	\[
	3.000.000.000 \text{ mm} = 3.000.000 \text{ m} = 3.000 \text{ km}
	\]
	
	\textbf{Antwort:} 3.000 km.\\
	
\textbf{d)} Der Umfang unserer Erde entlang des Äquators beträgt ca. 40.000 km. Wie oft könnte man den Stapel aus 1€-Münzen bzw. den Stapel aus 50€-Scheinen um die Erde legen?\\
	
	\textbf{Lösung:}
	
	Für den 1€-Münzen-Stapel:
	\[
	\frac{1.500.000 \text{ km}}{40.000 \text{ km}} = 37,5
	\]
	
	Für den 50€-Scheine-Stapel:
	\[
	\frac{3.000 \text{ km}}{40.000 \text{ km}} = 0,075
	\]
	
	\textbf{Antwort:} Den Stapel aus 1€-Münzen könnte man 37,5-mal, den Stapel aus 50€-Scheinen 0,075-mal um die Erde legen.\\
	
	\vspace{1cm}
	\textit{Viel Erfolg beim Rechnen!}
	
\end{document}
