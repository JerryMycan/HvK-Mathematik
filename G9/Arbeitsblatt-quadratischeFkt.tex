\documentclass[a4paper,12pt]{article}

\usepackage{tabularx}
\usepackage{amsmath}
\usepackage[utf8]{inputenc}
\usepackage{multicol}
\usepackage{amsmath, amssymb, amsthm}
\usepackage{graphicx}
\usepackage{enumitem}
\usepackage{array}
\usepackage[left=2cm, right=2cm, top=2cm, bottom=2cm]{geometry}
\usepackage{fancyhdr}
\usepackage{xfp}
\usepackage{pgf}

\setlength{\headheight}{28pt}
\pagestyle{fancy}
\fancyhf{}
\fancyhead[L]{\includegraphics[height=1.2cm]{logo.png}}
\fancyhead[C]{\small Arbeitsblatt – Quadratische Funktionen \ (Kl. G9a)}
\fancyhead[R]{\small Name:\ \rule{2.8cm}{0.4pt}}
\fancyfoot[C]{Seite \thepage \enspace\textbullet\enspace J.\,Mycan \textcopyright~06.09.2017}

\renewcommand{\footrulewidth}{0.4pt}

\begin{document}
	
	\vspace*{2cm}
	Bearbeite das Arbeitsblatt ohne elektronische Hilfsmittel. 
	Lies jede Aufgabe sorgfältig, notiere eine passende Funktionsgleichung und begründe deine Rechnungen. 
	Zeige alle Zwischenschritte und markiere das Endergebnis deutlich. \\[1cm]
	
	% -------------------------------------------------
	% Aufgabe 1
	% -------------------------------------------------
	\textbf{Aufgabe 1}\\[0.2cm]
	Ein Springbrunnen erzeugt einen Wasserstrahl von maximal \(3\,\text{m}\) Höhe. 
	Im Abstand von \(2\,\text{m}\) von der Austrittsöffnung trifft er auf die Wasseroberfläche des Brunnens.\\
	Ina steht \(1{,}5\,\text{m}\) weit von der Austrittsöffnung entfernt und möchte mit einem Becher das Wasser auffangen.\\[0.1cm]
	In welcher Höhe muss sie den Becher halten, damit der Strahl den Becher trifft?
	
	\vspace{2cm}
	
	% -------------------------------------------------
	% Aufgabe 2
	% -------------------------------------------------
	\textbf{Aufgabe 2}\\[0.2cm]
	Beim Kugelstoßen beschreibt die Kugel eine parabelförmige Flugbahn. 
	Die Kugel verlässt die Hand des Kugelstoßers in einer Höhe von \(2\,\text{m}\) über dem Erdboden und erreicht nach 
	\(4\,\text{m}\) (horizontal vom Abwurfpunkt gemessen) ihre maximale Höhe von \(5{,}84\,\text{m}\).\\[0.2cm]
	\begin{enumerate}[label=\alph*)]
		\item Welche Weite hat der Kugelstoßer erzielt?\\[0.6cm]
		\item Wie weit vom Abwurfpunkt entfernt hat die Kugel eine Höhe von \(0{,}75\,\text{m}\)?
	\end{enumerate}
	
	\vspace{2cm}
	
	% -------------------------------------------------
	% Aufgabe 3
	% -------------------------------------------------
	\textbf{Aufgabe 3}\\[0.2cm]
	Gesucht werden zwei Zahlen, deren Differenz \(2\) ist. 
	Das Produkt der Zahlen soll einen möglichst kleinen Wert ergeben.\\[0.2cm]
	Welche Zahlen sind das? Begründe deine Antwort mit Hilfe einer quadratischen Funktion.
	
	\vfill
	
\end{document}
